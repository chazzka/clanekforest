%% 
%% Copyright 2007-2020 Elsevier Ltd
%% 
%% This file is part of the 'Elsarticle Bundle'.
%% ---------------------------------------------
%% 
%% It may be distributed under the conditions of the LaTeX Project Public
%% License, either version 1.2 of this license or (at your option) any
%% later version.  The latest version of this license is in
%%    http://www.latex-project.org/lppl.txt
%% and version 1.2 or later is part of all distributions of LaTeX
%% version 1999/12/01 or later.
%% 
%% The list of all files belonging to the 'Elsarticle Bundle' is
%% given in the file `manifest.txt'.
%% 
%% Template article for Elsevier's document class `elsarticle'
%% with harvard style bibliographic references
%\documentclass[3p,times,procedia,table]{elsarticle}
\documentclass[12pt]{article}

%% Use the option review to obtain double line spacing
%% \documentclass[preprint,review,12pt]{elsarticle}

%% Use the options 1p,twocolumn; 3p; 3p,twocolumn; 5p; or 5p,twocolumn
%% for a journal layout:
%% \documentclass[final,1p,times]{elsarticle}
%% \documentclass[final,1p,times,twocolumn]{elsarticle}
%% \documentclass[final,3p,times]{elsarticle}
%% \documentclass[final,3p,times,twocolumn]{elsarticle}
%% \documentclass[final,5p,times]{elsarticle}
%% \documentclass[final,5p,times,twocolumn]{elsarticle}

%% For including figures, graphicx.sty has been loaded in
%% elsarticle.cls. If you prefer to use the old commands
%% please give \usepackage{epsfig}

%% The amssymb package provides various useful mathematical symbols
\usepackage{amssymb}
\usepackage{svg}
\usepackage{siunitx}
\usepackage{amsmath}
\usepackage{amsthm}
\usepackage{tabularray}
\UseTblrLibrary{amsmath}
\UseTblrLibrary{siunitx}
\UseTblrLibrary{diagbox}
\usepackage{rotating}
\usepackage{mathtools}
\usepackage{pstricks}    %for embedding pspicture.
\usepackage{authblk}

\theoremstyle{definition}
\newtheorem{example}{Example}[section]
\newtheorem{definition}{Definition}[section]

%% The amsthm package provides extended theorem environments
%% \usepackage{amsthm}

%% The lineno packages adds line numbers. Start line numbering with
%% \begin{linenumbers}, end it with \end{linenumbers}. Or switch it on
%% for the whole article with \linenumbers.
%% \usepackage{lineno}


\begin{document}

%% Title, authors and addresses

%% use the tnoteref command within \title for footnotes;
%% use the tnotetext command for theassociated footnote;
%% use the fnref command within \author or \address for footnotes;
%% use the fntext command for theassociated footnote;
%% use the corref command within \author for corresponding author footnotes;
%% use the cortext command for theassociated footnote;
%% use the ead command for the email address,
%% and the form \ead[url] for the home page:
%% \title{Title\tnoteref{label1}}
%% \tnotetext[label1]{}
%% \author{Name\corref{cor1}\fnref{label2}}
%% \ead{email address}
%% \ead[url]{home page}
%% \fntext[label2]{}
%% \cortext[cor1]{}
%% \affiliation{organization={},
%%             addressline={},
%%             city={},
%%             postcode={},
%%             state={},
%%             country={}}
%% \fntext[label3]{}

\title{Isolation Forest in Novelty Detection Scenario}
\author[1]{Adam Ulrich}
\author[1]{Jak Krňávek} 
\author[1]{Roman Šenkreřík}
\author[1]{Zuzana Komínková Oplatková}
\author[1]{Radek Vala}
\affil[1]{Faculty of Applied Informatics, Tomas Bata University in Zlin}
%% \linenumbers

%% For citations use: 
%%       \citet{<label>} ==> Jones et al. [21]
%%       \citep{<label>} ==> [21]
%%

\maketitle
%% main text
\section{Introduction}
\label{sec:introduction}
Tady popíšeme obecně jak se snažíme detekovat anomalie, proč je to důležité, tisice citací na lidi kteří detekují anomalie, pak už víc k novelty detection, ukázat jak se to dělalo dřív co je cílem a proč to může být problem a nakonec uplně specificky co my chceme udělat

\section{Glossary}
\label{sec:theory}

Since this article is centered around anomaly detection algorithms, several key terms used throughout the article are introduced below.

\begin{itemize}
    \item \emph{Data point} refers to any observable data with \(n\) dimensions.
\item Regular point is a data point included in the given dataset. Its features are expected.
\item \emph{Anomaly} is a data point that differs significantly from other observations.
\item \emph{Outlier} is an anomaly included in the given dataset.
\item \emph{Novelty} is an anomaly that is not present in the given dataset during learning. Novelties are supplied later during evaluation.
\item \emph{Supervised} algorithm is an algorithm that is trained on all available data labels.
\item \emph{Unsupervised} algorithm is an algorithm that does not get any labels for the training data.
\item \emph{Semi-supervised} algorithm is an algorithm that is trained on data of only one class.

\end{itemize}

% \section{Methods - toto se smaže}
\label{sec:methods}
Traditional approaches for anomaly detection consist of either novelty
detection or outlier detection. Novelty detection is an anomaly
detection mechanism where we search for unusual observations 
discovered due to their differences from the training data. Novelty
detection is a semi-supervised anomaly detection technique, whereas
outlier detection uses unsupervised methods. With novelty detection, the
training data is not polluted by anomalous elements, and we are
interested in detecting whether a new observation is an anomaly. In this context, such points are also called novelties. This is a crucial
distinction. The outlier detection is usually presented with data
containing both anomalies and regular observation; it then uses
mathematical models that try to make a distinction between them. The
novelty detection, on the other hand, is usually presented with data with
little to zero anomalies (the proportion of anomalies in the dataset is
called a contamination), and later, when conferred with an anomalous
observation, it makes a decision.

Consider the following example: Figure \ref{fig:example0} contains random datapoints
arranged in a way they form a cluster-like shape. Say this data is our
regular observations.

\begin{figure}[htbp]
\centering
\includesvg[width=0.9\textwidth]{figures/example0_gnu.svg}
\caption{Example figure}
\label{fig:example0}
\end{figure}

When an unsupervised, outlier detection algorithm tries to analyze such
data, it sees the datapoints as a cluster containing both regular and
anomalous observations. Figure X shows the result of evaluating
classical Isolation Forest on such a dataset.

\begin{figure}[htbp]
\centering
\includesvg[width=0.9\textwidth]{figures/example0_5_gnu.svg}
\caption{Example figure}
\label{fig:example05}
\end{figure}

Figure \ref{fig:example05} shows regular observations \(x\) and anomaly observations \(y\)
marked by Isolation Forest
(\texttt{batch\_size\ 128,\ trees\_count:\ 100,\ zbytek\ default}).
Figure x shows that approx. \(10\%\) of observations are anomalies. This
is not unwanted behavior in the sense of outlier detection but is false
positive observation in the sense of novelty detection because this
data are regular observations that should be marked so.

Another problem is with the unsupervised separation itself. Consider data polluted by anomalies in close to \(1:1\) ratio. Finding the line
itself is evident. Deciding which observations are anomalies without some domain knowledge, on the other hand, is close to impossible.

\section{Graph Theory}
\label{sec:graph_theory}
Tady budeme muset popsat odstaveček k teorii grafů a ujasnit si pojmy, které budeme používat napříč celým článkem

slova co jsme použili:

\begin{itemize}
    \item graph
    \item vertex
    \item edges
    \item depth??
    \item tree
    \item associated with - Each vertex \(R\) is associated with two new edges
\end{itemize}

vertex, depth, path length, root, leaf...

\section{Isolation Forest}
\label{sec:isolation_forest}

Isolation Forest
\cite{liu2008isolation} \cite{liu2012isolation} is an outlier
detection, unsupervised ensemble algorithm. This approach is well known for successfully identifying outliers by using recursive partitioning (forming a tree-like structure) to decide whether the analyzed datapoint is an anomaly or not. The fewer partitions required to isolate, the more probable it is for a datapoint to be an anomaly.

\paragraph{Isolation tree}
Isolation Tree is a rooted tree constructed with a subset of \(A\) items
(datapoints) with the size \(s=|A|\).

\begin{enumerate}
    \item To build an isolation tree, it is not necessary to have a large set; it
  may even be undesirable
  \item Well-chosen small \(s\) can help eliminate \emph{masking} and
  \emph{swamping}


\begin{itemize}
    \item \textbf{Masking} When the number of anomalies is high, it is possible that some of those aggregate in a dense and large cluster, making it more difficult to separate the single anomalies and, in turn, to detect such points as
  anomalous.
    \item \textbf{Swamping} When normal instances are too close to anomalies, the number of
  partitions required to separate them increases, making it more difficult for the Isolation Forest to discriminate between
  anomalies and normal points.
\end{itemize}

\item There are two types of vertices

\begin{itemize}
    \item \textbf{Internal vertex} Internal vertex contains a condition (a feature and a limit) and two children (one
  representing a fulfilled condition and the other an unfulfilled one)
    \item \textbf{External vertex} External vertex is created if the conditions of the parents are met (or not met) by
  one or none of the elements from the sample or the maximum depth of
  the tree \(l\) is reached, usually \(l=\ln_2(s)\). It contains the
  evaluation of \(h(x)\) using the distance from the root. If the max
  length of the tree is reached the \emph{distance} is estimated using
  \(h(x)=e+c(n)\), where \(e\) is the distance from the root, \(n\) is
  the number of elements from the sample satisfying the conditions of
  the parents, \(c(n)=2\,(H_{n-1}-\frac{n-1}{n})\) and \(H_{n-1}\) is
  \(n-1\) harmonic number.
\end{itemize}

\end{enumerate}

\begin{definition}
Let $\mathsf{s}=(s_0, \dots, s_d, \dots, s_{n-1})\in \mathbb{R}^n$ be a datapoint. Then we say that $\pi_d(\mathsf{s}):=s_d$ is a \emph{projection} of datapoint $\mathsf{s}$ onto dimension $d$  yielding $s_d$.
\end{definition}

\begin{definition}
Let $Z$ be a finite subset of $\mathbb{R}^n$:
$$Z \subseteq \mathbb{R}^n ;\quad n \in \mathbb{N}.$$

For each dimension \(d \in\{0, \dots, n - 1\}\), let
$$Z_d = \{ \pi_d(\mathsf{s})\ |\ \mathsf{s} \in Z \}.$$

Then we define hyperrectangle $R(Z)$ \emph{surrounding} $Z$, such that
$$R(Z) = r_0 \times r_1 \times \cdots \times r_{n-1},$$ where $r_d = \langle \min Z_d, \max Z_d \rangle$ for each $d \in \{0,1, \dots, n-1\}.$

\end{definition}


\subsection{Constructing decision tree $T$}

\begin{enumerate}

    \item Each leaf or internal vertex is a hyperrectangle $R$. 
    \item Function relation $\rho$ assigns each internal vertex $v$ to split point $z$ and dimension $d$.
    $$(v, (z,d)) \in \rho$$
    \item The maximum possible height of a tree is controlled by the \emph{max\_depth} parameter.
    \item The sample \(S\) contains \emph{b} number of input datapoints of $n$ dimensions, that is
    $$S \subseteq \mathbb{R}^n ;\quad |S| = b; \quad b \in \mathbb{N}_0.$$
    \item Leaves' ending condition is satisfied by one of two criteria: 
\begin{enumerate}
    \item Vertex \(R\) satisfies \(| S \cap R | = 1\).
    \item Vertex's depth reached the \emph{max\_depth} value.
\end{enumerate}
\end{enumerate}


\subparagraph{Basis step}

The trivial rooted tree \(T_0\) is a tuple with
vertices \(V_0 = \{R(S)\}\) and edges \(E_0 = \emptyset\), i.e. 
\[T_0= (V_0, E_0) = (\{R(S)\},\emptyset).\]

The function relation $\rho_0$ initially has no assignments, i.e.
$$\rho_0 = \emptyset.$$

\subparagraph{Recursive step}

The steps to reach the tree \(T_{j+1}\) from \(T_{j} = (V_j, E_j)\) and $\rho_{j+1}$ from $\rho_{j}$ are
as follows:

Let \(L_j \subseteq V_j\) be a subset of leaves not satisfying the
ending condition. 

By selecting a random dimension $d_R$,(TODO: dimenze taková aby $r_d$ nebyla $\langle a,a\rangle$)   create two new vertices \(R_l, R_r\) for each leaf \(R \in L_j\)\
\[R =  r_0 \times \cdots \times r_{d_R} \times \cdots \times r_{n-1}  \tag{xx}\,\] with a random split point $z_R \in r_{d_R}$.

Split point $z_R$ splits the $R \cap S$, creating two disjunctive sets, $S_l$ and $S_r$ respectively:

$$S_l = \{ \mathsf{s} \in{R \cap S}\ |\ \pi_{d_R}(\mathsf{s})\le z_R\}$$

$$S_r = \{ \mathsf{s} \in{R \cap S}\ |\ \pi_{d_R}(\mathsf{s}) > z_R\}$$

Then we obtain the left and right hyperrectangles \(R_l\), \(R_r\) as
follows:
$$R_l = R(S_l),$$
$$R_r = R(S_r).$$

Each vertex \(R\) is associated with two new
edges \((R,R_l ), (R, R_r)\) and is assigned with $(z_R,d_R)$ by function relation $\rho_{j+1}$ as in (x\ldots xxx).


\begin{align}
   \rho_{j+1} &= \rho_j \cup \bigcup_{R \in L_j} \{(R, (z_R, d_R))\}, \\
   V_{j+1} &= V_j \cup \bigcup_{R \in L_j} \{R_l, R_r\}, \\
   E_{j+1} &= E_j \cup \bigcup_{R \in L_j} \{(R, R_l), (R,R_r)\},\\
   T_{j+1} &= (V_{j+1}, E_{j+1})
\end{align}
i.e.~${R_l, R_r} \subset R$ are leaves and $R$ is an inner vertex in the new tree
\(T_{j+1}\).\footnote{Tree \(T_{j+1}\) is actually a Hasse diagram of the ordered set
\((V_{j+1},\subseteq)\)}


The algorithm moves to the next recursion step unless there is an equality of two consecutive trees \(T_j = T_{j+1}\). This happens when all leaves satisfy their ending condition, i.e., \(L_j = \emptyset\).
Thus, the desired tree $T$ is the tree $T_j$, and finite relation $\rho$ is $\rho_j$.


\begin{example}
\label{example:original_tree_create}
Consider now an example of creating a new Isolation tree based on the given input sample $S$ as shown in Figure \ref{fig:example_noutlier_gnu}.

\begin{align*}
    S = \{&[25,100],[30,90],[20,90],[35,85],\\
    &[25,85],[15,85],[105,20],[95,25], \\
    &[95,15],[90,30],[90,20],[90,10]\}
\end{align*}


\begin{figure}[htbp]
\centering
\includesvg[width=0.9\textwidth, inkscapelatex=false]{figures/example66_Noutlier_gnu.svg}
\caption{Popisek}
\label{fig:example_noutlier_gnu}
\end{figure}

After selecting \emph{max depth} of 8 (experimentally), the tree $T$ is created by starting with tree $T_0$ and expanding further as described by the recursive step until the final condition is met.

Figure \ref{fig:example_noutlier_gnu} shows the finished tree $T$, learned on dataset $S$.

\begin{enumerate}
    \item Basis step is to create a tree $T_0$ and a function relation $\rho_0$. 
    Observe that $E_0$ and $\rho_0$ are empty because there is just a single root node without any connections.
\begin{align*}
R &= R(S) = \langle 15, 105 \rangle \times \langle 10, 105 \rangle\\
V_0 &= \{R\}\\
E_0 &= \emptyset\\
T_0 &= (V_0, E_0)\\
\rho_0 &= \emptyset
\end{align*}    

\item In order to reach $T_1$ from $T_0$, dimension $d_R=1$ and split point $z_R = 72.63$ were chosen. The recursive step is as follows:

$$S_l = \{[105,20],[95,25],[95,15],[90,30],[90,20],[90,10]\}$$
$$S_r = \{[25,100],[30,90],[20,90],[35,85],[25,85],[15,85]\}$$

$$R_l = R(S_l) = \langle 90, 105 \rangle \times \langle 10, 30 \rangle$$
$$R_r = R(S_r) = \langle 15, 35 \rangle \times \langle 85, 100 \rangle$$
\begin{align*}
\rho_1 &= \{ (R, (72.63, 1))\}\\
V_1 &= \{R, R_l, R_r\}\\
E_1 &= \{(R,R_l), (R,R_r)\}\\
T_1 &= (V_1, E_1)
\end{align*}
\item The tree $T_1$ has two vertices and two edges; the ending condition is not met, that is $L_1 = \{R_l,R_r\}$. The tree $T_1$ now has a left and right vertex. We continue the recursive step with two vertices.

Left vertex $R_l$:
\begin{align*}
z_{R_l} &= 103.08\\
d_{R_l} &= 0\\
S_{ll} &= \{[95,25],[95,15],[90,30],[90,20],[90,10]\}\\
S_{lr} &= \{[105,20]\}\\
R_{ll} &= \langle 90, 95 \rangle \times \langle 10, 30\rangle\\
R_{lr} &= \langle 105, 105 \rangle \times \langle 20, 20\rangle
\end{align*}

Right vertex $R_r$:
\begin{align*}
z_{R_r}&= 20.32\\
d_{R_r}&= 0\\
S_{rl}&= \{[20,90],[15,85]\}\\
S_{rr}&= \{[25,100],[30,90],[35,85],[25,85]\}\\
R_{rl}&= \langle 15, 20 \rangle \times \langle 85, 90 \rangle\\
R_{rr}&= \langle 25, 35 \rangle \times \langle 85, 100 \rangle
\end{align*}
\item With the hyperrectangles prepared, we can assemble new vertices and edges and create a new $T_2$:
\begin{align*}
\rho_2 &= \{(R,(72.63,1), (R_l, (103.08, 0)), (R_r, (20.32, 0)) \}\\
V_2 &= \{ R, R_l, R_r, R_{lr}, R_{lr}, R_{rl}, R_{rr} \}\\
E_2 &= \{ (R,R_l),(R,R_r), (R_l,R_{ll}), (R_l,R_{lr}), (R_r,R_{rl}), (R_r,R_{rr}) \}\\
T_2 &= (V_2, E_2)
\end{align*}
\item With the $T_2$ created, we now have to check for leaves' ending condition. Since $|R_{lr}\cap S|=|S_{lr}| = 1$, the ending condition for leaf $R_{lr}$ is met, and the new set of leaves $L_2$ for the next recursion step is
$$L_2 = \{R_{ll},R_{rl}R_{rr}\}.$$

This continues until we reach tree $T_5$ such that $T_5=T_6$, which is the desired tree $T$ as shown in Figure \ref{fig:example_noutlier_tree_color}. 

\end{enumerate}

\end{example}


\subsection{Evaluating decision tree $T$}
The evaluation of desired element $a$ starts in the root $R$ of previously built tree $T$.
In a root, by applying function relation, $\rho(R) = (z,d)$, we obtain split point $z$ and dimension $d$.
The root $R$ of a tree $T$ has two ancestors $R_l$, $R_r$, such that
$\forall r_l\in R_l; \pi_d(r_l) \le z$ and $\forall r_r\in R_r; \pi_d(r_r)  > z$.

If $\pi_d(a)\le z$, we traverse to $R_l$, else, that is $\pi_d(a) > z$, we traverse to $R_r$.
We continue in this manner up until we reach the leaf, determined by the ending condition. The score of $a$ is calculated using the depth of the reached leaf.

\begin{example}
\label{ex:regular_point_evaluation_original}
    Consider now the evaluation of $a = [105,20]$ on tree $T$ built in Example \ref{example:original_tree_create}.

    We start with the root $R = \langle 15,105\rangle \times \langle 10, 100 \rangle$.
    By applying function $\rho(R)$, we obtain split point $z = 72.63$ and dimension $d = 1$.
    Root $R$ has two ancestors 
\begin{align*}
    &R_l = \langle 90,105\rangle \times \langle 10, 30 \rangle,&
    &R_r = \langle 15,35\rangle \times \langle 85, 100 \rangle,\\
    \intertext{such that}
    &\forall r_l\in R_l; \pi_1(r_l) \le 72.63,&
    &\forall r_r\in R_r; \pi_1(r_r) > 72.63.
\end{align*}
Now, by applying projection $\pi_1$ on $a$, we obtain $20$, which is less than split point $z = 72.63$, i.e.
$$\pi_1([105,20]) = 20 < 72.63.$$
We visit the vertex $R_l$ because the value obtained by applying the projection on any element of $R_l$ is smaller than $72.63$.

We reached the next recursive step. With vertex $R_l$ visited, we apply function $\rho(R_l)$, obtaining a new split point $z = 103.08$ and new dimension $d = 0$.
Vertex $R_l$ has two ancestors 
\begin{align*}
    &R_{ll} = \langle 90,95\rangle \times \langle 10, 30 \rangle,&
    &R_{lr} = \langle 105,105\rangle \times \langle 20,20 \rangle,\\
    \intertext{such that}
    &\forall r_{ll}\in R_{ll}; \pi_1(r_{ll}) \le 103.08,&
    &\forall r_{lr}\in R_{lr}; \pi_1(r_{lr}) > 103.08.
\end{align*}

We visit the vertex $R_{lr}$ because the value obtained by applying the projection on any element of $R_{lr}$ is less than $103.08$.

Since $R_{lr}$ has no more ancestors, we reached the final leaf with a depth of 2.
%%TODO: muzeme tu rict ze, vlasnte u tohoto prikladu se dalo normalne divat na hodnoty S_l a S_r a jit tam kde padneme, to ale nepujde vzdy
\end{example}


\begin{example}
\label{ex:novelty_point_evaluation_original}
    Consider now the evaluation of $a' = [25,20]$ on tree $T$ built in Example \ref{example:original_tree_create}.

    We start with the root $R = \langle 5,105\rangle \times \langle 10, 100 \rangle$.
    By applying function $\rho(R)$, we obtain split point $z = 72.63$ and dimension $d = 1$.
    Root $R$ has two ancestors 
\begin{align*}
    &R_l = \langle 90,105\rangle \times \langle 10, 30 \rangle,&
    &R_r = \langle 15,35\rangle \times \langle 85, 100 \rangle,\\
    \intertext{such that}
    &\forall r_l\in R_l; \pi_1(r_l) \le 72.63,&
    &\forall r_r\in R_r; \pi_1(r_r) > 72.63.
\end{align*}
Now, by applying projection $\pi_1$ on $a'$, we obtain $20$, which is less than split point $z = 72.63$, i.e.
$$\pi_1([25,20]) = 20 \le 72.63.$$
We visit the vertex $R_l$ because the value obtained by applying the projection on any element of $R_l$ is less than $72.63$.

We reached the next recursive step. With vertex $R_l$ visited, we apply function $\rho(R_l)$, obtaining a new split point $z = 103.08$ and new dimension $d = 0$.
Vertex $R_l$ has two ancestors 
\begin{align*}
    &R_{ll} = \langle 90,95\rangle \times \langle 10, 30 \rangle,&
    &R_{lr} = \langle 105,105\rangle \times \langle 20,20 \rangle,\\
    \intertext{such that}
    &\forall r_{ll}\in R_{ll}; \pi_0(r_{ll}) \le 103.08,&
    &\forall r_{lr}\in R_{lr}; \pi_0(r_{lr}) > 103.08.
\end{align*}

We visit the vertex $R_{ll}$ because the value obtained by applying the projection on any element of $R_{ll}$ is less than $103.08$.

This continues recursively until the leaf $R_{llllr}$ is reached. The reached depth is 5, as shown in Figure \ref{fig:example_noutlier_tree_color} (marked gray).

\end{example}


Note that each element that was part of the batch during the training --- tree building --- phase is always contained in each vertex it visits.
See $a \in R_{lr} \subset R_{l} \subset R$ in Example \ref{ex:regular_point_evaluation_original}.
Note that this is not true for elements that were unseen during the training phase (such as novelty points).
See $a' \in R$, but $a' \notin R_l$ (and of course $a' \notin R_r$) in Example \ref{ex:novelty_point_evaluation_original}.


\section{Proposed Range-based Enhancement For Isolation Forest}
\label{sec:novelty_isolation_forest}
In this section, we propose a new enhancement of the original Isolation Forest algorithm to make it possible to detect novelty observations.
The proposed enhancement takes the basic idea of an ensemble of trees with various depths but takes it further to make semi-supervised novelty detection possible.

\subsection{Initial Problem}
\label{sec:InitialProblem}
 The standard Isolation Forest algorithm cannot be used for novelty detection. This is because, in each step, the observation space is limited by the previous data.


Suppose we now want to use the original tree to evaluate the novelty data point $p$, which is not present in the training set.

Let us recall the equations needed for vertex creation:
\begin{align*}
S_l &= \{ \mathsf{s} \in{R \cap S}\ |\ \pi_{d_R}(\mathsf{s})\le z_R\},&
S_r &= \{ \mathsf{s} \in{R \cap S}\ |\ \pi_{d_R}(\mathsf{s}) > z_R\}.
\end{align*}
The point $p$ fits neither $S_l$ nor $S_r$ and does not even necessarily fit $R_l = R(S_l)$ nor $R_r = R(S_r)$. 

Nevertheless, because datapoint $p$ satisfies  $\pi_d(p) \le z$ (or $\pi_d(p) > z$), $p$ is assigned to a vertex $R_l$ (or $R_r$) even though $p \notin R_l$ nor $p \notin R_r$.

This is why the original Isolation Forest is a purely unsupervised algorithm. To work properly, the tree has to be constructed concerning all possible input data.

\subsection{Proposed Solution}
The proposed solution comes from the idea that the original tree lacks the possibility to isolate more data points than it currently observes.
The observed space is bounded by the minimum and maximum in each feature.

As in the original article, we use the concept of a binary decision tree. The proposed solution is altering the concept of the split point evaluation. Whereas the original Isolation Forest evaluates the split point based on the previous data, we
evaluate the split point based on a range in our proposed solution. For this to work, several alterations to the split point evaluation and the form of data passed between vertices must be made; however, the overall concept of the forest remains the same.
The proposed solution has several concepts of alteration to the original solution.

\begin{enumerate}
    \item We start with the root as the whole domain space bounded by ranges.
    \item Descendants cover the whole observable space of their associated ancestors. 
    \item The split point is in the middle of the given dimension’s range.
    \item The input data is only used to determine the ending condition.
\end{enumerate}

\subsection{Constructing decision tree $T$}

\begin{enumerate}
    % \item The maximum possible depth of a tree is controlled by the \emph{max\_depth} parameter.
    \item The sample \(S\) is nonempty set of input datapoints.
    \item Leaves and internal vertices are possibility-space hyperrectangles. 
    \item Hyperrectangle $R$ satisfies the ending condition when \(S \cap R = \emptyset\) or \(| S \cap R | = 1\).
\end{enumerate}



\subparagraph{Basis step}

Each dimension \(d \in\{0, \dots, n-1\}\), is bounded by the range \(r_d\). The ranges form the possibility-space hyperrectangle \(R_0\), i.e.

\[R_0 =  r_0 \times r_1 \times \cdots \times r_{n-1}.\]

The trivial rooted tree \(T_0\) is a tuple with
vertices \(V_0 = \{R_0\}\) and edges \(E_0 = \emptyset\), i.e.
\[T_0= (V_0, E_0) = (\{R_0\},\emptyset).\]

\subparagraph{Recursive step}

The steps to reach the tree \(T_{j+1}\) from \(T_{j} = (V_j, E_j)\) are
as follows:

Let \(L_j \subseteq V_j\) be a subset of leaves not satisfying the
ending condition.
For each leaf \(R \in L_j\) we
select a random dimension \(d\).

Let
\begin{align}
R &= r_0 \times  \cdots \times r_{d-1} \times  r_d\times r_{d+1} \times \cdots \times r_{n-1},
\end{align}
where $r_d = \langle x, y )$.

Then we obtain the left and right hyperrectangles \(R_l\), \(R_r\) as
follows
\begin{align}
R_l &= r_1 \times  \cdots \times r_{d-1} \times  \langle x, s ) \times r_{d+1} \times \cdots \times r_n, \\
R_r &= r_1 \times  \cdots \times r_{d-1} \times \langle s, y ) \times r_{d+1} \times \cdots \times r_n,
\end{align}
where \(s\) is
a number obtained as the middle of the range \(r_d\,\)

\[s = \frac{x + y}{2}\,.\] Each vertex \(R\) is associated with two new
edges \((R,R_l ), (R, R_r)\) as follows
\begin{align*}
V_{j+1} &= V_j \cup \bigcup_{R \in L_j} \{R_l, R_r\},\\
E_{j+1} &= E_j \cup \bigcup_{R \in L_j} \{(R, R_l), (R,R_r)\},\\
T_{j+1} &= (V_{j+1}, E_{j+1}),
\end{align*}
i.e.~${R_l, R_r} \subset R$ are leaves in the new tree
\(T_{j+1}\).

Recursion is terminated if there is an equality of two consecutive trees \(T_k = T_{k+1}\). This happens when all leaves satisfy the ending condition, i.e., \(L_k = \emptyset\).
If this is the case, the desired tree $T$ is the tree $T_k$; otherwise, move to the next recursion step.


Note that tree \(T_{j}\) is actually a Hasse diagram of the ordered set
\((V_j,\subseteq)\).


\begin{example}
\label{example:novelty_tree_create}
Consider now an example of creating a new enhanced Isolation tree based on the given input sample
\begin{align*}
    S = \{&[25,100],[30,90],[20,90],[35,85],\\
    &[25,85],[15,85],[105,20],[95,25], \\
    &[95,15],[90,30],[90,20],[90,10]\}.
\end{align*}
Figure \ref{fig:example_novelty_gnu} shows $S$ on the finished hyperplane created using an enhanced approach. Observe that the split points (red lines) are always in the middle of the previous observable space, and hyperrectangles now always cover the whole ancestor's area. Numbers represent the final leaves' depth (also seen in Figure \ref{fig:example_novelty_tree_color}.

\begin{figure}[htbp]
\centering
\includesvg[width=0.9\textwidth,inkscapelatex=false]{figures/example66_Novelty_gnu.svg}
\caption{Enhanced approach. Rectangles created by recursive splitting.}
\label{fig:example_novelty_gnu}
\end{figure}

 The tree $T$ is created by starting with tree $T_0$ and expanding further as described by the recursive step until the final condition is met.

Figure \ref{fig:example_novelty_tree_color} shows the finished tree $T$, learned on dataset $S$.


    \paragraph{Basis step} Creation of tree $T_0$ with one root vertex $R$ with experimentally set initial possibility space  $R= \langle 0,110) \times \langle -5,105)$ and no edges $E_0$, such that
    \begin{align*}
        V_0 &= \{R\},&
        E_0 &= \emptyset,&
        T_0 &= (V_0, E_0).
    \end{align*}
     Since $R \in L_0$, we create two hyperrectangles $R_l$, $R_r$ by selecting a random dimension $d=0$.
    \begin{align*}
        r_0 &= \langle 0, 110), &
        s &= \frac{0 + 110}{2} = 55, \\
        R_l &= \langle 0, 55) \times \langle -5,105), &
        R_r &= \langle 55, 110) \times \langle -5,105).
    \end{align*}
     New tree $T_1$ is then
    \begin{align*}
    V_1 &= \{R, R_l, R_r\}, &
    E_1 &= \{(R, R_l), (R, R_r)\}, &
    T_1 &= (V_1, E_1).
    \end{align*}
    After checking for the ending condition, set $L_1$ is $\{R_l, R_r\}$.

    \paragraph{Since there are vertices in $L_1$ left to be examined, we continue the next recursive step}
    Since $R_l \in L_1$ (resp. $R_r \in L_1$), we create two hyperrectangles $R_{ll}$, $R_{lr}$ -- left column (resp. $R_{rl}$, $R_{rr}$ -- right column) by selecting a random dimension $d=1$ (resp. $d=0$).
    \begin{align*}
        r_{l1} &= \langle -5, 105)& r_{r0} &= \langle 55, 110) \\
        s_l &= 50 & s_r&=82.5\\
        R_{ll} &= \langle 0, 55) \times \langle -5,50) & R_{rl} &= \langle 55, 82.5) \times \langle -5,105)\\
        R_{lr} &= \langle 0, 55) \times \langle 50,105) & R_{ll} &= \langle 82.5, 110) \times \langle -5,105)
    \end{align*}
 New tree $T_2$ is then
    \begin{align*}
        V_2 &= \{R, R_l, R_r, R_{ll}, R_{lr}, R_{rl}, R_{rr}\} \\
        E_2 &= \{(R, R_l), (R, R_r), (R_l, R_{ll}), (R_l, R_{lr}), (R_r, R_{rl}), (R_r, R_{rr})\} \\
        T_2 &= (V_2, E_2)
    \end{align*}
 We check the ending condition again, getting $L_2 = \{R_{lr}, R_{rr}\}$.

    This goes on until the final condition is met.
\end{example}

\subsection{Evaluating enhanced decision tree $T$}
The evaluation of our enhanced decision tree is more straightforward since the examined datapoint is always contained in the possibility space hyperrectangle of some vertex in each depth until a leaf is visited.
The evaluation starts in the root vertex. The initial possibility space should be reasonable enough to cover the whole domain of a given problem.
Until the leaf is reached, the examined point recursively visits the descendant within which it is contained.



\begin{example}
\label{ex:regular_point_evaluation_novelty}
    Consider now the evaluation of $a = [105,20]$ on tree $T$ built in Example \ref{example:novelty_tree_create}.

\begin{enumerate}
    \item  We start with the root $R = \langle 0,110\rangle \times \langle -5, 105 \rangle$.
    Root $R$ has two descendants 
\begin{align*}
    &R_l = \langle 0,55) \times \langle -5, 105),&
    &R_r = \langle 55,110) \times \langle -5, 105),
\end{align*}
Since $a \in R_r$, we visit $R_r$.
\item Vertex $R_r$ has two descendants
\begin{align*}
    &R_{rl} = \langle 5,82.5) \times \langle -5, 105),&
    &R_{rr} = \langle 82.5,110) \times \langle -5, 105),
\end{align*}
Since $a \in R_{rr}$, we visit vertex $R_{rr}$.
\item
we continue recursively in this manner for another two steps until leaf $R_{rrlr}$ is reached. This leaf has a depth of $4$.

\end{enumerate}
   
\end{example}

\begin{example}
\label{ex:novelty_point_evaluation_novelty}
    Consider now the evaluation of $a' = [25,20]$ on tree $T$ built in Example \ref{example:novelty_tree_create}.

\begin{enumerate}
    \item  We start with the root $R = \langle 0,110\rangle \times \langle -5, 105 \rangle$.
    Root $R$ has two descendants 
\begin{align*}
    &R_l = \langle 0,55) \times \langle -5, 105),&
    &R_r = \langle 55,110) \times \langle -5, 105),
\end{align*}
Since $a' \in R_l$, we visit $R_l$.
\item Vertex $R_l$ has two descendants
\begin{align*}
    &R_{ll} = \langle 0,55) \times \langle -5, 50),&
    &R_{lr} = \langle 0,5) \times \langle 50, 105),
\end{align*}
Since $a' \in R_{ll}$, we visit vertex $R_{ll}$.
\item
Since $R_{ll}$ is a leaf, we end here, and the reached leaf's depth is $2$.
\end{enumerate}
\end{example}
Note that each evaluated point of the given possibility space is always contained in each vertex it visits.
Hence $a \in R_{rrlr} \subset R_{rrl} \subset R_{rr} \subset R_{r} \subset R$ in \ref{ex:regular_point_evaluation_novelty}
and $a' \in R_{ll} \subset R_{l} \subset R$ as shown in Example \ref{ex:novelty_point_evaluation_novelty}.


%z tadyka jsem smazal puvodni evaluaci, je v evaluate_old.tex



\begin{figure}[htbp]
\centering
\includesvg[angle=90,inkscapelatex=false,width=1\textwidth]{figures/example66_Novelty_tree.svg}
\caption{Tree constructed using the enhanced novelty approach}
\label{fig:example_novelty_tree_color}
\end{figure}




\section{Expected value of depth}
One conclusive method to evaluate both algorithms for subsequent comparison is calculating the expected value of depth. It takes individual points and calculates the probability that a point in a given algorithm will reach that particular depth. It is then possible to compare the expected value of depth of the individual points with each other and see how different they are, giving a scale of abnormality.

For the calculations, we consider the points from the examples above.

\subsection{Original approach}
To show the resulting depth using the original approach, we find all possible paths that would isolate the given point. Then, for each of the possible paths, we calculate its probability.

% Node(Union{Number, Tuple{Vararg{T, N}} where {N, T}}[(25, 100)], BigRationals.BigRational(1,
% 179159040), 7, @NamedTuple{dim::Int64, range::UnitRange{Int64}}[(dim = 1, range = 95:105), (dim = 2, range = 10:15), (dim = 1, range = 90:95), (dim = 2, range = 20:30), (dim = 1, range = 30:35), (dim = 1, range = 15:20), (dim = 2, range = 90:100)])

% na zacakatu name X moznosti jak vybrat splitpoint

% pro nejaky bod vam ukazeme jak se pocita pravedpodobnost
% julie nam ukaze split dehpth (12 SP 345)
% pravepodobnost prvniho splitpointu (geometricka) sprangemax-sprangemin/rangemax - rangemin
% viz tabule 1/2* cosi minus cosi, jak se pocita je v julii
% tim poslednim split pointem je bod tedy izolovan
% jeste to ukazeme pro novelty bod, ukazeme ze zustane v hrnicku s bodem nahore a nebo s bodem vpravo dole
% tim ukazeme pravdepodobnost pro jeden pripad, je toho moc tak viz tabulka
% Node.new(data => $($[25, 85],), prob => <11/18662400>, depth => 7, split => [0 => 95..105, 0 => 90..95, 1 => 85..90, 1 => 10..20, 0 => 35..90, 0 => 25..35, 0 => 15..25])
% 15-105
% 15-20-25-30-35-90-95-105

For the point $[25,85]$, consider one of the possible paths that orphan the given point.

We start with the whole observable space, a range of $\langle 15, 105\rangle$ for the first dimension $x$ and $\langle 10,100\rangle$ for the dimension $y$.
First, the dimension $x$ and a split point in the $\langle 95, 105\rangle$ range were chosen (the split point is a random value from this interval). The possible range where the split point could have been chosen was $\langle 15,105\rangle$.
The probability of selecting $x$ is $0.5$ due to the two-dimensional setting.
The probability of the split point being from the given range is $\frac{105-95}{105-15}$, where the nominator is the size of the favourable range, and the denominator is the size of the whole possible range.
% Let $p(\langle 95, 105\rangle_x)$ be the probability of $n$ splits.
The probability of such event is then $$\frac{1}{2}\cdot\frac{105-95}{105-15} = \frac{1}{18}.$$
Since the given point is not yet orphaned, we continue this way. The observable space is now scaled down due to the split point to the range of $\langle 15, 95\rangle$ for the first dimension $x$ and $\langle 10,100\rangle$ for the second $y$. Table \ref{prob_table_25_85} shows the rest of the probabilities for this path.

\begin{table}[!t]
\centering
\caption{Probabilities of depths for point $[25,85]$.}
\label{prob_table_25_85}
\resizebox{\columnwidth}{!}{%
\begin{tblr}{
    width=\linewidth,
    hspan=minimal,
    cells={font=\footnotesize},
    colspec={c c c c c},
    %colsep=1pt,
    row{1}={guard},
    column{1-5}={mode=math}
}
Start space & Dim. & Split range & Probability & End space \\
\hline
\langle 15, 105\rangle \times \langle 10, 100\rangle & x & \langle 95, 105\rangle &  \frac{1}{2}\cdot\frac{105-95}{105-15} = \frac{1}{18} & \langle 15, 95\rangle \times \langle 10, 100\rangle \\
\langle 15, 95\rangle \times \langle 10, 100\rangle & x & \langle 90, 95\rangle &  \frac{1}{2}\cdot\frac{95-90}{95-15} = \frac{1}{32} & \langle 15, 90\rangle \times \langle 10, 100\rangle \\
\langle 15, 90\rangle \times \langle 10, 100\rangle & y & \langle 85, 90\rangle &  \frac{1}{2}\cdot\frac{90-85}{100-10} = \frac{1}{36} & \langle 15, 90\rangle \times \langle 10, 85\rangle \\
\langle 15, 90\rangle \times \langle 10, 85\rangle & y & \langle 10, 20\rangle &  \frac{1}{2}\cdot\frac{20-10}{85-10} = \frac{1}{15} & \langle 15, 90\rangle \times \langle 20, 85\rangle \\
\langle 15, 90\rangle \times \langle 20, 85\rangle & x & \langle 35, 90\rangle &  \frac{1}{2}\cdot\frac{90-35}{90-15} = \frac{11}{30} & \langle 15, 35\rangle \times \langle 85, 85\rangle \\
\langle 15, 35\rangle \times \langle 85, 85\rangle & x & \langle 25, 35\rangle &  \phantom{\frac{1}{2}\cdot}\frac{35-25}{35-15} = \frac{1}{2} & \langle 15, 25\rangle \times \langle 85, 85\rangle \\
\langle 15, 25\rangle \times \langle 85, 85\rangle & x & \langle 15, 25\rangle &  \phantom{\frac{1}{2}\cdot}\frac{25-15}{25-15} = \frac{1}{1} & \langle 25, 25\rangle \times \langle 85, 85\rangle
\end{tblr}
}
\end{table}


% cela cesta je pak takto: - vypisu ty vypocty
% cela pravdepodobnost je pak ze to cele vynasobim
% a reknu aha path je tedy 7 protoze to dal neslo

Note that rows $6$ and $7$ no longer contain the probability of selecting the dimensions since the second dimension cannot be chosen as it would not isolate any point (see the startspace's $y$ as $\langle 85, 85\rangle$).
The evaluation ends after seven splits (depth = $7$) since that is the last split to isolate the given point.
The probability of this case is then $\frac{1}{18}\cdot\frac{1}{32}\cdot\dots\cdot\frac{1}{2}\cdot 1$.

If we do this for all possible cases, we get the probabilities for the individual depths that could result in orphaning the given point. The first row in Table \ref{table_big_original} shows the values for all possible depths for the point $[25,100]$. 
The rest of the rows of this table show the probabilities of depth for the remaining points.
This is later used to calculate the expected value of depth.
Due to the symmetricity, the probabilities for the points in the bottom right corner in Figure \ref{fig:example_noutlier_gnu} are the same as those in the top left corner, so we only show the latter.

Let us recall the initial problem in Section \ref{sec:revisiting}.
If we consider the novelty point $[25,20]$, the evaluation results in the same path as the point $[25,85]$ in Table \ref{prob_table_25_85}. However, the novelty point no longer fits in the start space for the sixth and seventh rows. Nevertheless, because the point $[25,20]$ satisfies $25 \le z \in \langle 25, 35 \rangle$,  $[25,20]$ is assigned to the vertex $\langle 15, 25 \rangle \times \langle 85, 85 \rangle$ even though $[25,20] \notin \langle 15, 25 \rangle \times \langle 85, 85 \rangle$.

Note that the path for the novelty point $[25,20]$ is a path of the $[25,85]$ (the vertical neighbour in Figure \ref{fig:example_noutlier_gnu}) or the $[90,20]$ (the horizontal neighbour).



\begin{sidewaystable}[!t]
\caption{Probabilities for individual data points, original approach.}
\label{table_big_original}
\begin{tblr}{
    width=\linewidth,
    hspan=minimal,
    cells={font=\footnotesize},
    cell{1}{1-11}={halign=c},
    column{odd}={gray9},
    %colsep=1pt,
    colspec={
    c |
    S[round-mode=places ,round-precision=2, output-exponent-marker=E, table-format=1.2e+1]
    S[round-mode=places ,round-precision=2, output-exponent-marker=E, table-format=1.2e+1]
    S[round-mode=places ,round-precision=2, output-exponent-marker=E, table-format=1.2e+1]
    S[round-mode=places ,round-precision=2, output-exponent-marker=E, table-format=1.2e+1]
    S[round-mode=places ,round-precision=2, output-exponent-marker=E, table-format=1.2e+1]
    S[round-mode=places ,round-precision=2, output-exponent-marker=E, table-format=1.2e+1]
    S[round-mode=places ,round-precision=2, output-exponent-marker=E, table-format=1.2e+1]
    S[round-mode=places ,round-precision=2, output-exponent-marker=E, table-format=1.2e+1]S[round-mode=places ,round-precision=2, output-exponent-marker=E, table-format=1.2e+1]
    S[round-mode=places ,round-precision=2, output-exponent-marker=E, table-format=1.2e+1]
    S[round-mode=places ,round-precision=2, output-exponent-marker=E, table-format=1.2e+1]
    S[round-mode=places ,round-precision=2, output-exponent-marker=E, table-format=1.2e+1]
    },
    row{1}={guard},
    column{1}={guard, mode=math}
}
 \diagbox{Point}{Depth} & 1 & 2 & 3 & 4 & 5 & 6 & 7 & 8 & 9 & 10 & 11 \\
 \hline
\left[25, 100\right] & 5.5555555556E-02 & 2.5103323737E-01 & 2.5570744502E-01 & 2.4773022596E-01 & 1.3819675888E-01 & 4.3771373491E-02 & 7.2084454376E-03 & 7.4839864762E-04 & 4.6942907037E-05 & 1.5970541667E-06 & 1.9690558403E-08\\
\left[20, 90\right] & 0 & 3.5539215686E-02 & 1.7775615069E-01 & 4.0676493336E-01 & 2.6984418116E-01 & 9.2561204648E-02 & 1.5747066825E-02 & 1.6759728793E-03 & 1.0748807222E-04 & 3.7394349905E-06 & 4.7254206811E-08\\
\left[30, 90\right] & 0 & 1.3368055556E-02 & 1.6859439869E-01 & 4.1552713020E-01 & 2.8292729344E-01 & 9.9767533944E-02 & 1.7705833609E-02 & 1.9721214421E-03 & 1.3271521257E-04 & 4.8533703768E-06 & 6.4529234241E-08\\
\left[35, 85\right] & 0 & 9.7486772487E-02 & 2.5792650002E-01 & 3.2462216793E-01 & 2.4112764708E-01 & 6.6916307461E-02 & 1.0752400748E-02 & 1.0967974134E-03 & 6.8988815370E-05 & 2.3864551481E-06 & 3.1601016673E-08 \\
\left[25, 85\right] & 0 & 0 & 2.4722562636E-02 & 3.0688106140E-01 & 4.7133742736E-01 & 1.6542794157E-01 & 2.8432586746E-02 & 3.0013428173E-03 & 1.9044983168E-04 & 6.5444606635E-06 & 8.3183446589E-08 \\
\left[15, 85\right] & 2.7777777778E-02 & 1.2482638889E-01 & 2.5516544084E-01 & 3.0935396980E-01 & 2.1892437075E-01 & 5.5415608206E-02 & 7.8217752227E-03 & 6.7869855413E-04 & 3.5013796136E-05 & 9.4653744554E-07 & 9.6253761386E-09\\
\hline
\left[20, 25\right] & 0 & 3.5693536674E-02 & 1.8350184759E-01 & 3.4881778276E-01 & 3.4168298237E-01 & 8.0460443076E-02 & 9.2155826505E-03 & 6.0623378611E-04 & 2.1232873030E-05 & 3.5512746939E-07 & 3.0996586909E-09
\end{tblr}

\end{sidewaystable}


\subsection{Novelty approach}
To provide an argument for the resulting depth using the novelty approach, we calculate the expected value of depth for each point $p$, considering all possible trees that would isolate $p$ and their respective paths. Contrary to the original approach, we can get potentially infinite splits, resulting in variable depth.
We start with given range $\langle 0, 110.0\rangle \times \langle -5.0, 105.0\rangle$.

% 3Hy
\paragraph{The first point to consider is \([25,100]\).} There is only one way to isolate this point: using three horizontal splits (H). This is the only scenario S1 for this point. Table \ref{table_25_100} shows the probabilities for individual depths. Since the vertical splits (V) do not contribute to the isolation of a given point, there could be any number of them. Hence, we get the expected value of depth as
$$\sum_{n=3}^{\infty}\binom{n-1}{2}\cdot \frac{1}{2^n}\cdot n = 6.$$

%first point 25,100
\begin{table}[!t]
\centering
\caption{Probabilities of depths for point $[25,100]$.}
\label{table_25_100}
\resizebox{\columnwidth}{!}{%
\begin{tblr}{
    width=\linewidth,
    hspan=minimal,
    cells={font=\footnotesize},
    colspec={c| c | c | c | c},
    column{odd}={gray9},
    row{1}={guard},
    column{1-5}={guard, mode=math}
}
 \diagbox{Depth}{Probab.} & V & H & S1 & \sum \\
 \hline
3 & 0 & 3 & \binom{2}{2}\cdot\frac{1}{2^3} & \frac{1}{8} \\
4 & 1 & 3 & \binom{3}{2}\cdot\frac{1}{2^4} & \frac{3}{16} \\
5 & 2 & 3 & \binom{4}{2}\cdot\frac{1}{2^5} & \frac{3}{16} \\
\vdots & \vdots & \vdots & \vdots & \vdots  \\
k & k-3 & 3 & \binom{k-1}{2}\cdot \frac{1}{2^k} & \binom{k-1}{2}\cdot \frac{1}{2^k} \\
\vdots & \vdots & \vdots & \vdots & \vdots \\
\hline
\sum & - & - & 1 & 1
\end{tblr}
}
\end{table}




% 4Vx or 2Vx + 5Hy
\paragraph{The second point in our sample is $[20,90]$.} Now, there are more ways to isolate this point. That is, by exactly four vertical splits and up to four horizontal splits (S1) or at least five horizontal splits along with two or three vertical splits (S2X).
To get the expected value for all the possible scenarios, we must calculate their probabilities. Table \ref{table_20_90} shows the probabilities for individual depths.

To simplify the second scenario --- two vertical and five horizontal splits --- we divide it into two sub-scenarios.
\begin{enumerate}
    \item Sub-scenario (S2V), where the last split is vertical (exactly two vertical splits).
    \item Sub-scenario (S2H), where the last split is horizontal (exactly five horizontal splits).
\end{enumerate}

The expected value of depth for the point $[20,90]$ is
\begin{multline*}
  \sum_{n=4}^{8}\binom{n-1}{3}\cdot \frac{1}{2^n}\cdot n + \sum_{n=7}^{\infty}\binom{n-1}{1}\cdot \frac{1}{2^n}\cdot n +\\ +\sum_{n=7}^{8}\binom{n-1}{4}\cdot \frac{1}{2^n}\cdot n \doteq 6.82.
\end{multline*}

%second point [20,90]
\begin{table}[!t]
\caption{Probabilities of depths for point $[20,90]$.}
\label{table_20_90}
\centering
\resizebox{\columnwidth}{!}{%
\begin{tblr}{
    width=\linewidth,
    hspan=minimal,
    cells={font=\footnotesize},
    colspec={c| c c c | c},
    column{odd}={gray9},
    %colsep=1pt,
    row{1}={guard},
    column{1-5}={guard, mode=math}
}
 \diagbox{Depth}{Probab.} & S1 & S2V & S2H & \sum \\
 \hline
4 & \binom{3}{3}\cdot \frac{1}{2^4} & 0 & 0 & \frac{1}{16} \\
5 & \binom{4}{3}\cdot\frac{1}{2^5}  &  0 & 0 & \frac{1}{8}\\
6 & \binom{5}{3}\cdot\frac{1}{2^6}  &  0 & 0& \frac{5}{32}\\
7 & \binom{6}{3}\cdot\frac{1}{2^7}  & \binom{6}{1}\cdot\frac{1}{2^7} & \binom{6}{4}\cdot\frac{1}{2^7} & \frac{41}{128} \\
8 & \binom{7}{3}\cdot\frac{1}{2^8}  & \binom{7}{1}\cdot\frac{1}{2^8} & \binom{7}{4}\cdot\frac{1}{2^8} & \frac{77}{256}\\
9 & 0 & \binom{8}{1}\cdot\frac{1}{2^9} & 0 & \frac{1}{64}\\
\vdots & \vdots & \vdots & \vdots & \vdots\\
k & 0 & \binom{k-1}{1}\cdot \frac{1}{2^k} & 0 & (k-1)\cdot\frac{1}{2^k}\\
\vdots & \vdots & \vdots & \vdots & \vdots \\
\hline
\sum & \frac{163}{256} & \frac{7}{64} & \frac{65}{256} & 1
\end{tblr}
}
\end{table}



\paragraph{The third point in the sample is $[15,85]$.} We need four vertical splits or two or three vertical and five horizontal splits to isolate this point. That is the same scenario as the previous point $[20,90]$; hence, we get the same depths, resulting in the same probabilities.

%4. (30,90) 5Vx or 2Vx + 5Hy
\paragraph{The fourth point in the sample is $[30,90]$.} We need either exactly five vertical splits (S1) or two, three or four vertical and exactly five horizontal splits to isolate this point. That gives, again, two sub-scenarios. First, the last split is vertical (S2V), and conversely, the last is horizontal (S2H).
Table \ref{table_30_90} shows the probabilities for individual depths.


% In the first scenario (S1), we consider exactly five vertical splits.
% To simplify the second scenario - two vertical and five horizontal splits - we divide it into two sub-scenarios.
% \begin{enumerate}
%     \item Sub-scenario (S2V), where the last split is vertical.
%     \item Sub-scenario (S2H), where the last split is horizontal.
% \end{enumerate}

The expected value of depth for the point $[30,90]$ is

\begin{multline*}
\sum_{n=5}^{9}\binom{n-1}{4}\cdot \frac{1}{2^n}\cdot n + \sum_{n=7}^{\infty}\binom{n-1}{1}\cdot \frac{1}{2^n}\cdot n+\\
+ \sum_{n=7}^{9}\binom{n-1}{4}\cdot \frac{1}{2^n}\cdot n \doteq 7.82
\end{multline*}





%ctvrty point [30,90]
\begin{table}[!t]
\caption{Probabilities of depths for point $[30,90]$.}
\label{table_30_90}
\centering
\resizebox{\columnwidth}{!}{%
\begin{tblr}{
    width=\linewidth,
    hspan=minimal,
    cells={font=\footnotesize},
    colspec={c | c c c | c },
    column{odd}={gray9},
    %colsep=1pt,
    row{1}={guard},
    column{1-5}={guard, mode=math}
}
 \diagbox{Depth}{Probab.} & S1 & S21 & S22 & \sum  \\
 \hline
5 & \binom{4}{4}\cdot\frac{1}{2^5} & 0 & 0 & \frac{1}{32}\\
6 & \binom{5}{4}\cdot\frac{1}{2^6} & 0 & 0 & \frac{5}{64}\\
7 & \binom{6}{4}\cdot\frac{1}{2^7} & \binom{6}{1}\cdot\frac{1}{2^7} & \binom{6}{4}\cdot\frac{1}{2^7} & \frac{9}{32}\\
8 & \binom{7}{4}\cdot\frac{1}{2^8} & \binom{7}{1}\cdot\frac{1}{2^8} & \binom{7}{4}\cdot\frac{1}{2^8} & \frac{77}{256}\\
9 & \binom{8}{4}\cdot\frac{1}{2^9} & \binom{8}{1}\cdot\frac{1}{2^9} & \binom{8}{4}\cdot\frac{1}{2^9} & \frac{37}{128}\\
10 & 0 & \binom{9}{1}\cdot\frac{1}{2^{10}} & 0 & \frac{9}{1024}\\
\vdots & \vdots & \vdots & \vdots & \vdots \\
k & 0 & \binom{k-1}{1}\cdot \frac{1}{2^k} & 0 & (k-1)\cdot \frac{1}{2^k} \\
\vdots & \vdots & \vdots & \vdots & \vdots \\
\hline
\sum & \frac{1}{2} & \frac{7}{64} & \frac{25}{64} & 1
\end{tblr}
}
\end{table}


\paragraph{The fifth point in the sample is $[35,85]$.}  We need four vertical splits to isolate this point. That is, any number of horizontal splits and four vertical splits are mixed so that the last split is always vertical.
We get the expected value of depth as follows
$$\sum_{n=4}^{\infty}\binom{n-1}{3}\cdot \frac{1}{2^n}\cdot n = 8.$$

Table \ref{table_35_85} shows the probabilities for individual depths for the fifth point.

%paty point [35,85]
\begin{table}[!t]
\caption{Probabilities of depths for point $[35,85]$.}
\label{table_35_85}
\centering
\resizebox{\columnwidth}{!}{%
\begin{tblr}{
    width=\linewidth,
    hspan=minimal,
    cells={font=\footnotesize},
    colspec={c | c | c},
    column{odd}={gray9},
    %colsep=1pt,
    row{1}={guard},
    column{1-3}={guard, mode=math}
}
 \diagbox{Depth}{Probab.} & S1 & \sum \\
 \hline
4 & \binom{3}{3}\cdot\frac{1}{2^4} & \frac{1}{16} \\
5 & \binom{4}{3}\cdot\frac{1}{2^5} &  \frac{1}{8}\\
6 & \binom{5}{3}\cdot\frac{1}{2^6} & \frac{5}{32} \\
\vdots & \vdots & \vdots \\
k & \binom{k-1}{3}\cdot \frac{1}{2^k} & \binom{k-1}{3}\cdot \frac{1}{2^k}\\
\vdots & \vdots & \vdots \\
\hline
\sum & 1 & 1
\end{tblr}
}
\end{table}


%5Vx + 3Hy || 4Vx + 5Hy
\paragraph{The sixth point in the sample is the point  $[25,85]$.} To isolate this point, we need five vertical and three or four horizontal splits (S1V and S1H) or four vertical and at least five horizontal splits (S2V and S2H).
Table \ref{table_25_85} shows the probabilities for individual depths for the above scenarios.
% Let us begin with the first scenario, five vertical and three horizontal splits. We divide it into more sub-scenarios:

% \begin{enumerate}
%     \item Sub-scenario (S1V), where the last split is vertical.
%     \item Sub-scenario (S1H), where the last split is horizontal.
% \end{enumerate}

% \begin{enumerate}
%     \item Sub-scenario - the last split is vertical - is generalized as: 
% $$\sum_{n=8}^{9}\binom{n-1}{4}\cdot \frac{1}{2^n}\cdot n$$
% \item Sub-scenario - the last split is horizontal - is generalized as:  
% $$\sum_{n=8}^{\infty}\binom{n-1}{2}\cdot \frac{1}{2^n}\cdot n$$
% \end{enumerate}

% In the second scenario, four vertical and five horizontal splits, we are left with the last two sub-scenarios:

% \begin{enumerate}
%     \item Sub-scenario (S2V), where the last split is vertical.
%     \item Sub-scenario (S2H), where the last split is horizontal. Here, only 9 splits are possible since in ten splits, there would be five vertical splits, making it the previous scenario.
% \end{enumerate}
The expected value of depth for the point $[25,85]$ is:
\begin{multline*}
\sum_{n=8}^{9}\binom{n-1}{4}\cdot \frac{1}{2^n}\cdot n
+ \sum_{n=8}^{\infty}\binom{n-1}{2}\cdot \frac{1}{2^n}\cdot n+\\
+ \sum_{n=9}^{\infty}\binom{n-1}{3}\cdot \frac{1}{2^n}\cdot n 
+ \binom{8}{4}\cdot \frac{1}{2^9} \doteq 9.734
\end{multline*}



% \begin{enumerate}
%     \item Sub-scenario - the last split is vertical - can be generalized as:
% $$\sum_{n=9}^{\infty}\binom{n-1}{3}\cdot \frac{1}{2^n}\cdot n$$
% \item Sub-scenario - the last split is horizontal. ; hence, for the depth of 9, we get:
% $$\binom{8}{4}\cdot \frac{1}{2^9}$$
% \end{enumerate}


%sesty point [25,85]
\begin{table}[!t]
% increase table row spacing, adjust to taste
\caption{Probabilities of depths for point $[25,85]$.}
\label{table_25_85}
\renewcommand{\arraystretch}{1.3}
\centering
\resizebox{\columnwidth}{!}{%
\begin{tblr}{
    width=\linewidth,
    hspan=minimal,
    cells={font=\footnotesize},
    colspec={c | c c c c | c},
    column{odd}={gray9},
    %colsep=1pt,
    row{1}={guard},
    column{1-6}={guard, mode=math}
}
 \diagbox{Depth}{Probab.} & S1V & S1H & S2V & S2H & \sum \\
 \hline
8 & \binom{7}{4}\cdot\frac{1}{2^8} &  \binom{7}{2}\cdot\frac{1}{2^8} & 0 & 0 & \frac{7}{32}\\
9 & \binom{8}{4}\cdot\frac{1}{2^9} & \binom{8}{2}\cdot\frac{1}{2^9} & \binom{8}{3}\cdot\frac{1}{2^9} & \binom{8}{4}\cdot\frac{1}{2^9} & \frac{7}{16} \\
10 & 0 & \binom{9}{2}\cdot\frac{1}{2^{10}} & \binom{9}{3}\cdot\frac{1}{2^{10}} & 0 & \frac{15}{128}\\
\vdots & \vdots & \vdots & \vdots & \vdots & \vdots \\
k & 0 & \binom{k-1}{2}\cdot\frac{1}{2^k} & \binom{k-1}{3}\cdot \frac{1}{2^k} & 0 & \binom{k}{3}\cdot \frac{1}{2^k}  \\
\vdots & \vdots & \vdots & \vdots & \vdots & \vdots \\
\hline
\sum & \frac{35}{128} & \frac{29}{128} & \frac{93}{256} & \frac{35}{256} & 1
\end{tblr}
}
\end{table}

We can determine that our novelty point $[25,20]$ gets the value of the expected depth of $3$. Table \ref{table_novelty} shows the probabilities for the individual depths. When orphaning this point, we only have two possible sub-scenarios.

If the last split is horizontal, we get
$$\sum_{n=2}^{\infty}\binom{n-1}{0}\frac{1}{2^{n}}\cdot n = 1.5.$$
Conversely, if the last split is vertical, we get
$$\sum_{n=2}^{\infty}\binom{n-1}{0}\frac{1}{2^{n}}\cdot n = 1.5.$$
This can be simplified as
$$\sum_{n=2}^{\infty}\frac{1}{2^{n-1}}\cdot n = 3.$$
The expected value of depth of $[25,85]$ is $3$.


%novelty 
\begin{table}[!t]
\centering
\caption{Probabilities of depths for the novelty point $[25,20]$.}
\label{table_novelty}
\begin{tblr}{
    width=\linewidth,
    hspan=minimal,
    cells={font=\footnotesize},
    colspec={c| c c |c},
    column{odd}={gray9},
    %colsep=1pt,
    row{1}={guard},
    column{2-5}={guard, mode=math}
}
 \diagbox{Depth}{Probab.} & S1V & S1H & \sum \\
 \hline
2 & \binom{1}{0}\cdot\frac{1}{2^2} &  \binom{1}{0}\cdot\frac{1}{2^2} & \frac{1}{2}\\
3 & \binom{2}{0}\cdot\frac{1}{2^3} & \binom{2}{0}\cdot\frac{1}{2^3} &  \frac{1}{4}\\
4 & \binom{3}{0}\cdot\frac{1}{2^4} & \binom{3}{0}\cdot\frac{1}{2^4} & \frac{1}{8}\\
\vdots & \vdots & \vdots &\vdots \\
k & \binom{k-1}{0}\cdot\frac{1}{2^k} & \binom{k-1}{0}\cdot\frac{1}{2^k}&\frac{1}{2^{k-1}} \\
\vdots & \vdots & \vdots & \vdots\\
\hline
\sum & \frac{1}{2} & \frac{1}{2} & 1 \\
\end{tblr}
\end{table}

Table \ref{table_big_novelty} shows the aggregated sums for individual depths for comparison with the expected values for the original approach in Table \ref{table_big_original}.

% stars tabulka se zlomky
% \begin{sidewaystable}[p]
% \label{table_big_novelty_old}
% \begin{tblr}{
%     width=\linewidth,
%     hspan=minimal,
%     cells={font=\footnotesize},
%     cell{1}{1-11}={halign=c},
%     column{odd}={gray9},
%     %colsep=1pt,
%     colspec={
%     c |
%     S[round-mode=places ,round-precision=2, output-exponent-marker=E, table-format=1.2e+1]
%     S[round-mode=places ,round-precision=2, output-exponent-marker=E, table-format=1.2e+1]
%     S[round-mode=places ,round-precision=2, output-exponent-marker=E, table-format=1.2e+1]
%     S[round-mode=places ,round-precision=2, output-exponent-marker=E, table-format=1.2e+1]
%     S[round-mode=places ,round-precision=2, output-exponent-marker=E, table-format=1.2e+1]
%     S[round-mode=places ,round-precision=2, output-exponent-marker=E, table-format=1.2e+1]
%     S[round-mode=places ,round-precision=2, output-exponent-marker=E, table-format=1.2e+1]
%     S[round-mode=places ,round-precision=2, output-exponent-marker=E, table-format=1.2e+1]S[round-mode=places ,round-precision=2, output-exponent-marker=E, table-format=1.2e+1]
%     S[round-mode=places ,round-precision=2, output-exponent-marker=E, table-format=1.2e+1]
%     S[round-mode=places ,round-precision=2, output-exponent-marker=E, table-format=1.2e+1]
%     S[round-mode=places ,round-precision=2, output-exponent-marker=E, table-format=1.2e+1]
%     },
%     row{1}={guard},
%     column{2-12}={mode=math},
%     column{1}={guard, mode=math}
% }
%  \diagbox{Point}{Depth} & 1 & 2 & 3 & 4 & 5 & 6 & 7 & 8 & 9 & 10 & 11 \\
%  \hline
% \left[25, 100\right] & 0 & 0 & \frac{1}{8} & \frac{3}{16} & \frac{3}{16} & \frac{5}{32} & \frac{15}{128} & \frac{21}{256} & \frac{7}{128} & \frac{9}{256} & \frac{55}{4096} \\
% \left[20, 90\right] & 0 & 0 & 0 & \frac{1}{16} & \frac{1}{8} & \frac{5}{32} & \frac{41}{128} & 
% \frac{77}{256} & \frac{1}{64} & \frac{9}{1024} & \frac{5}{1024}\\
% \left[30, 90\right] & 0 & 0 & 0 & 0 & \frac{1}{32} & \frac{5}{64} & \frac{9}{32} & \frac{77}{256} & \frac{37}{128} & \frac{9}{1024} & \frac{5}{1024}\\
% \left[35, 85\right] & 0 & 0 & 0 & \frac{1}{16} & \frac{1}{8} & \frac{5}{32} & \frac{5}{32} & \frac{35}{256} & \frac{7}{64} & \frac{21}{256} & \frac{15}{256} \\
% \left[25, 85\right] & 0 & 0 & 0 & 0 & 0 & 0 & 0 & \frac{7}{32} & \frac{7}{16} & \frac{15}{128} & \frac{165}{2048} \\
% \left[15, 85\right] & 0 & 0 & 0 & \frac{1}{16} & \frac{1}{8} & \frac{5}{32} & \frac{41}{128} & 
% \frac{77}{256} & \frac{1}{64} & \frac{9}{1024} & \frac{5}{1024}\\
% \hline
% \left[20, 25\right] & 0 & \frac{1}{2} & \frac{1}{4} & \frac{1}{8} & \frac{1}{16} & \frac{1}{32} & \frac{1}{64} & \frac{1}{128} & \frac{1}{256} & \frac{1}{512} & \frac{1}{1024}
% \end{tblr}
% \end{sidewaystable}





\begin{sidewaystable}[!t]
\caption{Probabilities for individual data points, enhanced approach.}
\label{table_big_novelty}
\begin{tblr}{
    width=\linewidth,
    hspan=minimal,
    cells={font=\footnotesize},
    cell{1}{1-11}={halign=c},
    column{odd}={gray9},
    %colsep=1pt,
    colspec={
    c |
    S[round-mode=places ,round-precision=2, output-exponent-marker=E, table-format=1.2e+1]
    S[round-mode=places ,round-precision=2, output-exponent-marker=E, table-format=1.2e+1]
    S[round-mode=places ,round-precision=2, output-exponent-marker=E, table-format=1.2e+1]
    S[round-mode=places ,round-precision=2, output-exponent-marker=E, table-format=1.2e+1]
    S[round-mode=places ,round-precision=2, output-exponent-marker=E, table-format=1.2e+1]
    S[round-mode=places ,round-precision=2, output-exponent-marker=E, table-format=1.2e+1]
    S[round-mode=places ,round-precision=2, output-exponent-marker=E, table-format=1.2e+1]
    S[round-mode=places ,round-precision=2, output-exponent-marker=E, table-format=1.2e+1]
    S[round-mode=places ,round-precision=2, output-exponent-marker=E, table-format=1.2e+1]
    S[round-mode=places ,round-precision=2, output-exponent-marker=E, table-format=1.2e+1]
    S[round-mode=places ,round-precision=2, output-exponent-marker=E, table-format=1.2e+1]
    },
    row{1}={guard},
    column{2-13}={mode=math},
    column{1}={guard, mode=math}
}
 \diagbox{Point}{Depth} & 1 & 2 & 3 & 4 & 5 & 6 & 7 & 8 & 9 & 10 & >10 \\
 \hline
\left[25, 100\right] & 0 & 0 & 1.250e-1 & 1.880e-1 & 1.880e-1 & 1.560e-1 & 1.170e-1 & 8.200e-2 & 5.470e-2 & 3.520e-2 & 4.07e-02 \\
\left[20, 90\right] & 0 & 0 & 0 & 6.250e-2 & 1.250e-1 & 1.560e-1 & 3.200e-1 & 3.010e-1 & 1.560e-2 & 8.790e-3 & 6.23e-03 \\
\left[30, 90\right] & 0 & 0 & 0 & 0 & 3.130e-2 & 7.810e-2 & 2.810e-1 & 3.010e-1 & 2.890e-1 & 8.790e-3 & 5.93e-03 \\
\left[35, 85\right] & 0 & 0 & 0 & 6.250e-2 & 1.250e-1 & 1.560e-1 & 1.560e-1 & 1.370e-1 & 1.090e-1 & 8.200e-2 & 1.14e-01 \\
\left[25, 85\right] & 0 & 0 & 0 & 0 & 0 & 0 & 0 & 2.190e-1 & 4.380e-1 & 1.170e-1 & 1.45e-01 \\
\left[15, 85\right] & 0 & 0 & 0 & 6.250e-2 & 1.250e-1 & 1.560e-1 & 3.200e-1 & 3.010e-1 & 1.560e-2 & 8.790e-3 & 6.23e-03\\
\hline
\left[20, 25\right] & 0 & 5.000e-1 & 2.500e-1 & 1.250e-1 & 6.250e-2 & 3.130e-2 & 1.560e-2 & 7.810e-3 & 3.910e-3 & 1.950e-3 & 9.53e-04
\end{tblr}
\end{sidewaystable}

\subsection{Conclusion}
The expected values of depths (EXD) for the HST algorithm are shown in the section above.
To compare the values with the expected values of depths for the original isolation forest, we sum the rows in Table \ref{table_big_original} multiplied by respective depth values. Table \ref{table_ex_comparison} shows the side-by-side comparison. The important outcome is the ratio of expected depths in the respective column. As seen in this table, the difference in depth for the novelty point in the case of the original approach is relatively small compared to the difference in the HST forest.
This example proved that it is feasible for the HST algorithm to encapsulate the previously unseen data points in the higher leaves of the tree, making novelty detection possible.
This example closely resembles other similar novelty detection problems.

\begin{table}[!t]
\caption{Expected values of depths for both algorithms.}
\label{table_ex_comparison}
\centering
\begin{tblr}{
    width=\linewidth,
    cells={font=\footnotesize},
    colspec={c | 
    S[table-format=1.3, round-mode=places ,round-precision=3] 
    S[table-format=1.3]},
    %colsep=1pt,
    column{1-3}={mode=math},
    column{1}={preto=[, appto=]},    
    row{1}={guard,mode=text}
}
point & EXD\ outlier & EXD\ novelty \\
\hline
25,100 & 3.32616229 & 6\\
20,90 & 4.26063731 & 6.82\\
30,90 & 4.34883099 & 7.82\\
35,85& 3.95906409 & 8\\
25,85 & 4.87576598 & 9.734\\
15,85 & 3.76796497 & 6.82\\
\hline
20,25 & 4.27789495 & 3
\end{tblr}
\end{table}

\section{Discussion and Conclusions}
\label{sec:conclusion}

% todo: tady vic rozepsat co vidi ctenar v tech tabulkach
% pak jsme dokazali na konkretnim prikladu ze opravdu funguje pro novelty bod
% kdyz se podivame na rozdily v deptech originalu a toho naseho tak vidime ze hlavne pro ten novelty bod jsou tam velke zmeny v hloubce, podobne to funguje pro jakekoli jine novelty body


The research presented herein introduces a novel algorithm which has been demonstrated to effectively handle specific novelty points as evidenced through a detailed example. Notably, comparisons between the depths of the original algorithm and the new one indicate significant modifications at these novelty points, suggesting enhanced performance in these areas. Such findings not only affirm the practical utility of the new algorithm but also highlight its potential adaptability to various situations within the algorithmic framework. Future work will focus on establishing optimal range settings at the onset, addressing unique scenarios within the algorithmic structure, and validating these enhancements through comprehensive benchmarks. This proposed trajectory aims to further substantiate the robustness and efficiency of the new algorithm in diverse operational contexts.




%% If you have bibdatabase file and want bibtex to generate the
%% bibitems, please use
%%
\bibliographystyle{bibliography/model1-num-names} 
\bibliography{bibliography/bibli}

%% else use the following coding to input the bibitems directly in the
%% TeX file.

% \begin{thebibliography}{00}

% %% \bibitem[Author(year)]{label}
% %% Text of bibliographic item

% \bibitem[ ()]{}

% \end{thebibliography}
\end{document}

\endinput
%%
%% End of file `elsarticle-template-num-names.tex'.
