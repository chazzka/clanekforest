%% 
%% Copyright 2007-2020 Elsevier Ltd
%% 
%% This file is part of the 'Elsarticle Bundle'.
%% ---------------------------------------------
%% 
%% It may be distributed under the conditions of the LaTeX Project Public
%% License, either version 1.2 of this license or (at your option) any
%% later version.  The latest version of this license is in
%%    http://www.latex-project.org/lppl.txt
%% and version 1.2 or later is part of all distributions of LaTeX
%% version 1999/12/01 or later.
%% 
%% The list of all files belonging to the 'Elsarticle Bundle' is
%% given in the file `manifest.txt'.
%% 
%% Template article for Elsevier's document class `elsarticle'
%% with harvard style bibliographic references
%\documentclass[3p,times,procedia,table]{elsarticle}
\documentclass[12pt]{article}

%% Use the option review to obtain double line spacing
%% \documentclass[preprint,review,12pt]{elsarticle}

%% Use the options 1p,twocolumn; 3p; 3p,twocolumn; 5p; or 5p,twocolumn
%% for a journal layout:
%% \documentclass[final,1p,times]{elsarticle}
%% \documentclass[final,1p,times,twocolumn]{elsarticle}
%% \documentclass[final,3p,times]{elsarticle}
%% \documentclass[final,3p,times,twocolumn]{elsarticle}
%% \documentclass[final,5p,times]{elsarticle}
%% \documentclass[final,5p,times,twocolumn]{elsarticle}

%% For including figures, graphicx.sty has been loaded in
%% elsarticle.cls. If you prefer to use the old commands
%% please give \usepackage{epsfig}

%% The amssymb package provides various useful mathematical symbols
\usepackage{amssymb}
\usepackage{svg}
\usepackage{siunitx}
\usepackage{amsmath}
\usepackage{amsthm}
\usepackage{tabularray}
\UseTblrLibrary{amsmath}
\UseTblrLibrary{siunitx}
\UseTblrLibrary{diagbox}
\usepackage{rotating}
\usepackage{mathtools}
\usepackage{pstricks}    %for embedding pspicture.
\usepackage{authblk}

\theoremstyle{definition}
\newtheorem{example}{Example}[section]
\newtheorem{definition}{Definition}[section]

%% The amsthm package provides extended theorem environments
%% \usepackage{amsthm}

%% The lineno packages adds line numbers. Start line numbering with
%% \begin{linenumbers}, end it with \end{linenumbers}. Or switch it on
%% for the whole article with \linenumbers.
%% \usepackage{lineno}


\begin{document}

%% Title, authors and addresses

%% use the tnoteref command within \title for footnotes;
%% use the tnotetext command for theassociated footnote;
%% use the fnref command within \author or \address for footnotes;
%% use the fntext command for theassociated footnote;
%% use the corref command within \author for corresponding author footnotes;
%% use the cortext command for theassociated footnote;
%% use the ead command for the email address,
%% and the form \ead[url] for the home page:
%% \title{Title\tnoteref{label1}}
%% \tnotetext[label1]{}
%% \author{Name\corref{cor1}\fnref{label2}}
%% \ead{email address}
%% \ead[url]{home page}
%% \fntext[label2]{}
%% \cortext[cor1]{}
%% \affiliation{organization={},
%%             addressline={},
%%             city={},
%%             postcode={},
%%             state={},
%%             country={}}
%% \fntext[label3]{}

\title{Isolation Forest in Novelty Detection Scenario}
\author[1]{Adam Ulrich}
\author[1]{Jak Krňávek} 
\author[1]{Roman Šenkreřík}
\author[1]{Zuzana Komínková Oplatková}
\author[1]{Radek Vala}
\affil[1]{Faculty of Applied Informatics, Tomas Bata University in Zlin}
%% \linenumbers

%% For citations use: 
%%       \citet{<label>} ==> Jones et al. [21]
%%       \citep{<label>} ==> [21]
%%

\maketitle
%% main text
\section{Introduction}
\label{sec:introduction}
% Tady popíšeme obecně jak se snažíme detekovat anomalie, proč je to důležité, tisice citací na lidi kteří detekují anomalie, pak už víc k novelty detection, ukázat jak se to dělalo dřív co je cílem a proč to může být problem a nakonec uplně specificky co my chceme udělat

% několik bodů

% general to specific, hodně citací

% 1. nejdřív popíšeme že datamining je nějaké automatizované zpracování dat, může hledat vzory, citace, může hledat clustery, citace, může hledat anomalie
% ukaž tu hlavně algoritmy, ne konkretni aplikace, možná jednu větu ke každému?


Datamining is a vast topic where we use automation mechanisms to process large data in various formats. That could be binary data from numerous electromechanical sensors, numerical data serialized by some processing computer, or even nominal data stored in the database.
Datamining algorithms can be used to find specific patterns in data, which is a topic of a pattern mining subfield. Algorithms in this subfield solve tasks like sequential mining of patterns \cite{agrawal1995mining} or frequent-itemset mining \cite{agrawal1994fast}. 
Some solutions lay in mining for similarities in data. Such similarities often form a batch in specific parts of analyzed space. It can be formed when specific attributes correlate. Another subfield of data mining focuses on identifying these batches --- also called clusters in data. Various clustering algorithms have been developed, such as DBSCAN \cite{Ester1996dbscan} or \(k\)-means \cite{lloyd1982kmeans} and their successful derivates.
With the recent upsurge of IOT, a subfield of anomaly data mining has become popular. Often, the data obtained is not what we expect it to be; sometimes, it can differ significantly from the rest and be identified as anomalies.


% 2. teď popíšu teda co jsou ty anomalie, proč je chceme hledat, furu citaci na članky na algoritmy, kde se hledaji anomalie, opět jednu větu ke každému

Anomaly detection problems can be viewed in various ways depending on the specific domain. There are anomaly detectors based on statistics (Z-score or Grubbs's test \cite{grubbs1949sample}), clusters \cite{he2003discovering} and density-based methods like Isolation Forest \cite{liu2008isolation, liu2012isolation}.


% 3. ted už konkretně na outlier vs novelty se zaměřením na novelty, ukaž tu svm, linear svm, lof

Usually, anomaly detection is used to solve one of the following tasks (see Markou et al. in \cite{MARKOU20032481}).
\begin{description}
    \item[Outlier Detection]  task with all of the data available in advance, and the algorithm is to identify outlying anomalies. This process is unsupervised since no labels are available in advance for the input data.
    \item[Novelty Detection]  task with the majority of regular data available in advance and no anomalies. The algorithm is to learn how the regular data looks and later identify anomalies (in this scenario called \emph{novelties}).
\end{description}

Novelty detection is a semi-supervised technique for detecting anomalies (\emph{novelties}) unavailable in the training set.
The first algorithm mentioned in novelty detection is the One-class SVM algorithm \cite{tax2004support}. Despite being primarily developed for classification, this algorithm is often referred to as one of the first novelty detection algorithms. It learns from the input data it surrounds; hence, it can identify data outside this boundary.
This algorithm is based on quadratic programming, although Zhou et al. \cite{ZHOU20022927} provide an enhancement based on linear programming.
Another novelty detection algorithm is called the Local Outlier Factor from the family of distance-based algorithms \cite{breunig2000lof} that assigns each object a degree of being an anomaly.



% 4. a tady už konkrétní problem, proč teda my tvoříme novelty detection algoritmus, protože jsme potřebovali něco jiného než lof a pracovali jsme s isolation forestem, ktery dobře fungoval na outliery a krásně šlo přidávat stromy další a my chtěli zachovat tu myšlenku toho kooperativního algoritmu více stromů ale použít v semi supervised manneru
One of the downsides when dealing with the above algorithms is their lack of interpretability. It is not trivial to visualize the outcomes and understand the reasons for the point being a novelty or otherwise, making it arduous to understand the dataset's properties. That is not the case for the Isolation Forest, though, where the visualization is its superiority.
Elaborate research on novelty detectors has highlighted several determining properties when dealing with such problems. First, the novelty detection algorithm should be semi-supervised (that is, it can be trained on a dataset first and evaluated later). Then, the algorithm should be able to work with \(n\)-dimensional spaces. Lastly, the algorithm should evaluate both data seen during the learning phase and those never seen before. Only then will the novelty detector be able to work properly.

The Half-Space Tree (HST) algorithm, introduced by Tan et al. in \cite{tan2011fast} for anomaly detection for streaming data, is a data structure that recursively partitions the data space into half-spaces to model the distribution of normal instances.
The core of this algorithm is based on the Isolation Forest by Liu et al. \cite{liu2008isolation}, \cite{liu2012isolation}.
It builds on the idea of the binary decision tree and alters the process of building and evaluating the decision tree to isolate anomalies. By using an algorithm based on binary trees, the interpretability of the problem is significantly enhanced, as the tree structure allows us to visually trace and understand the decision-making process.

Our main task is to utilize the HST algorithm to develop a solid and adaptable novelty detector.
This article proposes a new HST modification, similar to Ting's in \cite{ting2013mass}, specifically tailored for detecting novelties.
We show that this modification is particularly well-suited for novelty detection due to its unique structure and operational approach.
First, we provide a theoretical framework for this modified algorithm and the original Isolation Forest.
We then provide examples of using it to build and evaluate the tree.
With this framework built up, we show the distinction between those two algorithms and examples of using them in novelty search scenarios.





\section{Glossary}
\label{sec:theory}
Since this article is centered around the anomaly detection algorithm topics, several terms that will be seen across the rest of the article have to be introduced.
\begin{description}
    \item[(Data) point] Any observable data with \(n\) dimensions.
    \item[Regular point] Data point included in the given dataset. Its features are expected.
    \item[Anomaly] Data point that differs significantly from other observations.
    \item[Outlier] Anomaly included in the given dataset.
    \item[Novelty] Anomaly that is not present in the given dataset during
learning. Novelties are supplied later during evaluation.
    \item[Supervised algorithm] Algorithm, that is being trained on all available data labels.
    \item[Unsupervised algorithm] Algorithm, that does not get any labels for the training data.
    \item[Semi-supervised algorithm] Algorithm that is trained on data of only one class.
\end{description}

% \section{Methods - toto se smaže}
\label{sec:methods}
Traditional approaches for anomaly detection consist of either novelty
detection or outlier detection. Novelty detection is an anomaly
detection mechanism where we search for unusual observations 
discovered due to their differences from the training data. Novelty
detection is a semi-supervised anomaly detection technique, whereas
outlier detection uses unsupervised methods. With novelty detection, the
training data is not polluted by anomalous elements, and we are
interested in detecting whether a new observation is an anomaly. In this context, such points are also called novelties. This is a crucial
distinction. The outlier detection is usually presented with data
containing both anomalies and regular observation; it then uses
mathematical models that try to make a distinction between them. The
novelty detection, on the other hand, is usually presented with data with
little to zero anomalies (the proportion of anomalies in the dataset is
called a contamination), and later, when conferred with an anomalous
observation, it makes a decision.

Consider the following example: Figure \ref{fig:example0} contains random datapoints
arranged in a way they form a cluster-like shape. Say this data is our
regular observations.

\begin{figure}[htbp]
\centering
\includesvg[width=0.9\textwidth]{figures/example0_gnu.svg}
\caption{Example figure}
\label{fig:example0}
\end{figure}

When an unsupervised, outlier detection algorithm tries to analyze such
data, it sees the datapoints as a cluster containing both regular and
anomalous observations. Figure X shows the result of evaluating
classical Isolation Forest on such a dataset.

\begin{figure}[htbp]
\centering
\includesvg[width=0.9\textwidth]{figures/example0_5_gnu.svg}
\caption{Example figure}
\label{fig:example05}
\end{figure}

Figure \ref{fig:example05} shows regular observations \(x\) and anomaly observations \(y\)
marked by Isolation Forest
(\texttt{batch\_size\ 128,\ trees\_count:\ 100,\ zbytek\ default}).
Figure x shows that approx. \(10\%\) of observations are anomalies. This
is not unwanted behavior in the sense of outlier detection but is false
positive observation in the sense of novelty detection because this
data are regular observations that should be marked so.

Another problem is with the unsupervised separation itself. Consider data polluted by anomalies in close to \(1:1\) ratio. Finding the line
itself is evident. Deciding which observations are anomalies without some domain knowledge, on the other hand, is close to impossible.

\section{Graph Theory}
\label{sec:graph_theory}

 The isolation forest is composed of trees that are, in fact, directed graphs.
 Hence, the theoretical framework used in this article is centred mainly around graph theory.
 The definitions used in this section are based on Rosen et~al. \cite{rosen2012discrete}. 


\begin{definition}
 \emph{Directed graph} (or digraph) $D = (V, E)$ consists of a nonempty set of vertices $V$ and a set of directed edges $E \in V \times V$.
 Each directed edge is an ordered pair of vertices.
 The directed edge $(u, v)$ is said to start at $u$ and end at $v$.
\end{definition}


%\begin{definition}
For the digraph $D = (V,E)$ we define:
\begin{itemize}
    \item \emph{Descendant} of a vertex $v \in V$ is a vertex $v' \in V$ such that $(v,v') \in E$. 
    We say that $v$ is the \emph{ancestor} associated with $v'$.
    \item \emph{Walk} $W$ from a vertex $v_0 \in V$ to a vertex $v_n \in V$ is a finite set of edges ($W \subseteq E$), such that $$W = \{(v_0, v_1),(v_1, v_2),\dots,(v_{n-1}, v_n)\}.$$
    The empty set $W = \emptyset$ is a trivial walk from $v$ to $v$ for every $v \in V$. 
\end{itemize}
%\end{definition}

With the isolation tree starting at the root, we define a rooted tree.

\begin{definition}
\emph{Rooted} tree is the digraph $T = (V,E)$ with a single vertex $r \in V$ designated as the \emph{root}, with exactly one walk (so-call \emph{path}) from $r$ to $v$ for each $v \in V$.
\end{definition}

%\begin{definition}
For the rooted tree $T = (V,E)$ we define:
\begin{itemize}
    \item \emph{Depth} of $v \in V$ is the size of the path from the root to $v$.
    \item \emph{Leaf} is a vertex $v \in V$ that has no descendants. 
    \item \emph{Internal  vertex} conversely has a descendant.
    \item To \emph{traverse} a tree $T$ is to take a path $P$ from the root to the leaf. If $(v,v') \in P$, we say that we \emph{visit} $v$ and $v'$ while traversing.
\end{itemize}
%\end{definition}


\section{Isolation Forest}
\label{sec:isolation_forest}

Isolation Forest
\cite{liu2008isolation, liu2012isolation} is an outlier
detection, unsupervised ensemble algorithm. This approach is well-known for successfully identifying outliers using recursive partitioning (forming a decision tree) to decide whether the analyzed data point is an anomaly. The fewer partitions required to isolate, the more probable it is for a point to be an anomaly.

% \paragraph{Isolation tree}
% Isolation Tree is a rooted tree constructed with a subset of \(A\) items
% (datapoints) with the size \(s=|A|\).

% \begin{enumerate}
%     \item To build an isolation tree, it is not necessary to have a large set; it
%   may even be undesirable
%   \item Well-chosen small \(s\) can help eliminate \emph{masking} and
%   \emph{swamping}


% \begin{itemize}
%     \item \textbf{Masking} When the number of anomalies is high, it is possible that some of those aggregate in a dense and large cluster, making it more difficult to separate the single anomalies and, in turn, to detect such points as
%   anomalous.
%     \item \textbf{Swamping} When normal instances are too close to anomalies, the number of
%   partitions required to separate them increases, making it more difficult for the Isolation Forest to discriminate between
%   anomalies and normal points.
% \end{itemize}

% \item There are two types of vertices

% \begin{itemize}
%     \item \textbf{Internal vertex} Internal vertex contains a condition (a feature and a limit) and two children (one
%   representing a fulfilled condition and the other an unfulfilled one)
%     \item \textbf{External vertex} External vertex is created if the conditions of the parents are met (or not met) by
%   one or none of the elements from the sample or the maximum depth of
%   the tree \(l\) is reached, usually \(l=\ln_2(s)\). It contains the
%   evaluation of \(h(x)\) using the distance from the root. If the max
%   length of the tree is reached the \emph{distance} is estimated using
%   \(h(x)=e+c(n)\), where \(e\) is the distance from the root, \(n\) is
%   the number of elements from the sample satisfying the conditions of
%   the parents, \(c(n)=2\,(H_{n-1}-\frac{n-1}{n})\) and \(H_{n-1}\) is
%   \(n-1\) harmonic number.
% \end{itemize}

% \end{enumerate}

\begin{definition}
Let $\mathsf{s}=(s_0, \dots, s_d, \dots, s_{n-1})\in \mathbb{R}^n$ be a data point. Then we say that $\pi_d(\mathsf{s}):=s_d$ is a \emph{projection} of data point $\mathsf{s}$ onto  dimension $d$  yielding $s_d$.
\end{definition}

\begin{definition}
Let $Z$ be a finite subset of $\mathbb{R}^n$,
$$Z \subseteq \mathbb{R}^n ;\quad n \in \mathbb{N}.$$

For each dimension \(d \in\{0, \dots, n - 1\}\), let
$$Z_d = \{ \pi_d(\mathsf{s})\ |\ \mathsf{s} \in Z \}.$$

Then we define a hyperrectangle $R(Z)$ \emph{surrounding} $Z$, such that
$$R(Z) = r_0 \times r_1 \times \cdots \times r_{n-1},$$ where $r_d = \langle \min Z_d, \max Z_d \rangle$ for each $d \in \{0,1, \dots, n-1\}.$

\end{definition}




In this section, we describe the original decision tree.
We deviated slightly from Liu et al.~\cite{liu2008isolation} for better interpretability and comparability with the proposed solution.
Note that this deviation is theoretical and does not affect the original functionality.

\subsection{Constructing a decision tree $T$}
\begin{enumerate}

    \item Each leaf or internal vertex is a hyperrectangle $R$. 
    \item Function relation $\rho$ assigns the split point $z$ and the dimension $d$ to the internal vertex $v$,
    $$(v, (z,d)) \in \rho.$$
    % \item The maximum possible height of a tree is controlled by the \emph{max\_depth} parameter.
    \item The sample \(S\) contains \emph{b} number of input points of $n$ dimensions, that is
    $$S \subseteq \mathbb{R}^n ;\quad |S| = b; \quad b \in \mathbb{N}.$$
    \item The ending condition of the leaves is satisfied if the vertex \(R\) satisfies \[| S \cap R | = 1.\]
%     \item Leaves' ending condition is satisfied by one of two criteria: 
% \begin{enumerate}
%     \item Vertex \(R\) satisfies \(| S \cap R | = 1\).
%     \item Vertex's depth reached the \emph{max\_depth} value.
% \end{enumerate}
\end{enumerate}


\paragraph{Basis step}
The trivial rooted tree \(T_0\) is a tuple with
vertices \(V_0 = \{R(S)\}\) and edges \(E_0 = \emptyset\), i.e. 
\[T_0= (V_0, E_0) = (\{R(S)\},\emptyset).\]
The function relation $\rho_0$ initially has no assignments, i.e.
$$\rho_0 = \emptyset.$$

% The steps to reach the tree \(T_{j+1}\) from \(T_{j} = (V_j, E_j)\) and $\rho_{j+1}$ from $\rho_{j}$ are
% as follows:
% Let us create \(T_{j+1}\) from \(T_{j} = (V_j, E_j)\) and $\rho_{j+1}$ from $\rho_{j}$.
% The creation of \(T_{j+1}\) from \(T_{j} = (V_j, E_j)\) and $\rho_{j+1}$ from $\rho_{j}$ starts by forming a new hyperrcentagle R.
% to reach the tree \(T_{j+1}\) from \(T_{j} = (V_j, E_j)\) and $\rho_{j+1}$ from $\rho_{j}$
\paragraph{Recursive step}
The task of this step is to reach the tree \(T_{j+1}\) from \(T_{j} = (V_j, E_j)\) and $\rho_{j+1}$ from $\rho_{j}.$
Let \(L_j \subseteq V_j\) be a subset of leaves not satisfying the
ending condition
% By selecting a random dimension $d_R$ (such that $|r_{d_R}| > 1$) create two new vertices \(R_l, R_r\) for each leaf \(R \in L_j\)\
% \[R =  r_0 \times \cdots \times r_{d_R} \times \cdots \times r_{n-1} \] with a random split point $z_R \in r_{d_R}$.
and \(R \in L_j\)\ be a leaf such that
\[R =  r_0 \times \cdots \times r_{d_R} \times \cdots \times r_{n-1}. \]
We first select a random dimension $d_R$ (such that $r_{d_R}$ is infinite) and a random split point $z_R \in r_{d_R}$.
The split point $z_R$ splits $R \cap S$, creating two disjunctive sets $S_l$ and $S_r$ respectively, such that
\begin{align*}
S_l &= \{ \mathsf{s} \in{R \cap S}\ |\ \pi_{d_R}(\mathsf{s})\le z_R\},&
S_r &= \{ \mathsf{s} \in{R \cap S}\ |\ \pi_{d_R}(\mathsf{s}) > z_R\}.
\end{align*}
Then we obtain left and right hyperrectangles \(R_l\), \(R_r\) as
follows:
\begin{align*}
R_l &= R(S_l),&
R_r &= R(S_r).
\end{align*}

Each vertex \(R \in L_j\) is associated with two new
edges \((R,R_l ), (R, R_r)\) and is assigned with $(z_R,d_R)$ by function relation $\rho_{j+1}$, such that


\begin{align*}
   \rho_{j+1} &= \rho_j \cup \bigcup_{R \in L_j} \{(R, (z_R, d_R))\}, \\
   V_{j+1} &= V_j \cup \bigcup_{R \in L_j} \{R_l, R_r\}, \\
   E_{j+1} &= E_j \cup \bigcup_{R \in L_j} \{(R, R_l), (R,R_r)\},\\
   T_{j+1} &= (V_{j+1}, E_{j+1}),
\end{align*}
i.e.~${R_l, R_r} \subset R$ are leaves and $R$ is an inner vertex in the new tree
\(T_{j+1}\).\footnote{Tree \(T_{j+1}\) is actually a Hasse diagram of the ordered set
\((V_{j+1},\subseteq)\)}


\paragraph{Termination} The algorithm moves to the next recursion step unless there is an equality of two consecutive trees \(T_k = T_{k+1}\). Such equality happens when all leaves satisfy their ending condition, i.e., \(L_k = \emptyset\).
Thus, the desired tree $T$ is the tree $T_k$, and the finite relation $\rho$ is $\rho_k$.


\begin{example}
\label{example:original_tree_create}
Consider now an example of creating a new Isolation tree based on the given input sample
\begin{align*}
    S = \{&[25,100],[30,90],[20,90],[35,85],\\
    &[25,85],[15,85],[105,20],[95,25], \\
    &[95,15],[90,30],[90,20],[90,10]\},
\end{align*}
as shown in Figure \ref{fig:example_noutlier_gnu}.

\begin{figure}[htbp]
\centering
\includesvg[width=0.9\textwidth, inkscapelatex=false]{figures/example6_noutlier_gnu_legend.svg}
\caption{Original solution. Rectangles created by recursive splitting.}
\label{fig:example_noutlier_gnu}
\end{figure}



The tree $T$ is created by starting with the tree $T_0$ and expanding further as described by the recursive step until the ending condition is met.
Figure \ref{fig:example_noutlier_tree_color} shows the finished tree $T$, trained on the dataset $S$. Numbers represent the final leaves’ depth.

\paragraph{Basis step} 
We create the tree $T_0$ and a function relation $\rho_0$.
   There is just a single vertex (root) 
   \[R = R(S) = \langle 15, 105 \rangle \times \langle 10, 105 \rangle,\]
   without any connections, so both $E_0$ and $\rho_0$ are empty, i.e.
\begin{align*}
V_0 &= \{R\},& E_0 &= \emptyset,\\
T_0 &= (V_0, E_0),& \rho_0 &= \emptyset.
\end{align*}
Figure \ref{fig:example_noutlier_gnu} shows $R$ as the largest and the brightest coloured rectangle.

\paragraph{Recursive step} 
In order to reach $T_1$ from $T_0$, a dimension $d_R=1$ and a split point $z_R = 72.63$ were randomly chosen. Figure \ref{fig:example_noutlier_gnu} shows split points as purple lines in their respective rectangles. The recursive step starts with creating left and right descendants of $R$ as
\begin{align*}
 S_l & = \left\{\begin{smallmatrix}
 [105,20],& [95,25], &[95,15],\\
[90,30],&[90,20],&[90,10],
\end{smallmatrix}\right\},&
S_r &= \left\{\begin{smallmatrix}
    [25,100],&[30,90],&[20,90],\\
    [35,85], &[25,85],&[15,85],
\end{smallmatrix}\right\},\\
R_l &= R(S_l) = \langle 90, 105 \rangle \times \langle 10, 30 \rangle,&
R_r &= R(S_r) = \langle 15, 35 \rangle \times \langle 85, 100 \rangle,
\end{align*}
forming a new tree $T_1$ and a new function relation $\rho_1$ as follows:
\begin{align*}
V_1 &= \{R, R_l, R_r\},&
E_1 &= \{(R,R_l), (R,R_r)\},\\
T_1 &= (V_1, E_1),&
\rho_1 &= \{ (R, (72.63, 1))\}.
\end{align*}
The tree $T_1$ has the left leaf $R_l$ and the right leaf $R_r$; the ending condition is not met, that is $L_1 = \{R_l,R_r\}$. 

We continue the recursive step with two vertices.
For the left vertex $R_l$, the dimension $d_{R_l} =0$ and the split point $z_{R_L}= 103.08$ were chosen randomly, giving
\begin{align*}
S_{ll} &= \left\{\begin{smallmatrix}
    [95,25],& [95,15], &[90,30],\\
    [90,20],& [90,10],
\end{smallmatrix}\right\},&
S_{lr} &= \{[105,20]\},\\
R_{ll} &= \langle 90, 95 \rangle \times \langle 10, 30\rangle,&
R_{lr} &= \langle 105, 105 \rangle \times \langle 20, 20\rangle
\end{align*}
and $z_{R_r}= 20.32$, $d_{R_r} = 0$ for the right vertex $R_r$ respectively
\begin{align*}
S_{rl}&= \{[20,90],[15,85]\},&
S_{rr} &= \left\{\begin{smallmatrix}
    [25, 100],& [30,90],\\
    [35,85],& [25,85],
\end{smallmatrix}\right\},\\
%S_{rr}&= \{[25,100],[30,90],[35,85],[25,85]\}\\
R_{rl}&= \langle 15, 20 \rangle \times \langle 85, 90 \rangle,&
R_{rr}&= \langle 25, 35 \rangle \times \langle 85, 100 \rangle.
\end{align*}

With the rectangles prepared, we can assemble new vertices and edges and create a new $T_2$:
\begin{align*}
\rho_2 &= \{(R,(72.63,1), (R_l, (103.08, 0)), (R_r, (20.32, 0)) \},\\
V_2 &= \{ R, R_l, R_r, R_{lr}, R_{lr}, R_{rl}, R_{rr} \},\\
E_2 &= \{ (R,R_l),(R,R_r), (R_l,R_{ll}), (R_l,R_{lr}), (R_r,R_{rl}), (R_r,R_{rr}) \},\\
T_2 &= (V_2, E_2).
\end{align*}

\paragraph{Termination} With the $T_2$ created, we now have to check for the ending condition of the leaves. Since $|R_{lr}\cap S|=|S_{lr}| = 1$, the ending condition for the leaf $R_{lr}$ is met, and the new set of leaves $L_2$ for the next recursion step is
$$L_2 = \{R_{ll},R_{rl},R_{rr}\}.$$

This continues until we reach tree $T_5$ such that $T_5=T_6$ is the desired tree $T$ as shown in Figure \ref{fig:example_noutlier_tree_color}. 
\end{example}


\subsection{Evaluating decision tree $T$}
The evaluation of desired element $a$ starts in the root $R$ of the previously built tree $T$.
In a root, by applying function relation, $\rho(R) = (z,d)$, we obtain split point $z$ and dimension $d$.
The root $R$ of a tree $T$ has two descendants $R_l$, $R_r$, such that
$\forall r_l\in R_l; \pi_d(r_l) \le z$ and $\forall r_r\in R_r; \pi_d(r_r)  > z$.

If $\pi_d(a)\le z$, we visit $R_l$, else, that is $\pi_d(a) > z$, we visit $R_r$.
We continue in this manner until we reach the leaf. Figures \ref{fig:example_noutlier_gnu} and \ref{fig:example_noutlier_tree_color} show the final depths of individual leaves. Based on them, we can decide on the level of outlierness. The deeper the evaluated point in the tree, the less anomalous it gets.

\begin{example}
\label{ex:regular_point_evaluation_original}
    Consider evaluating the point $a = [105,20]$ on the tree $T$ built in Example \ref{example:original_tree_create}.

    We start with the root $R = \langle 15,105\rangle \times \langle 10, 100 \rangle$.
    By applying the function $\rho(R)$, we obtain the split point $z = 72.63$ and the dimension $d = 1$.
    The root $R$ has two descendants 
\begin{align*}
    &R_l = \langle 90,105\rangle \times \langle 10, 30 \rangle,&
    &R_r = \langle 15,35\rangle \times \langle 85, 100 \rangle,\\
    \intertext{such that}
    &\forall r_l\in R_l; \pi_1(r_l) \le 72.63,&
    &\forall r_r\in R_r; \pi_1(r_r) > 72.63.
\end{align*}
Now, by applying the projection $\pi_1$ on $a$, we obtain $20$, which is less than the split point $z = 72.63$, i.e.
$$\pi_1([105,20]) = 20 < 72.63.$$
We visit the vertex $R_l$ because the value obtained by applying the projection on any element of $R_l$ is smaller than $72.63$.

We reached the next recursive step. With the vertex $R_l$ visited, we apply the function $\rho(R_l)$, obtaining the new split point $z = 103.08$ and the new dimension $d = 0$.
The vertex $R_l$ has two descendants 
\begin{align*}
    &R_{ll} = \langle 90,95\rangle \times \langle 10, 30 \rangle,&
    &R_{lr} = \langle 105,105\rangle \times \langle 20,20 \rangle,\\
    \intertext{such that}
    &\forall r_{ll}\in R_{ll}; \pi_0(r_{ll}) \le 103.08,&
    &\forall r_{lr}\in R_{lr}; \pi_0(r_{lr}) > 103.08.
\end{align*}

We visit the vertex $R_{lr}$ because the value obtained by applying the projection $\pi_0$ on any element of $R_{lr}$ and also on $a$ is more than $103.08$.

Since $R_{lr}$ has no descendants, we reached the final leaf with the depth of~$2$.
\end{example}


\begin{example}
\label{ex:novelty_point_evaluation_original}
    Consider now the evaluation of $a' = [25,20]$ on the tree $T$ built in Example \ref{example:original_tree_create}. Note that $a'$ was not contained in the training set for building a tree.

    We start with the root $R = \langle 5,105\rangle \times \langle 10, 100 \rangle$.
    By applying function $\rho(R)$, we obtain the split point $z = 72.63$ and the dimension $d = 1$.
%     Root $R$ has two descendants 
% \begin{align*}
%     &R_l = \langle 90,105\rangle \times \langle 10, 30 \rangle,&
%     &R_r = \langle 15,35\rangle \times \langle 85, 100 \rangle,\\
%     \intertext{such that}
%     &\forall r_l\in R_l; \pi_1(r_l) \le 72.63,&
%     &\forall r_r\in R_r; \pi_1(r_r) > 72.63.
% \end{align*}
% Now, by applying projection $\pi_1$ on $a'$, we obtain $20$, which is less than split point $z = 72.63$, i.e.
We visit $R_l = \langle 90,105\rangle \times \langle 10, 30 \rangle$ 
because $\pi_1([25,20]) = 20 \le 72.63.$
% We visit the vertex $R_l$ because the value obtained by applying the projection on any element of $R_l$ is less than $72.63$.

We reached the next recursive step. Obtaining the new split point $z = 103.08$ and the new dimension $d = 0$, we visit $R_{ll} = \langle 90,95\rangle \times \langle 10, 30 \rangle$, since $\pi_0([25,20]) \le 103.08$.
% Vertex $R_l$ has two descendants 
% \begin{align*}
%     &R_{ll} = \langle 90,95\rangle \times \langle 10, 30 \rangle,&
%     &R_{lr} = \langle 105,105\rangle \times \langle 20,20 \rangle,\\
%     \intertext{such that}
%     &\forall r_{ll}\in R_{ll}; \pi_0(r_{ll}) \le 103.08,&
%     &\forall r_{lr}\in R_{lr}; \pi_0(r_{lr}) > 103.08.
% \end{align*}

% We visit the vertex $R_{ll}$ because the value obtained by applying the projection on any element of $R_{ll}$ is less than $103.08$.

This repeats recursively until the leaf $R_{llllr}$ is reached. Note that this is the leaf with the point $[90,20]$. The reached depth is 5, as shown in Figure \ref{fig:example_noutlier_tree_color} (the leaf is marked grey).


Figure \ref{fig:example_noutlier_gnu} shows that after just two steps, the $[25,20]$ is no longer a part of any further evaluated rectangles.

% Figure \ref{fig:example_noutlier_gnu} shows that $[25,20]$ is evaluated in a way that it is left in the vertex with the point $[90,20]$, thus being evaluated as such.
\end{example}

Note that each element that was part of the batch during the training --- tree building --- phase is always contained in each vertex it visits. See $a \in R_{lr} \subset R_{l} \subset R$ in Example \ref{ex:regular_point_evaluation_original}.
This is not true for elements unseen during the training phase (such as novelty points).
See $a' \in R$, but $a' \notin R_l$ (and of course $a' \notin R_r$) in Example \ref{ex:novelty_point_evaluation_original}.

\begin{sidewaysfigure}[htbp]
\centering
\includesvg[inkscapelatex=false,width=1\textwidth]{figures/example6_Noutlier_tree_tb.svg}
\caption{tree constructed using the original approach}
\label{fig:example_noutlier_tree_color}
\end{sidewaysfigure}


\section{Proposed Range-based Enhancement For Isolation Forest}
\label{sec:novelty_isolation_forest}
In this section, we propose a new enhancement of the original Isolation Forest algorithm to make it possible to detect novelty observations.
The proposed enhancement takes the basic idea of an ensemble of trees with various depths but takes it further to make semi-supervised novelty detection possible.

\subsection{Initial Problem}
 The standard Isolation Forest algorithm cannot be used for novelty detection. This is because in each step, it limits the observation with the previously separated data.

Suppose we now want to use the original tree to evaluate the novelty data point $p$, which is not present in the training set.

Let us recall the equations needed for vertex creation:
\begin{align*}
S_l &= \{ \mathsf{s} \in{R \cap S}\ |\ \pi_{d_R}(\mathsf{s})\le z_R\},&
S_r &= \{ \mathsf{s} \in{R \cap S}\ |\ \pi_{d_R}(\mathsf{s}) > z_R\}.
\end{align*}
The point $p$ fits neither $S_l$ nor $S_r$ and does not even necessarily fit $R_l = R(S_l)$ nor $R_r = R(S_r)$. 

Nevertheless, because datapoint $p$ satisfies  $\pi_d(p) \le z$ (or $\pi_d(p) > z$), $p$ is assigned to a vertex $R_l$ (or $R_r$) even though $p \notin R_l$ nor $p \notin R_r$.

This is why the original Isolation Forest is a purely unsupervised algorithm. To work properly, the tree has to be constructed concerning all possible input data.

\subsection{Proposed Solution}
The proposed solution comes from the idea that the original tree lacks the possibility to isolate more datapoints than it currently observes.
The observed space is bounded by the minimum and maximum in each feature.

As in the original article, we use the concept of a binary decision tree. The proposed solution is altering the concept of the split point evaluation. Whereas the original Isolation Forest evaluates the split point based on the previous data, we
evaluate the split point based on a range in our proposed solution. For this to work, several alterations to the split point evaluation and the form of data passed between vertices must be made; however, the overall concept of the forest remains the same.
The proposed solution has two main concepts of alteration to the original solution.

\begin{enumerate}
    \item Each of the vertices gets assigned a space bounded by ranges. Each range should be reasonable enough to separate all the domain space correctly.
    \item The split point is in the middle of the given dimension’s range.
    \item The input observations are only used to determine the ending condition.
\end{enumerate}

\subsection{Constructing decision tree $T$}

\begin{enumerate}
    \item The maximum possible depth of a tree is controlled by the \emph{max\_depth} parameter.
    \item The sample \(S\) contains \emph{batch\_size} number of input datapoints.
    \item Leaves and internal vertices are possibility-space hyperrectangles \(R\). 
    \item Leaves' ending condition is satisfied by one of two criteria: 
\begin{enumerate}
    \item Vertex \(R\) satisfies \(S \cap R = \emptyset\) or \(| S \cap R | = 1\).
    \item Vertex's depth reached the \emph{max\_depth} value.
\end{enumerate}
\end{enumerate}



\subparagraph{Basis step}

Each dimension \(d \in\{0, \dots, n-1\}\), is bounded by the range \(r_d\). The ranges form the possibility-space hyperrectangle \(R_0\) as in:

\[R_0 =  r_0 \times r_1 \times \cdots \times r_{n-1}  \tag{xx}\,.\]

The trivial rooted tree \(T_0\) is a tuple with
vertices \(V_0 = \{R_0\}\) and edges \(E_0 = \emptyset\), i.e.
\[T_0= (V_0, E_0) = (\{R_0\},\emptyset).\]

\subparagraph{Recursive step}

The steps to reach the tree \(T_{j+1}\) from \(T_{j} = (V_j, E_j)\) are
as follows:

Let \(L_j \subseteq V_j\) be a subset of leaves not satisfying the
ending condition. 

For each leaf \(R \in L_j\) create two new vertices \(R_l, R_r\) by
selecting a random dimension \(d\).

Let
\begin{align}
R &= r_0 \times  \cdots \times r_{d-1} \times  r_d\times r_{d+1} \times \cdots \times r_{n-1},
\end{align}
where $r_d = \langle x, y ).$ %TODO: NEVIME k cemu to je

Then we obtain the left and right hyperrectangles \(R_l\), \(R_r\) as
follows:
\begin{align}
R_l &= r_1 \times  \cdots \times r_{d-1} \times  r_l \times r_{d+1} \times \cdots \times r_n \\
R_r &= r_1 \times  \cdots \times r_{d-1} \times  r_r \times r_{d+1} \times \cdots \times r_n
\end{align}
where \(r_l = \langle x, s )\) and \(r_r = \langle s, y )\) and \(s\) is
a number obtained as the middle of the range \(r_d\,\):

\[s = \frac{x + y}{2}\tag{x}.\] Each vertex \(R\) is associated with two new
edges \((R,R_l ), (R, R_r)\) as follows:
\begin{align*}
V_{j+1} &= V_j \cup \bigcup_{R \in L_j} \{R_l, R_r\},\\
E_{j+1} &= E_j \cup \bigcup_{R \in L_j} \{(R, R_l), (R,R_r)\},\\
T_{j+1} &= (V_{j+1}, E_{j+1}),
\end{align*}
i.e.~${R_l, R_r} \subset R$ are leaves in the new tree
\(T_{j+1}\).

Recursion is terminated if there is an equality of two consecutive trees \(T_j = T_{j+1}\). This happens when all leaves satisfy the ending condition, i.e., \(L_j = \emptyset\).
If this is the case, then the desired tree $T$ is the tree $T_j$ otherwise, move to the next recursion step.


Note that tree \(T_{j}\) is actually a Hasse diagram of the ordered set
\((V_j,\subseteq)\).


\begin{example}
\label{example:novelty_tree_create}
Consider now an example of creating a new enhanced Isolation tree based on the given input sample $S$.

\begin{align*}
    S = \{&[25,100],[30,90],[20,90],[35,85],\\
    &[25,85],[15,85],[105,20],[95,25], \\
    &[95,15],[90,30],[90,20],[90,10]\}
\end{align*}


\begin{figure}[htbp]
\centering
\includesvg[width=0.9\textwidth,inkscapelatex=false]{figures/example66_Novelty_gnu.svg}
\caption{example}
\label{fig:example_novelty_gnu}
\end{figure}

After selecting \emph{max depth} of 8 (experimentally), the tree $T$ is created by starting with tree $T_0$ and expanding further as described by the recursive step until the final condition is met.

Figure \ref{fig:example_novelty_gnu} shows the finished tree $T$, learned on dataset $S$.

\begin{enumerate}
    \item Basis step is to create a tree $T_0$ with one root vertex $R = \langle 0,110) \times \langle -5,105)$ with experimentally set initial possibility space and no edges $E_0$, such that
    \begin{align*}
        V_0 &= \{R\},&
        E_0 &= \emptyset,&
        T_0 &= (V_0, E_0)
    \end{align*}
    \item Since $R \in L_0$, we create two hyperrectangles $R_l$, $R_r$ by selecting a random dimension $d=0$.
    \begin{align*}
        r_0 &= \langle 0, 110) \\
        s &= \frac{0 + 110}{2} = 55 \\
        r_l &= \langle 0, 55) \\
        R_l &= \langle 0, 55) \times \langle -5,105) \\
        r_r &= \langle 55, 110) \\
        R_r &= \langle 55, 110) \times \langle -5,105)
    \end{align*}

    \item New tree $T_1$ is then
    \begin{align*}
    V_1 &= \{R, R_l, R_r\} \\
    E_1 &= \{(R, R_l), (R, R_r)\} \\
    T_1 &= (V_1, E_1)
    \end{align*}

    \item After checking for the ending condition, $L_1$ is as follows:
    $$L_1 = \{R_l, R_r\}$$

    \item Since there are vertices in $L_1$ left to be examined, we continue the next recursive step:
    Since $R_l \in L_1$ (resp. $R_r \in L_1$), we create two hyperrectangles $R_{ll}$, $R_{lr}$ --- left column (resp. $R_{rl}$, $R_{rr}$ -- right column) by selecting a random dimension $d=1$ (resp. $d=0$).
    \begin{align*}
        r_{l1} &= \langle -5, 105)& r_{r0} &= \langle 55, 110) \\
        s_l &= 50 & s_r&=82.5\\
        r_{ll} &= \langle -5, 50) & r_{rl} &= \langle 55, 82.5) \\
        R_{ll} &= \langle 0, 55) \times \langle -5,50) & R_{rl} &= \langle 55, 82.5) \times \langle -5,105)\\
        r_{lr} &= \langle 50, 105) & r_{rr} &= \langle 82.5, 110) \\
        R_{lr} &= \langle 0, 55) \times \langle 50,105) & R_{ll} &= \langle 82.5, 110) \times \langle -5,105)
    \end{align*}

    \item New tree $T_2$ is then
    \begin{align*}
        V_2 &= \{R, R_l, R_r, R_{ll}, R_{lr}, R_{rl}, R_{rr}\} \\
        E_2 &= \{(R, R_l), (R, R_r), (R_l, R_{ll}), (R_l, R_{lr}), (R_r, R_{rl}), (R_r, R_{rr})\} \\
        T_2 &= (V_2, E_2)
    \end{align*}

    \item After checking for the ending condition, $L_2$ is as follows:
    $$L_2 = \{R_{lr}, R_{rr}\}$$

    \item This goes on until one of the final conditions is met.

\end{enumerate}

\end{example}

\subsection{Evaluating enhanced decision tree $T$}
The evaluation of our enhanced decision tree is more straightforward since the examined datapoint is always contained in the possibility space hyperrectangle of some vertex in each depth until a leaf is visited.
The evaluation starts in the root vertex. The initial possibility space should be reasonable enough to cover the whole domain of a given problem.
Until the leaf is reached, the examined point recursively visits the ancestor within which it is contained.



\begin{example}
\label{ex:regular_point_evaluation_novelty}
    Consider now the evaluation of $a = [105,20]$ on tree $T$ built in Example \ref{example:novelty_tree_create}.

\begin{enumerate}
    \item  We start with the root $R = \langle 0,110\rangle \times \langle -5, 105 \rangle$.
    Root $R$ has two ancestors 
\begin{align*}
    &R_l = \langle 0,55) \times \langle -5, 105),&
    &R_r = \langle 55,110) \times \langle -5, 105),
\end{align*}
Since $a \in R_r$, we visit $R_r$.
\item Vertex $R_r$ has two ancestors
\begin{align*}
    &R_{rl} = \langle 5,82.5) \times \langle -5, 105),&
    &R_{rr} = \langle 82.5,110) \times \langle -5, 105),
\end{align*}
Since $a \in R_{rr}$, we visit vertex $R_{rr}$.
\item
we continue recursively in this manner for another two steps until leaf $R_{rrlr}$ is reached. This leaf has a depth of $4$.

\end{enumerate}
   
\end{example}

\begin{example}
\label{ex:novelty_point_evaluation_novelty}
    Consider now the evaluation of $a' = [25,20]$ on tree $T$ built in Example \ref{example:novelty_tree_create}.

\begin{enumerate}
    \item  We start with the root $R = \langle 0,110\rangle \times \langle -5, 105 \rangle$.
    Root $R$ has two ancestors 
\begin{align*}
    &R_l = \langle 0,55) \times \langle -5, 105),&
    &R_r = \langle 55,110) \times \langle -5, 105),
\end{align*}
Since $a' \in R_l$, we visit $R_l$.
\item Vertex $R_l$ has two ancestors
\begin{align*}
    &R_{ll} = \langle 0,55) \times \langle -5, 50),&
    &R_{lr} = \langle 0,5) \times \langle 50, 105),
\end{align*}
Since $a' \in R_{ll}$, we visit vertex $R_{ll}$.
\item
Since $R_{ll}$ is a leaf, we end here, and the reached leaf's depth is $2$.
\end{enumerate}
\end{example}
Note that each evaluated point of the given possibility space is always contained in each vertex it visits.
Hence $a \in R_{rrlr} \subset R_{rrl} \subset R_{rr} \subset R_{r} \subset R$ in \ref{ex:regular_point_evaluation_novelty}
and $a' \in R_{ll} \subset R_{l} \subset R$ as shown in Example \ref{ex:novelty_point_evaluation_novelty}.


Consider a minified example depicted in \ref{fig:example_data}. In this example, two chunks of data are in the top left and bottom right corners, respectively. Then, we selected a specific data point that shares one dimension similar to the first chunk and the other with the second one.

Note that the data have no specific distribution. The point at the bottom left has a value of $P_x = [5,5]$ and is not present for the learning phase.

\begin{figure}[htbp]
\centering
\includesvg[width=0.9\textwidth]{figures/example6_experiment.svg}
\caption{Example figure}
\label{fig:example_data}
\end{figure}

%TODO: ODTADYKA TO SMAZEME

\paragraph{Solution: Original approach}
With the original approach being fully unsupervised, we feed the whole input (excluding the $P_x$) to the forest and examine the resulting tree.

First, the trivial binary tree $T_0$ is created (xx), with ranges being the min-max values of input.
By doing the recursive steps, the whole tree is constructed.
Figure \ref{fig:example_noutlier_tree_color} shows the constructed binary tree based on the input data.

The evaluation of $P_x$ is as follows:
\begin{itemize}
    \item In the first step, after randomly selecting the dimension $d=1$ and split point $z$ as in (x) $z = 64.37$, the $P_x$ visits the node $\langle 102.0, 105.0\rangle \times \langle 3.0, 7.0\rangle$
    Note that it is enough for the point to fit only the selected dimension.
    \item In the second step, the point visits the node $\langle 102.0, 105.0\rangle \times \langle 5.0, 7.0\rangle$.
    \item This continues until the final leaf $\langle 102.0, 105.0\rangle \times \langle 5.0, 5.0\rangle$ marked gray in Figure \ref{fig:example_noutlier_tree_color}) is reached. This leaf has a depth of 4.
    
\end{itemize}


If we look at the constructed tree in Figure \ref{fig:example_noutlier_tree_color}, we can see that the resulting leaf is relatively deep, considering the depth of the deepest leaf.


\begin{figure}[htbp]
\centering
\includesvg[angle=90,inkscapelatex=false,width=1\textwidth]{figures/example66_Noutlier_tree.svg}
\caption{Tree constructed using the original approach}
\label{fig:example_noutlier_tree_color}
\end{figure}



\paragraph{Solution: Novelty approach}

The novelty approach, on the other hand, is a semi-supervised method. We feed the whole input (excluding the $P_x$) to the forest and examine the resulting tree.

First, the trivial binary tree $T_0$ with initial ranges is created (xx).
By doing the recursive steps, the whole tree is constructed.
Figure \ref{fig:example_novelty_tree_color} shows the constructed binary tree based on the input data.

The evaluation of $P_x$ is as follows:
\begin{itemize}
    \item In the first step, since the root node has dimension $d=1$ and split point $z = 54$ assigned during the training phase, the point visits the node $\langle 3.0, 54.0\rangle \times \langle 3.0, 105.0\rangle$
    \item In the second step, the point visits the node $\langle 3.0, 28.5\rangle \times \langle 3.0, 105.0\rangle$.
    \item This continues until the final leaf (marked grey in Figure \ref{fig:example_novelty_tree_color}) is reached. This leaf has a depth of 3.
    
\end{itemize}


Now, if we look at the resulting tree in Figure \ref{fig:example_novelty_tree_color}, we can see that the resulting leaf is relatively close to the root considering the rest of the leaves.


\begin{figure}[htbp]
\centering
\includesvg[angle=90,inkscapelatex=false,width=1\textwidth]{figures/example66_Novelty_tree.svg}
\caption{Tree constructed using the enhanced novelty approach}
\label{fig:example_novelty_tree_color}
\end{figure}


\paragraph{Proof: Novelty approach}
To prove the resulting depth using the novelty approach, we calculate the expected value of depth for each point $p$, considering all possible trees that would isolate $p$ and their respective paths.
We start with given range $\langle 0, 110.0\rangle \times \langle -5.0, 105.0\rangle$.

% 3Hy
The first point to consider is the $[25,100]$. Here, there is only one way to isolate this point: using three horizontal splits. Since the vertical splits do not matter in this case, there could be any number of them; hence, we get the expected value of depth as:

$$\sum_{n=3}^{\infty}\binom{n-1}{2}\cdot \frac{1}{2^n}\cdot n = 6$$    

% 4Vx or 2Vx + 5Hy
The second point in our sample is $[20,90]$. Here, there are more ways to isolate this point. That is by exactly four vertical splits (and any number of horizontals) or by exactly five horizontal and two vertical splits. To get the expected value for all the possible scenarios, we must calculate their probabilities. 

In the first scenario - exactly four vertical splits - we calculate the probabilities of four vertical splits and any number of horizontal up to five. Five horizontal splits scenario interferes with the second possible isolation:

\begin{center}
\begin{tabular}{||c c||} 
 \hline
 Depth & Probability \\ [1ex] 
 \hline
 4 & $\binom{3}{3}\cdot \frac{1}{2^4}$\\ 
 \hline
 5 & $\binom{4}{3}\cdot \frac{1}{2^5}$  \\
 \hline
 6 & $\binom{5}{3}\cdot \frac{1}{2^6}$  \\
 \hline
 7 & $\binom{6}{3}\cdot \frac{1}{2^7}$  \\
 \hline
 8 & $\binom{7}{3}\cdot \frac{1}{2^8}$ \\ [1ex] 
 \hline
\end{tabular}
\end{center}

We can generalize this as:

$$\sum_{n=4}^{8}\binom{n-1}{3}\cdot \frac{1}{2^n}\cdot n$$

To simplify the second scenario - two vertical and five horizontal splits - we divide it into two sub-scenarios.
\begin{enumerate}
    \item Sub-scenario - the last split is vertical (vysvetli proc zaciname sedmickou)

\begin{center}
\begin{tabular}{||c c||} 
 \hline
 Depth & Probability \\ [1ex] 
 \hline
 7 & $\binom{6}{1}\cdot \frac{1}{2^7}$\\ 
 \hline
 8 & $\binom{7}{1}\cdot \frac{1}{2^8}$  \\
 \hline
 9 & $\binom{8}{1}\cdot \frac{1}{2^9}$  \\
 \hline
 ... & ... \\ [1ex] 
 \hline
\end{tabular}
\end{center}
We generalize this as:
$$\sum_{n=7}^{\infty}\binom{n-1}{1}\cdot \frac{1}{2^n}\cdot n$$
\item Sub-scenario - the last split is horizontal
\begin{center}
\begin{tabular}{||c c||} 
 \hline
 Depth & Probability \\ [1ex] 
 \hline
 7 & $\binom{6}{4}\cdot \frac{1}{2^7}$\\ 
 \hline
 8 & $\binom{7}{4}\cdot \frac{1}{2^8}$  \\[1ex] 
 \hline
\end{tabular}
\end{center}
We generalize this as: 
$$\sum_{n=7}^{8}\binom{n-1}{4}\cdot \frac{1}{2^n}\cdot n$$
\end{enumerate}

Note that the sum of all the scenarios should equal to 1.
$$\sum_{n=4}^{8}\binom{n-1}{3}\cdot \frac{1}{2^n} +
\sum_{n=7}^{\infty}\binom{n-1}{1}\cdot \frac{1}{2^n} +
\sum_{n=7}^{8}\binom{n-1}{4}\cdot \frac{1}{2^n} = 1
$$

The third point in the sample is $[15,85]$. We need four vertical splits or two vertical and five horizontal splits to isolate this point. That is the same scenario as the previous point $[20,90]$; hence we omit this.

%4. (30,90) 5Vx or 2Vx + 5Hy
The fourth point in the sample is $[30,90]$. We need either five vertical splits or two vertical and five horizontal splits to isolate this point.

Similarly to the previous point, we simplify this by creating more sub-scenarios.
The first scenario - five horizontal splits - is 
$$\sum_{n=5}^{9}\binom{n-1}{4}\cdot \frac{1}{2^n}\cdot n$$

The second scenario is once more dependent on whether the last split was horizontal or vertical.

\begin{enumerate}
    \item Sub-scenario - the last split is vertical
    $$\sum_{n=7}^{\infty}\binom{n-1}{1}\cdot \frac{1}{2^n}\cdot n$$
    \item Sub-scenario - the last split is horizontal
    $$\sum_{n=7}^{9}\binom{n-1}{4}\cdot \frac{1}{2^n}\cdot n$$
\end{enumerate}

The fifth point in the sample is $[35,85]$.  We need four vertical splits to isolate this point. That is four vertical splits and any number of horizontal splits before, generalized as:

$$\sum_{n=4}^{\infty}\binom{n-1}{3}\cdot \frac{1}{2^n}\cdot n$$

%5Vx + 3Hy || 4Vx + 5Hy
The sixth point in the sample is $[25,85]$. We need five vertical and three horizontal splits or four vertical and five horizontal splits to isolate this point.
Let us begin with the first scenario, five vertical and three horizontal splits. We divide it into more sub-scenarios:

\begin{enumerate}
    \item Sub-scenario - the last split is vertical
\begin{center}
\begin{tabular}{||c c||} 
 \hline
 Depth & Probability \\ [1ex] 
 \hline
 8 & $\binom{7}{4}\cdot \frac{1}{2^8}$\\ 
 \hline
 9 & $\binom{8}{4}\cdot \frac{1}{2^9}$  \\[1ex] 
 \hline
\end{tabular}
\end{center}
We generalize this as: 
$$\sum_{n=8}^{9}\binom{n-1}{4}\cdot \frac{1}{2^n}\cdot n$$
\item Sub-scenario - the last split is horizontal
    \begin{center}
\begin{tabular}{||c c||} 
 \hline
 Depth & Probability \\ [1ex] 
 \hline
 8 & $\binom{7}{2}\cdot \frac{1}{2^8}$\\ 
 \hline
 9 & $\binom{8}{2}\cdot \frac{1}{2^9}$  \\[1ex] 
 \hline
\end{tabular}
\end{center}
We generalize this as: 
$$\sum_{n=8}^{\infty}\binom{n-1}{2}\cdot \frac{1}{2^n}\cdot n$$
\end{enumerate}

In the second scenario, four vertical and five horizontal, we are left with the last two sub-scenarios:

\begin{enumerate}
    \item Sub-scenario - the last split is vertical
\begin{center}
\begin{tabular}{||c c||} 
 \hline
 Depth & Probability \\ [1ex] 
 \hline
 9 & $\binom{8}{3}\cdot \frac{1}{2^9}$\\ 
 \hline
 10 & $\binom{9}{3}\cdot \frac{1}{2^10}$  \\
 \hline
 ... & ... \\ [1ex] 
  \hline
\end{tabular}
\end{center}
We generalize this as: 
$$\sum_{n=9}^{\infty}\binom{n-1}{3}\cdot \frac{1}{2^n}\cdot n$$
\item Sub-scenario - the last split is horizontal. Here, only 9 splits are possible since in ten splits, there would be five vertical splits, making it the previous scenario.
    \begin{center}
\begin{tabular}{||c c||} 
 \hline
 Depth & Probability \\ [1ex] 
 \hline
 9 & $\binom{8}{4}\cdot \frac{1}{2^9}$\\ [1ex]
 \hline
\end{tabular}
\end{center}
\end{enumerate}

Using this, we can prove that our novelty point $[25,20]$ gets the expected depth of 3:

$$\sum_{n=1}^{\infty}\binom{n+1}{1}\cdot \frac{1}{2^n}\cdot n = 3$$


\paragraph{Proof: Original approach}
To show the resulting depth using the original approach, we find all possible paths that would isolate the given point. Then, for each of the possible paths, we calculate its probability.

% Node(Union{Number, Tuple{Vararg{T, N}} where {N, T}}[(25, 100)], BigRationals.BigRational(1,
% 179159040), 7, @NamedTuple{dim::Int64, range::UnitRange{Int64}}[(dim = 1, range = 95:105), (dim = 2, range = 10:15), (dim = 1, range = 90:95), (dim = 2, range = 20:30), (dim = 1, range = 30:35), (dim = 1, range = 15:20), (dim = 2, range = 90:100)])

% na zacakatu name X moznosti jak vybrat splitpoint

% pro nejaky bod vam ukazeme jak se pocita pravedpodobnost
% julie nam ukaze split dehpth (12 SP 345)
% pravepodobnost prvniho splitpointu (geometricka) sprangemax-sprangemin/rangemax - rangemin
% viz tabule 1/2* cosi minus cosi, jak se pocita je v julii
% tim poslednim split pointem je bod tedy izolovan
% jeste to ukazeme pro novelty bod, ukazeme ze zustane v hrnicku s bodem nahore a nebo s bodem vpravo dole
% tim ukazeme pravdepodobnost pro jeden pripad, je toho moc tak viz tabulka
% Node.new(data => $($[25, 85],), prob => <11/18662400>, depth => 7, split => [0 => 95..105, 0 => 90..95, 1 => 85..90, 1 => 10..20, 0 => 35..90, 0 => 25..35, 0 => 15..25])
% 15-105
% 15-20-25-30-35-90-95-105

For the point $[25,85]$, consider one of the possible paths that orphan the given point.

We start with the whole observable space, a range of $\langle 15, 105\rangle$ for the first dimension $x$ and $\langle 10,100\rangle$ for the dimension $y$.
First, the dimension $x$ and a split point in the $\langle 95, 105\rangle$ range were chosen. The possible range where the split point could have been chosen was $\langle 15,105\rangle$.
The probability of selecting $x$ is $0.5$ due to the two-dimensional setting.
The probability of the split point being from the given range is $\frac{105-95}{105-15}$, where the nominator is the size of the favorable range, and the denominator is the size of the whole possible range.
% Let $p(\langle 95, 105\rangle_x)$ be the probability of $n$ splits.
The probability of such event is then $$\frac{1}{2}\cdot\frac{105-95}{105-15} = \frac{1}{18}.$$
Since the given point is not yet orphaned, we continue this way. The observable space is now scaled down due to the split point to the range of $\langle 15, 95\rangle$ for the first dimension $x$ and $\langle 10,100\rangle$ for the second $y$. Table \ref{prob_table_25_85} shows the rest of the probabilities for this path.

\begin{table}[h]
\centering
\begin{tblr}{
    width=\linewidth,
    hspan=minimal,
    cells={font=\footnotesize},
    colspec={c c c c c},
    %colsep=1pt,
    row{1}={guard},
    column{1-5}={mode=math}
}
Start space & Dim. & Split range & Probability & End space \\
\hline
\langle 15, 105\rangle \times \langle 10, 100\rangle & x & \langle 95, 105\rangle &  \frac{1}{2}\cdot\frac{105-95}{105-15} = \frac{1}{18} & \langle 15, 95\rangle \times \langle 10, 100\rangle \\
\langle 15, 95\rangle \times \langle 10, 100\rangle & x & \langle 90, 95\rangle &  \frac{1}{2}\cdot\frac{95-90}{95-15} = \frac{1}{32} & \langle 15, 90\rangle \times \langle 10, 100\rangle \\
\langle 15, 90\rangle \times \langle 10, 100\rangle & y & \langle 85, 90\rangle &  \frac{1}{2}\cdot\frac{90-85}{100-10} = \frac{1}{36} & \langle 15, 90\rangle \times \langle 10, 85\rangle \\
\langle 15, 90\rangle \times \langle 10, 85\rangle & y & \langle 10, 20\rangle &  \frac{1}{2}\cdot\frac{20-10}{85-10} = \frac{1}{15} & \langle 15, 90\rangle \times \langle 20, 85\rangle \\
\langle 15, 90\rangle \times \langle 20, 85\rangle & x & \langle 35, 90\rangle &  \frac{1}{2}\cdot\frac{90-35}{90-15} = \frac{11}{30} & \langle 15, 35\rangle \times \langle 85, 85\rangle \\
\langle 15, 35\rangle \times \langle 85, 85\rangle & x & \langle 25, 35\rangle &  \phantom{\frac{1}{2}\cdot}\frac{35-25}{35-15} = \frac{1}{2} & \langle 15, 25\rangle \times \langle 85, 85\rangle \\
\langle 15, 25\rangle \times \langle 85, 85\rangle & x & \langle 15, 25\rangle &  \phantom{\frac{1}{2}\cdot}\frac{25-15}{25-15} = \frac{1}{1} & \langle 25, 25\rangle \times \langle 85, 85\rangle
\end{tblr}
\caption{Probabilities of depths for point $[25,85]$.}
\label{prob_table_25_85}
\end{table}


% cela cesta je pak takto: - vypisu ty vypocty
% cela pravdepodobnost je pak ze to cele vynasobim
% a reknu aha path je tedy 7 protoze to dal neslo

Note that rows $6$ and $7$ no longer contain the probability of selecting the dimensions since the second dimension cannot be chosen as it would not isolate any point (see the startspace of $y$ as $\langle 85, 85\rangle$).
The evaluation ends after seven splits (depth = $7$) since that is the last split to isolate the given point.
The probability of this case is then $\frac{1}{18}\cdot\frac{1}{32}\cdot\dots\cdot\frac{1}{2}\cdot 1$.

If we do this for all possible cases and points, we get the probabilities for the individual depths that could result in orphaning a given point. Table \ref{table_big_original} shows the probabilities for all possible depths. This is later used to calculate the expected depth value.
Due to the symetricity, the probabilities for the points in the bottom right corner in the figure \ref{fig:example_noutlier_gnu} are the same as those in the top left corner, so we only show the latter.

Let us recall the initial problem in Section \ref{sec:InitialProblem}.
If we consider the novelty point $[25,20]$, the evaluation results in the same path as the point $[25,85]$ in Table \ref{prob_table_25_85}. However, the novelty point no longer fits in the start space for the sixth and seventh rows. Nevertheless, because datapoint $[25,20]$ satisfies  $25 \le z \in \langle 25, 35 \rangle$,  $[25,20]$ is assigned to a vertex $\langle 15, 25 \rangle \times \langle 85, 85 \rangle$ even though $[25,20] \notin \langle 15, 25 \rangle \times \langle 85, 85 \rangle$.

Note that the path for the novelty point $[25,20]$ is a path of the $[25,85]$ (the vertical neighbor) or the $[90,20]$ (the horizontal neighbor).



\begin{sidewaystable}[p]
\begin{tblr}{
    width=\linewidth,
    hspan=minimal,
    cells={font=\footnotesize},
    cell{1}{1-11}={halign=c},
    column{odd}={gray9},
    %colsep=1pt,
    colspec={
    c |
    S[round-mode=places ,round-precision=2, output-exponent-marker=E, table-format=1.2e+1]
    S[round-mode=places ,round-precision=2, output-exponent-marker=E, table-format=1.2e+1]
    S[round-mode=places ,round-precision=2, output-exponent-marker=E, table-format=1.2e+1]
    S[round-mode=places ,round-precision=2, output-exponent-marker=E, table-format=1.2e+1]
    S[round-mode=places ,round-precision=2, output-exponent-marker=E, table-format=1.2e+1]
    S[round-mode=places ,round-precision=2, output-exponent-marker=E, table-format=1.2e+1]
    S[round-mode=places ,round-precision=2, output-exponent-marker=E, table-format=1.2e+1]
    S[round-mode=places ,round-precision=2, output-exponent-marker=E, table-format=1.2e+1]S[round-mode=places ,round-precision=2, output-exponent-marker=E, table-format=1.2e+1]
    S[round-mode=places ,round-precision=2, output-exponent-marker=E, table-format=1.2e+1]
    S[round-mode=places ,round-precision=2, output-exponent-marker=E, table-format=1.2e+1]
    S[round-mode=places ,round-precision=2, output-exponent-marker=E, table-format=1.2e+1]
    },
    row{1}={guard},
    column{1}={guard, mode=math}
}
 \diagbox{Point}{Depth} & 1 & 2 & 3 & 4 & 5 & 6 & 7 & 8 & 9 & 10 & 11 \\
 \hline
\left[25, 100\right] & 5.5555555556E-02 & 2.5103323737E-01 & 2.5570744502E-01 & 2.4773022596E-01 & 1.3819675888E-01 & 4.3771373491E-02 & 7.2084454376E-03 & 7.4839864762E-04 & 4.6942907037E-05 & 1.5970541667E-06 & 1.9690558403E-08\\
\left[20, 90\right] & 0 & 3.5539215686E-02 & 1.7775615069E-01 & 4.0676493336E-01 & 2.6984418116E-01 & 9.2561204648E-02 & 1.5747066825E-02 & 1.6759728793E-03 & 1.0748807222E-04 & 3.7394349905E-06 & 4.7254206811E-08\\
\left[30, 90\right] & 0 & 1.3368055556E-02 & 1.6859439869E-01 & 4.1552713020E-01 & 2.8292729344E-01 & 9.9767533944E-02 & 1.7705833609E-02 & 1.9721214421E-03 & 1.3271521257E-04 & 4.8533703768E-06 & 6.4529234241E-08\\
\left[35, 85\right] & 0 & 9.7486772487E-02 & 2.5792650002E-01 & 3.2462216793E-01 & 2.4112764708E-01 & 6.6916307461E-02 & 1.0752400748E-02 & 1.0967974134E-03 & 6.8988815370E-05 & 2.3864551481E-06 & 3.1601016673E-08 \\
\left[25, 85\right] & 0 & 0 & 2.4722562636E-02 & 3.0688106140E-01 & 4.7133742736E-01 & 1.6542794157E-01 & 2.8432586746E-02 & 3.0013428173E-03 & 1.9044983168E-04 & 6.5444606635E-06 & 8.3183446589E-08 \\
\left[15, 85\right] & 2.7777777778E-02 & 1.2482638889E-01 & 2.5516544084E-01 & 3.0935396980E-01 & 2.1892437075E-01 & 5.5415608206E-02 & 7.8217752227E-03 & 6.7869855413E-04 & 3.5013796136E-05 & 9.4653744554E-07 & 9.6253761386E-09\\
\hline
\left[20, 25\right] & 0 & 3.5693536674E-02 & 1.8350184759E-01 & 3.4881778276E-01 & 3.4168298237E-01 & 8.0460443076E-02 & 9.2155826505E-03 & 6.0623378611E-04 & 2.1232873030E-05 & 3.5512746939E-07 & 3.0996586909E-09
\end{tblr}
\caption{Probabilities for individual data points, original approach.}
\label{table_big_original}
\end{sidewaystable}


\paragraph{Proof: Novelty approach}
To prove the resulting depth using the novelty approach, we calculate the expected value of depth for each point $p$, considering all possible trees that would isolate $p$ and their respective paths. Contrary to the previous proof, we can get infinite splits, resulting in infinite depth.
We start with given range $\langle 0, 110.0\rangle \times \langle -5.0, 105.0\rangle$.

% 3Hy
The first point to consider is the $[25,100]$. There is only one way to isolate this point: using three horizontal splits. This is the only scenario S1 for this point. Table \ref{table_25_100} shows the probabilities for individual depths. Since the vertical splits do not matter in this case, there could be any number of them; hence, we get the expected value of depth as:

$$\sum_{n=3}^{\infty}\binom{n-1}{2}\cdot \frac{1}{2^n}\cdot n = 6.$$

%first point 25,100
\begin{table}[h]
\centering
\begin{tblr}{
    width=\linewidth,
    hspan=minimal,
    cells={font=\footnotesize},
    colspec={c| c |c},
    column{odd}={gray9},
    %colsep=1pt,
    row{1}={guard},
    column{1-3}={guard, mode=math}
}
 \diagbox{Depth}{Probab.} & S1 & \sum \\
 \hline
3 & \binom{2}{2}\cdot\frac{1}{2^3} & \frac{1}{8} \\
4 & \binom{3}{2}\cdot\frac{1}{2^4} & \frac{3}{16} \\
5 & \binom{4}{2}\cdot\frac{1}{2^5} & \frac{3}{16} \\
\vdots & \vdots & \vdots  \\
k & \binom{k-1}{2}\cdot \frac{1}{2^k} & \binom{k-1}{2}\cdot \frac{1}{2^k} \\
\vdots & \vdots & \vdots \\
\hline
\sum & 1 & 1
\end{tblr}
\caption{Probabilities of depths for point $[25,100]$.}
\label{table_25_100}
\end{table}


% 4Vx or 2Vx + 5Hy
The second point in our sample is $[20,90]$. Here, there are more ways to isolate this point. That is by exactly four vertical splits (and any number of horizontals --- up to five --- S1) or by exactly five horizontal and two or three vertical splits (S2X). To get the expected value for all the possible scenarios, we must calculate their probabilities. Table \ref{table_20_90} shows the probabilities for individual depths.

To simplify the second scenario - two vertical and five horizontal splits - we divide it into two sub-scenarios.
\begin{enumerate}
    \item Sub-scenario (S2V), where the last split is vertical.
    \item Sub-scenario (S2H), where the last split is horizontal.
\end{enumerate}

The expected value of depth for the point $[20,90]$ is:
$$\sum_{n=4}^{8}\binom{n-1}{3}\cdot \frac{1}{2^n}\cdot n + \sum_{n=7}^{\infty}\binom{n-1}{1}\cdot \frac{1}{2^n}\cdot n + \sum_{n=7}^{8}\binom{n-1}{4}\cdot \frac{1}{2^n}\cdot n \doteq 6.82$$
%second point [20,90]
\begin{table}[h]
\centering
\begin{tblr}{
    width=\linewidth,
    hspan=minimal,
    cells={font=\footnotesize},
    colspec={c| c c c | c},
    column{odd}={gray9},
    %colsep=1pt,
    row{1}={guard},
    column{1-5}={guard, mode=math}
}
 \diagbox{Depth}{Probab.} & S1 & S2V & S2H & \sum \\
 \hline
4 & \binom{3}{3}\cdot \frac{1}{2^4} & 0 & 0 & \frac{1}{16} \\
5 & \binom{4}{3}\cdot\frac{1}{2^5}  &  0 & 0 & \frac{1}{8}\\
6 & \binom{5}{3}\cdot\frac{1}{2^6}  &  0 & 0& \frac{5}{32}\\
7 & \binom{6}{3}\cdot\frac{1}{2^7}  & \binom{6}{1}\cdot\frac{1}{2^7} & \binom{6}{4}\cdot\frac{1}{2^7} & \frac{41}{128} \\
8 & \binom{7}{3}\cdot\frac{1}{2^8}  & \binom{7}{1}\cdot\frac{1}{2^8} & \binom{7}{4}\cdot\frac{1}{2^8} & \frac{77}{256}\\
9 & 0 & \binom{8}{1}\cdot\frac{1}{2^9} & 0 & \frac{1}{64}\\
\vdots & \vdots & \vdots & \vdots & \vdots\\
k & 0 & \binom{k-1}{1}\cdot \frac{1}{2^k} & 0 & (k-1)\cdot\frac{1}{2^k}\\
\vdots & \vdots & \vdots & \vdots & \vdots \\
\hline
\sum & \frac{163}{256} & \frac{7}{64} & \frac{65}{256} & 1
\end{tblr}
\caption{Probabilities of depths for point $[20,90]$.}
\label{table_20_90}
\end{table}


The third point in the sample is $[15,85]$. We need four vertical splits or two vertical and five horizontal splits to isolate this point. That is the same scenario as the previous point $[20,90]$; hence we omit this.

%4. (30,90) 5Vx or 2Vx + 5Hy
The fourth point in the sample is $[30,90]$. We need either exactly five vertical splits (S1) or two vertical and at least five horizontal splits to isolate this point. That is, again, two sub-scenarios. First, the last split is vertical (S2V), and conversely, the last is horizontal (S2H).
Table \ref{table_30_90} shows the probabilities for individual depths.


% In the first scenario (S1), we consider exactly five vertical splits.
% To simplify the second scenario - two vertical and five horizontal splits - we divide it into two sub-scenarios.
% \begin{enumerate}
%     \item Sub-scenario (S2V), where the last split is vertical.
%     \item Sub-scenario (S2H), where the last split is horizontal.
% \end{enumerate}

The expected value of depth for the point $[30,90]$ is:
$$\sum_{n=5}^{9}\binom{n-1}{4}\cdot \frac{1}{2^n}\cdot n + \sum_{n=7}^{\infty}\binom{n-1}{1}\cdot \frac{1}{2^n}\cdot n + \sum_{n=7}^{9}\binom{n-1}{4}\cdot \frac{1}{2^n}\cdot n \doteq 7.82$$



%ctvrty point [30,90]
\begin{table}[h]
\centering
\begin{tblr}{
    width=\linewidth,
    hspan=minimal,
    cells={font=\footnotesize},
    colspec={c | c c c | c },
    column{odd}={gray9},
    %colsep=1pt,
    row{1}={guard},
    column{1-5}={guard, mode=math}
}
 \diagbox{Depth}{Probab.} & S1 & S21 & S22 & \sum  \\
 \hline
5 & \binom{4}{4}\cdot\frac{1}{2^5} & 0 & 0 & \frac{1}{32}\\
6 & \binom{5}{4}\cdot\frac{1}{2^6} & 0 & 0 & \frac{5}{64}\\
7 & \binom{6}{4}\cdot\frac{1}{2^7} & \binom{6}{1}\cdot\frac{1}{2^7} & \binom{6}{4}\cdot\frac{1}{2^7} & \frac{9}{32}\\
8 & \binom{7}{4}\cdot\frac{1}{2^8} & \binom{7}{1}\cdot\frac{1}{2^8} & \binom{7}{4}\cdot\frac{1}{2^8} & \frac{77}{256}\\
9 & \binom{8}{4}\cdot\frac{1}{2^9} & \binom{8}{1}\cdot\frac{1}{2^9} & \binom{8}{4}\cdot\frac{1}{2^9} & \frac{37}{128}\\
10 & 0 & \binom{9}{1}\cdot\frac{1}{2^{10}} & 0 & \frac{9}{1024}\\
\vdots & \vdots & \vdots & \vdots & \vdots \\
k & 0 & \binom{k-1}{1}\cdot \frac{1}{2^k} & 0 & (k-1)\cdot \frac{1}{2^k} \\
\vdots & \vdots & \vdots & \vdots & \vdots \\
\hline
\sum & \frac{1}{2} & \frac{7}{64} & \frac{25}{64} & 1
\end{tblr}
\caption{Probabilities of depths for point $[30,90]$.}
\label{table_30_90}
\end{table}


The fifth point in the sample is $[35,85]$.  We need four vertical splits to isolate this point. There are four vertical splits and any number of horizontal splits before. We get the expected value of depth as follows:

$$\sum_{n=4}^{\infty}\binom{n-1}{3}\cdot \frac{1}{2^n}\cdot n = 8$$

Table \ref{table_35_85} shows the probabilities for individual depths for the fifth point.

%paty point [35,85]
\begin{table}[h]
\centering
\begin{tblr}{
    width=\linewidth,
    hspan=minimal,
    cells={font=\footnotesize},
    colspec={c | c | c},
    column{odd}={gray9},
    %colsep=1pt,
    row{1}={guard},
    column{1-3}={guard, mode=math}
}
 \diagbox{Depth}{Probab.} & S1 & \sum \\
 \hline
4 & \binom{3}{3}\cdot\frac{1}{2^4} & \frac{1}{16} \\
5 & \binom{4}{3}\cdot\frac{1}{2^5} &  \frac{1}{8}\\
6 & \binom{5}{3}\cdot\frac{1}{2^6} & \frac{5}{32} \\
\vdots & \vdots & \vdots \\
k & \binom{k-1}{3}\cdot \frac{1}{2^k} & \binom{k-1}{3}\cdot \frac{1}{2^k}\\
\vdots & \vdots & \vdots \\
\hline
\sum & 1 & 1
\end{tblr}
\caption{Probabilities of depths for point $[35,85]$.}
\label{table_35_85}
\end{table}


%5Vx + 3Hy || 4Vx + 5Hy
The sixth point in the sample is $[25,85]$. We need five vertical and three or four horizontal splits (S1V and S1H) or four vertical and at least five horizontal splits (S2V and S2H) to isolate this point.
Table \ref{table_25_85} shows the probabilities for individual depths for the above scenarios.
% Let us begin with the first scenario, five vertical and three horizontal splits. We divide it into more sub-scenarios:

% \begin{enumerate}
%     \item Sub-scenario (S1V), where the last split is vertical.
%     \item Sub-scenario (S1H), where the last split is horizontal.
% \end{enumerate}

% \begin{enumerate}
%     \item Sub-scenario - the last split is vertical - is generalized as: 
% $$\sum_{n=8}^{9}\binom{n-1}{4}\cdot \frac{1}{2^n}\cdot n$$
% \item Sub-scenario - the last split is horizontal - is generalized as:  
% $$\sum_{n=8}^{\infty}\binom{n-1}{2}\cdot \frac{1}{2^n}\cdot n$$
% \end{enumerate}

% In the second scenario, four vertical and five horizontal splits, we are left with the last two sub-scenarios:

% \begin{enumerate}
%     \item Sub-scenario (S2V), where the last split is vertical.
%     \item Sub-scenario (S2H), where the last split is horizontal. Here, only 9 splits are possible since in ten splits, there would be five vertical splits, making it the previous scenario.
% \end{enumerate}

The expected value of depth for the point $[25,85]$ is:
$$\sum_{n=8}^{9}\binom{n-1}{4}\cdot \frac{1}{2^n}\cdot n + \sum_{n=8}^{\infty}\binom{n-1}{2}\cdot \frac{1}{2^n}\cdot n + \sum_{n=9}^{\infty}\binom{n-1}{3}\cdot \frac{1}{2^n}\cdot n + \binom{8}{4}\cdot \frac{1}{2^9} \doteq 9.734$$


% \begin{enumerate}
%     \item Sub-scenario - the last split is vertical - can be generalized as:
% $$\sum_{n=9}^{\infty}\binom{n-1}{3}\cdot \frac{1}{2^n}\cdot n$$
% \item Sub-scenario - the last split is horizontal. ; hence, for the depth of 9, we get:
% $$\binom{8}{4}\cdot \frac{1}{2^9}$$
% \end{enumerate}


%sesty point [25,85]
\begin{table}[h]
\centering
\begin{tblr}{
    width=\linewidth,
    hspan=minimal,
    cells={font=\footnotesize},
    colspec={c | c c c c | c},
    column{odd}={gray9},
    %colsep=1pt,
    row{1}={guard},
    column{1-6}={guard, mode=math}
}
 \diagbox{Depth}{Probab.} & S1V & S1H & S2V & S2H & \sum \\
 \hline
8 & \binom{7}{4}\cdot\frac{1}{2^8} &  \binom{7}{2}\cdot\frac{1}{2^8} & 0 & 0 & \frac{7}{32}\\
9 & \binom{8}{4}\cdot\frac{1}{2^9} & \binom{8}{2}\cdot\frac{1}{2^9} & \binom{8}{3}\cdot\frac{1}{2^9} & \binom{8}{4}\cdot\frac{1}{2^9} & \frac{7}{16} \\
10 & 0 & \binom{9}{2}\cdot\frac{1}{2^{10}} & \binom{9}{3}\cdot\frac{1}{2^{10}} & 0 & \frac{15}{128}\\
\vdots & \vdots & \vdots & \vdots & \vdots & \vdots \\
k & 0 & \binom{k-1}{2}\cdot\frac{1}{2^k} & \binom{k-1}{3}\cdot \frac{1}{2^k} & 0 & \binom{k}{3}\cdot \frac{1}{2^k}  \\
\vdots & \vdots & \vdots & \vdots & \vdots & \vdots \\
\hline
\sum & \frac{35}{128} & \frac{29}{128} & \frac{93}{256} & \frac{35}{256} & 1
\end{tblr}
\caption{Probabilities of depths for point $[25,85]$.}
\label{table_25_85}
\end{table}



We can prove that our novelty point $[25,20]$ gets the expected depth 3. Table \ref{table_novelty} shows the probabilities for the individual depths. When orphaning this point, we only have two possible scenarios.

If the last split is horizontal, we get:

$$\sum_{n=2}^{\infty}\binom{n-1}{0}\frac{1}{2^{n}}\cdot n = 1.5$$

Conversely, if the last split is vertical, we get:
$$\sum_{n=2}^{\infty}\binom{n-1}{0}\frac{1}{2^{n}}\cdot n = 1.5$$

This can be simplified as:

$$\sum_{n=2}^{\infty}\frac{1}{2^{n-1}}\cdot n = 3$$

The expected value of depth of $[25,85]$ is 3.


%novelty 
\begin{table}[h]
\centering
\begin{tblr}{
    width=\linewidth,
    hspan=minimal,
    cells={font=\footnotesize},
    colspec={c| c c |c},
    column{odd}={gray9},
    %colsep=1pt,
    row{1}={guard},
    column{2-5}={guard, mode=math}
}
 \diagbox{Depth}{Probab.} & S1V & S1H & \sum \\
 \hline
2 & \binom{1}{0}\cdot\frac{1}{2^2} &  \binom{1}{0}\cdot\frac{1}{2^2} & \frac{1}{2}\\
3 & \binom{2}{0}\cdot\frac{1}{2^3} & \binom{2}{0}\cdot\frac{1}{2^3} &  \frac{1}{4}\\
4 & \binom{3}{0}\cdot\frac{1}{2^4} & \binom{3}{0}\cdot\frac{1}{2^4} & \frac{1}{8}\\
\vdots & \vdots & \vdots &\vdots \\
k & \binom{k-1}{0}\cdot\frac{1}{2^k} & \binom{k-1}{0}\cdot\frac{1}{2^k}&\frac{1}{2^{k-1}} \\
\vdots & \vdots & \vdots & \vdots\\
\hline
\sum & \frac{1}{2} & \frac{1}{2} & 1 \\
\end{tblr}
\caption{Probabilities of depths for the novelty point $[25,20]$.}
\label{table_novelty}
\end{table}

Table \ref{table_big_novelty} shows the aggregated sums for individual depths for comparison with the expected values for the original approach in table \ref{table_big_original}.

% stars tabulka se zlomky
% \begin{sidewaystable}[p]
% \label{table_big_novelty_old}
% \begin{tblr}{
%     width=\linewidth,
%     hspan=minimal,
%     cells={font=\footnotesize},
%     cell{1}{1-11}={halign=c},
%     column{odd}={gray9},
%     %colsep=1pt,
%     colspec={
%     c |
%     S[round-mode=places ,round-precision=2, output-exponent-marker=E, table-format=1.2e+1]
%     S[round-mode=places ,round-precision=2, output-exponent-marker=E, table-format=1.2e+1]
%     S[round-mode=places ,round-precision=2, output-exponent-marker=E, table-format=1.2e+1]
%     S[round-mode=places ,round-precision=2, output-exponent-marker=E, table-format=1.2e+1]
%     S[round-mode=places ,round-precision=2, output-exponent-marker=E, table-format=1.2e+1]
%     S[round-mode=places ,round-precision=2, output-exponent-marker=E, table-format=1.2e+1]
%     S[round-mode=places ,round-precision=2, output-exponent-marker=E, table-format=1.2e+1]
%     S[round-mode=places ,round-precision=2, output-exponent-marker=E, table-format=1.2e+1]S[round-mode=places ,round-precision=2, output-exponent-marker=E, table-format=1.2e+1]
%     S[round-mode=places ,round-precision=2, output-exponent-marker=E, table-format=1.2e+1]
%     S[round-mode=places ,round-precision=2, output-exponent-marker=E, table-format=1.2e+1]
%     S[round-mode=places ,round-precision=2, output-exponent-marker=E, table-format=1.2e+1]
%     },
%     row{1}={guard},
%     column{2-12}={mode=math},
%     column{1}={guard, mode=math}
% }
%  \diagbox{Point}{Depth} & 1 & 2 & 3 & 4 & 5 & 6 & 7 & 8 & 9 & 10 & 11 \\
%  \hline
% \left[25, 100\right] & 0 & 0 & \frac{1}{8} & \frac{3}{16} & \frac{3}{16} & \frac{5}{32} & \frac{15}{128} & \frac{21}{256} & \frac{7}{128} & \frac{9}{256} & \frac{55}{4096} \\
% \left[20, 90\right] & 0 & 0 & 0 & \frac{1}{16} & \frac{1}{8} & \frac{5}{32} & \frac{41}{128} & 
% \frac{77}{256} & \frac{1}{64} & \frac{9}{1024} & \frac{5}{1024}\\
% \left[30, 90\right] & 0 & 0 & 0 & 0 & \frac{1}{32} & \frac{5}{64} & \frac{9}{32} & \frac{77}{256} & \frac{37}{128} & \frac{9}{1024} & \frac{5}{1024}\\
% \left[35, 85\right] & 0 & 0 & 0 & \frac{1}{16} & \frac{1}{8} & \frac{5}{32} & \frac{5}{32} & \frac{35}{256} & \frac{7}{64} & \frac{21}{256} & \frac{15}{256} \\
% \left[25, 85\right] & 0 & 0 & 0 & 0 & 0 & 0 & 0 & \frac{7}{32} & \frac{7}{16} & \frac{15}{128} & \frac{165}{2048} \\
% \left[15, 85\right] & 0 & 0 & 0 & \frac{1}{16} & \frac{1}{8} & \frac{5}{32} & \frac{41}{128} & 
% \frac{77}{256} & \frac{1}{64} & \frac{9}{1024} & \frac{5}{1024}\\
% \hline
% \left[20, 25\right] & 0 & \frac{1}{2} & \frac{1}{4} & \frac{1}{8} & \frac{1}{16} & \frac{1}{32} & \frac{1}{64} & \frac{1}{128} & \frac{1}{256} & \frac{1}{512} & \frac{1}{1024}
% \end{tblr}
% \end{sidewaystable}





\begin{sidewaystable}[p]
\begin{tblr}{
    width=\linewidth,
    hspan=minimal,
    cells={font=\footnotesize},
    cell{1}{1-11}={halign=c},
    column{odd}={gray9},
    %colsep=1pt,
    colspec={
    c |
    S[round-mode=places ,round-precision=2, output-exponent-marker=E, table-format=1.2e+1]
    S[round-mode=places ,round-precision=2, output-exponent-marker=E, table-format=1.2e+1]
    S[round-mode=places ,round-precision=2, output-exponent-marker=E, table-format=1.2e+1]
    S[round-mode=places ,round-precision=2, output-exponent-marker=E, table-format=1.2e+1]
    S[round-mode=places ,round-precision=2, output-exponent-marker=E, table-format=1.2e+1]
    S[round-mode=places ,round-precision=2, output-exponent-marker=E, table-format=1.2e+1]
    S[round-mode=places ,round-precision=2, output-exponent-marker=E, table-format=1.2e+1]
    S[round-mode=places ,round-precision=2, output-exponent-marker=E, table-format=1.2e+1]
    S[round-mode=places ,round-precision=2, output-exponent-marker=E, table-format=1.2e+1]
    S[round-mode=places ,round-precision=2, output-exponent-marker=E, table-format=1.2e+1]
    S[round-mode=places ,round-precision=2, output-exponent-marker=E, table-format=1.2e+1]
    },
    row{1}={guard},
    column{2-13}={mode=math},
    column{1}={guard, mode=math}
}
 \diagbox{Point}{Depth} & 1 & 2 & 3 & 4 & 5 & 6 & 7 & 8 & 9 & 10 & >10 \\
 \hline
\left[25, 100\right] & 0 & 0 & 1.250e-1 & 1.880e-1 & 1.880e-1 & 1.560e-1 & 1.170e-1 & 8.200e-2 & 5.470e-2 & 3.520e-2 & 4.07e-02 \\
\left[20, 90\right] & 0 & 0 & 0 & 6.250e-2 & 1.250e-1 & 1.560e-1 & 3.200e-1 & 3.010e-1 & 1.560e-2 & 8.790e-3 & 6.23e-03 \\
\left[30, 90\right] & 0 & 0 & 0 & 0 & 3.130e-2 & 7.810e-2 & 2.810e-1 & 3.010e-1 & 2.890e-1 & 8.790e-3 & 5.93e-03 \\
\left[35, 85\right] & 0 & 0 & 0 & 6.250e-2 & 1.250e-1 & 1.560e-1 & 1.560e-1 & 1.370e-1 & 1.090e-1 & 8.200e-2 & 1.14e-01 \\
\left[25, 85\right] & 0 & 0 & 0 & 0 & 0 & 0 & 0 & 2.190e-1 & 4.380e-1 & 1.170e-1 & 1.45e-01 \\
\left[15, 85\right] & 0 & 0 & 0 & 6.250e-2 & 1.250e-1 & 1.560e-1 & 3.200e-1 & 3.010e-1 & 1.560e-2 & 8.790e-3 & 6.23e-03\\
\hline
\left[20, 25\right] & 0 & 5.000e-1 & 2.500e-1 & 1.250e-1 & 6.250e-2 & 3.130e-2 & 1.560e-2 & 7.810e-3 & 3.910e-3 & 1.950e-3 & 9.53e-04\\
\end{tblr}
\caption{Probabilities for individual data points, enhanced approach.}
\label{table_big_novelty}
\end{sidewaystable}

\subsection{Expected value of depth}
The expected values of depths (EXD) for the enhanced isolation forest algorithm are shown in the section above.
To compare the values with the expected values of depths for the original isolation forest, we sum the rows in Table \ref{table_big_original} multiplied by respective depth values. Table \ref{table_ex_comparison} shows the side-by-side comparison. The important outcome is the ratio of expected depths in the respective column. As seen in this table, the difference of depth for the novelty point in the case of the original approach is relatively small compared to the difference of enhanced novelty isolation forest.
This example proved that it is feasible for the enhanced algorithm to encapsulate the previously unseen data points in the higher leaves of the tree, making novelty detection possible.
Note that a representative example of one possible distribution of data in the hyperplane was chosen for this proof. However, it shares most of the attributes with general novelty detection problems.

\begin{table}[h]
\centering
\begin{tblr}{
    width=\linewidth,
    cells={font=\footnotesize},
    colspec={c | 
    S[table-format=1.3, round-mode=places ,round-precision=3] 
    S[table-format=1.3]},
    %colsep=1pt,
    row{1}={guard},
    column{1-3}={mode=math}
}
point & EXD outlier & EXD novelty \\
\hline
25,100 & 3.32616229 & 6\\
20,90 & 4.26063731 & 6.82\\
30,90 & 4.34883099 & 7.82\\
35,85& 3.95906409 & 8\\
25,85 & 4.87576598 & 9.734\\
15,85 & 3.76796497 & 6.82\\
\hline
20,25 & 4.27789495 & 3\\

\hline
\end{tblr}
\caption{Expected values of depths for both algorithms.}
\label{table_ex_comparison}
\end{table}

\section{Discussion and Conclusions}
\label{sec:conclusion}

% todo: tady vic rozepsat co vidi ctenar v tech tabulkach

% pridej jeste neco jako ukazes co vlastne jsi dokazal

% pridej nejake downnsides

% Experiments in this paper fully demonstrate that using the proposed WSS 2 (and hence
% WSS 3) leads to faster convergence (i.e., fewer iterations), 

% on the other hand...

In Section \ref{sec:InitialProblem} we presented the initial problem of novelty detection and highlighted the exact areas where the original Isolation Forest can be enhanced to be successfully used in novelty detection scenarios.
Later in this section, we proposed our solution based on several alterations to the original solution, mainly on the range-based training of a tree.
Through the examples in Section \ref{example:novelty_tree_create}, we show how to build such an enhanced tree using a descriptive dataset.
Experiments in this paper fully demonstrate that using our proposed enhancement, it is possible to successfully isolate previously unseen novelty data points, with their depth being reasonably different from the regular trained-on data points.
This assumption is later solidified with proofs that show the theoretical outcomes if all possible solutions were assessed. Namely, Table \ref{table_big_novelty} shows that the expected depth of the novelty point when evaluated using the enhanced approach differs much more significantly from other points as opposed to the original approach seen in Table \ref{table_big_original}.

In summary, the research presented herein introduces a novel algorithm that has been demonstrated to effectively handle specific novelty points, as evidenced through a detailed example. Notably, comparisons between the depths of the original algorithm and the new one indicate significant modifications at these novelty points, suggesting enhanced performance in these areas. Such findings not only affirm the practical utility of the new algorithm but also highlight its potential adaptability to various situations within the algorithmic framework. However, this method caused the Isolation Forest to no longer detect outliers as the algorithm now works in offline mode. Also, there is a need to choose the proper starting range and other hyperparameters. This is why future work will focus on establishing optimal range settings at the onset, addressing unique scenarios within the algorithmic structure, and validating these enhancements through comprehensive benchmarks. This proposed trajectory aims to further substantiate the robustness and efficiency of the new algorithm in diverse operational contexts.




%% If you have bibdatabase file and want bibtex to generate the
%% bibitems, please use
%%
\bibliographystyle{bibliography/model1-num-names} 
\bibliography{bibliography/bibli}

%% else use the following coding to input the bibitems directly in the
%% TeX file.

% \begin{thebibliography}{00}

% %% \bibitem[Author(year)]{label}
% %% Text of bibliographic item

% \bibitem[ ()]{}

% \end{thebibliography}
\end{document}

\endinput
%%
%% End of file `elsarticle-template-num-names.tex'.
