\section{Discussion and Conclusions}
\label{sec:conclusion}

% todo: tady vic rozepsat co vidi ctenar v tech tabulkach
% pak jsme dokazali na konkretnim prikladu ze opravdu funguje pro novelty bod
% kdyz se podivame na rozdily v deptech originalu a toho naseho tak vidime ze hlavne pro ten novelty bod jsou tam velke zmeny v hloubce, podobne to funguje pro jakekoli jine novelty body


The research presented herein introduces a novel algorithm which has been demonstrated to effectively handle specific novelty points as evidenced through a detailed example. Notably, comparisons between the depths of the original algorithm and the new one indicate significant modifications at these novelty points, suggesting enhanced performance in these areas. Such findings not only affirm the practical utility of the new algorithm but also highlight its potential adaptability to various situations within the algorithmic framework. Future work will focus on establishing optimal range settings at the onset, addressing unique scenarios within the algorithmic structure, and validating these enhancements through comprehensive benchmarks. This proposed trajectory aims to further substantiate the robustness and efficiency of the new algorithm in diverse operational contexts.
