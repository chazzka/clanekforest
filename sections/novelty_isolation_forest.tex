\section{Proposed Range-based Enhancement For Isolation Forest}
\label{sec:novelty_isolation_forest}
In this section, we propose a new enhancement of the original Isolation Forest algorithm to make it possible to detect novelty observations.
The proposed enhancement takes the basic idea of an ensemble of trees with various depths but takes it further to make semi-supervised detection possible.

\subsection{Initial Problem}
\label{sec:InitialProblem}
 The standard Isolation Forest algorithm cannot be used for novelty detection. This is because, in each step, the observation space is limited by the previous data.


Suppose we now want to use the original tree to evaluate the novelty data point $p$, which is not present in the training set.

Let us recall the equations needed for vertex creation:
\begin{align*}
S_l &= \{ \mathsf{s} \in{R \cap S}\ |\ \pi_{d_R}(\mathsf{s})\le z_R\},&
S_r &= \{ \mathsf{s} \in{R \cap S}\ |\ \pi_{d_R}(\mathsf{s}) > z_R\}.
\end{align*}
The point $p$ fits neither $S_l$ nor $S_r$ and does not even necessarily fit $R_l = R(S_l)$ nor $R_r = R(S_r)$. 

Nevertheless, because the data point $p$ satisfies  $\pi_d(p) \le z$ (or $\pi_d(p) > z$), $p$ is assigned to the vertex $R_l$ (or $R_r$) even though $p \notin R_l$ nor $p \notin R_r$.

This is why the original Isolation Forest is a purely unsupervised algorithm. To work properly, the tree has to be constructed concerning all possible input data.

\subsection{Proposed Solution}
The proposed solution comes from the idea that the original tree lacks the possibility to isolate more data points than it currently observes.
The observed space is bounded by the minimum and maximum in each dimension.

As in the original article, we use the concept of a binary decision tree. The proposed solution is altering the idea of the split point evaluation. Whereas the original Isolation Forest evaluates the split point based on the previous data, we
evaluate the split point based on a range in our proposed solution. For this to work, several alterations to the split point evaluation and the form of data passed between vertices must be made; however, the overall concept of the forest remains the same.
The proposed solution has several concepts of alteration to the original solution:

\begin{enumerate}
    \item We start with the root as the whole domain space bounded by ranges.
    \item Descendants cover the whole observable space of their associated ancestors. 
    \item The split point is in the middle of the given dimension’s range.
    \item The input data is only used to determine the ending condition.
\end{enumerate}

\subsection{Constructing the decision tree $T$}

\begin{enumerate}
    % \item The maximum possible depth of a tree is controlled by the \emph{max\_depth} parameter.
    \item The sample \(S\) is the nonempty set of input points.
    \item Leaves and internal vertices are possibility-space hyperrectangles. 
    \item A hyperrectangle $R$ satisfies the ending condition when \(S \cap R = \emptyset\) or \(| S \cap R | = 1\).
\end{enumerate}



\paragraph{Basis step}
Each dimension \(d \in\{0, \dots, n-1\}\), is bounded by the range \(r_d\). The ranges form the possibility-space hyperrectangle \(R_0\), i.e.
\[R_0 =  r_0 \times r_1 \times \cdots \times r_{n-1}.\]

The trivial rooted tree \(T_0\) is a tuple with
vertices \(V_0 = \{R_0\}\) and edges \(E_0 = \emptyset\), i.e.
\[T_0= (V_0, E_0) = (\{R_0\},\emptyset).\]

\paragraph{Recursive step}
The steps to reach the tree \(T_{j+1}\) from \(T_{j} = (V_j, E_j)\) are
as follows:

Let \(L_j \subseteq V_j\) be a subset of leaves not satisfying the
ending condition.
For each leaf \(R \in L_j\) we
select a random dimension \(d\).

Let
\begin{align*}
R &= r_0 \times  \cdots \times r_{d-1} \times  r_d\times r_{d+1} \times \cdots \times r_{n-1},
\end{align*}
where $r_d = \langle x, y )$.
Then, we obtain the left and right hyperrectangles
\begin{align*}
R_l &= r_1 \times  \cdots \times r_{d-1} \times  \langle x, s ) \times r_{d+1} \times \cdots \times r_n, \\
R_r &= r_1 \times  \cdots \times r_{d-1} \times \langle s, y ) \times r_{d+1} \times \cdots \times r_n,
\end{align*}
where 
\(s = \frac{x + y}{2}\)
is a number obtained as the middle of the range \(r_d\,\).

In the new tree $T_{j+1}$ each vertex \(R\) is associated with two new
edges \((R,R_l ), (R, R_r)\) giving
\begin{align*}
V_{j+1} &= V_j \cup \bigcup_{R \in L_j} \{R_l, R_r\},\\
E_{j+1} &= E_j \cup \bigcup_{R \in L_j} \{(R, R_l), (R,R_r)\},\\
T_{j+1} &= (V_{j+1}, E_{j+1}),
\end{align*}
i.e.~${R_l, R_r} \subset R$ are leaves in the new tree
\(T_{j+1}\).

\paragraph{Recursion is terminated} if there is an equality of two consecutive trees \(T_k = T_{k+1}\). This happens when all leaves satisfy the ending condition, i.e., \(L_k = \emptyset\).
If this is the case, the desired tree $T$ is the tree $T_k$; otherwise, move to the next recursion step.


Note that the tree \(T_{j}\) is actually a Hasse diagram of the ordered set
\((V_j,\subseteq)\).


\begin{example}
\label{example:novelty_tree_create}
Consider now an example of creating a new enhanced Isolation tree based on the given input sample
\begin{align*}
    S = \{&[25,100],[30,90],[20,90],[35,85],\\
    &[25,85],[15,85],[105,20],[95,25], \\
    &[95,15],[90,30],[90,20],[90,10]\}.
\end{align*}

Figure \ref{fig:example_novelty_gnu} shows $S$ in the finished plane created using an enhanced approach. Observe that the split points (red lines) are always in the middle of the previous observable space, and rectangles now always cover the whole ancestor's area. Numbers represent the final leaves' depth (also seen in Figure \ref{fig:example_novelty_tree_color}).

\begin{figure}[htbp]
\centering
\includesvg[width=0.9\textwidth,inkscapelatex=false]{figures/example66_Novelty_gnu.svg}
\caption{Enhanced approach. Rectangles created by recursive splitting.}
\label{fig:example_novelty_gnu}
\end{figure}

The tree $T$ is created by starting with the tree $T_0$ and expanding further as described by the recursive step until the ending condition is met.
Figure \ref{fig:example_novelty_tree_color} shows the finished tree $T$, trained on the dataset $S$.

\paragraph{Basis step} 
The trivial step is to create the tree $T_0$ with one root vertex $R$ with the experimentally set initial possibility space  $R= \langle 0,110) \times \langle -5,105)$ and no edges $E_0$, such that
    \begin{align*}
        V_0 &= \{R\},&
        E_0 &= \emptyset,&
        T_0 &= (V_0, E_0).
    \end{align*}

\paragraph{Recursive step}
     Since $R \in L_0$, we create two rectangles $R_l$, $R_r$ by selecting a random dimension $d=0$.
    \begin{align*}
        r_0 &= \langle 0, 110), &
        s &= \frac{0 + 110}{2} = 55, \\
        R_l &= \langle 0, 55) \times \langle -5,105), &
        R_r &= \langle 55, 110) \times \langle -5,105).
    \end{align*}
     New tree $T_1$ is then
    \begin{align*}
    V_1 &= \{R, R_l, R_r\}, &
    E_1 &= \{(R, R_l), (R, R_r)\}, &
    T_1 &= (V_1, E_1).
    \end{align*}
    After checking for the ending condition, set $L_1$ contains precisely two elements $R_l$ and $R_r$.
     
    Since there are vertices in $L_1$ left to be examined, we continue with recursive steps.
    Since $R_l \in L_1$ (resp. $R_r \in L_1$), we create two new rectangles $R_{ll}$, $R_{lr}$ -- left column (resp. $R_{rl}$, $R_{rr}$ -- right column) by selecting a random dimension $d=1$ (resp. $d=0$).
    \begin{align*}
        r_{l1} &= \langle -5, 105)& r_{r0} &= \langle 55, 110) \\
        s_l &= 50 & s_r&=82.5\\
        R_{ll} &= \langle 0, 55) \times \langle -5,50) & R_{rl} &= \langle 55, 82.5) \times \langle -5,105)\\
        R_{lr} &= \langle 0, 55) \times \langle 50,105) & R_{ll} &= \langle 82.5, 110) \times \langle -5,105)
    \end{align*}
 New tree $T_2$ is then
    \begin{align*}
        V_2 &= \{R, R_l, R_r, R_{ll}, R_{lr}, R_{rl}, R_{rr}\} \\
        E_2 &= \{(R, R_l), (R, R_r), (R_l, R_{ll}), (R_l, R_{lr}), (R_r, R_{rl}), (R_r, R_{rr})\} \\
        T_2 &= (V_2, E_2)
    \end{align*}
We recheck the ending condition. The set of leaves $L_2$ now contains two hyperrectangles $R_{lr}$ and $R_{rr}$.
This goes on until the ending condition is met. The resulting tree $T_8$ is depicted in Figure \ref{fig:example_novelty_tree_color}.
\end{example}

\subsection{Evaluating enhanced decision tree $T$}
The evaluation of our enhanced decision tree is more straightforward since the examined data point is always contained in the possibility space hyperrectangle of some vertex in each depth until a leaf is visited.
The evaluation starts in the root vertex. The initial possibility space should be reasonable enough to cover the whole domain of a given problem. Until the leaf is reached, the examined point recursively visits the descendant within which it is contained.



\begin{example}
\label{ex:regular_point_evaluation_novelty}
    Consider now the evaluation of $a = [105,20]$ on the tree $T$ built in Example \ref{example:novelty_tree_create}.

\begin{enumerate}
    \item  We start with the root $R = \langle 0,110\rangle \times \langle -5, 105 \rangle$.
    The root $R$ has two descendants 
\begin{align*}
    &R_l = \langle 0,55) \times \langle -5, 105),&
    &R_r = \langle 55,110) \times \langle -5, 105),
\end{align*}
Since $a \in R_r$, we visit $R_r$.
\item Vertex $R_r$ has two descendants
\begin{align*}
    &R_{rl} = \langle 5,82.5) \times \langle -5, 105),&
    &R_{rr} = \langle 82.5,110) \times \langle -5, 105),
\end{align*}
Since $a \in R_{rr}$, we visit the vertex $R_{rr}$.
\item
we continue recursively for another two steps until the leaf $R_{rrlr}$ is reached. This leaf has a depth of $4$.

\end{enumerate}
   
\end{example}

\begin{example}
\label{ex:novelty_point_evaluation_novelty}
    Consider now the evaluation of $a' = [25,20]$ on the tree $T$ built in Example \ref{example:novelty_tree_create}.

\begin{enumerate}
    \item  We start with the root $R = \langle 0,110\rangle \times \langle -5, 105 \rangle$.
    The root $R$ has two descendants 
\begin{align*}
    &R_l = \langle 0,55) \times \langle -5, 105),&
    &R_r = \langle 55,110) \times \langle -5, 105),
\end{align*}
Since $a' \in R_l$, we visit $R_l$.
\item Vertex $R_l$ has two descendants
\begin{align*}
    &R_{ll} = \langle 0,55) \times \langle -5, 50),&
    &R_{lr} = \langle 0,5) \times \langle 50, 105),
\end{align*}
Since $a' \in R_{ll}$, we visit the vertex $R_{ll}$.
\item
Since $R_{ll}$ is a leaf, we end here, and the reached leaf's depth is $2$.
\end{enumerate}
\end{example}
Note that each evaluated point of the given possibility space is always contained in each vertex it visits.
Hence $a \in R_{rrlr} \subset R_{rrl} \subset R_{rr} \subset R_{r} \subset R$ in Example \ref{ex:regular_point_evaluation_novelty}
and $a' \in R_{ll} \subset R_{l} \subset R$ as shown in Example \ref{ex:novelty_point_evaluation_novelty}.


%z tadyka jsem smazal puvodni evaluaci, je v evaluate_old.tex



\begin{figure}[htbp]
\centering
\includesvg[angle=90,inkscapelatex=false,width=1\textwidth]{figures/example66_Novelty_tree.svg}
\caption{Tree constructed using the enhanced novelty approach}
\label{fig:example_novelty_tree_color}
\end{figure}


