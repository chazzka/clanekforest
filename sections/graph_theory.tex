\section{Graph Theory}
\label{sec:graph_theory}

The theoretical framework used in this article is centered mainly around the graph theory.
The definitions used in this section are based on Rosen et.al in \cite{rosen2012discrete}. Since the isolation forest is composed of trees, we define a tree as a directed graph.

\begin{definition}
Directed graph (or digraph) $D = (V, E)$ consists of a nonempty set of vertices $V$ and a set of directed edges $E \in V \times V$. Each directed edge is an ordered pair of vertices.
The directed edge $(u, v)$ is said to start at $u$ and end at $v$.
\end{definition}


\begin{definition}
For digraph $D = (V,E)$ we define:
\begin{itemize}
    \item Descendant of vertex $v \in V$ is a vertex $v' \in V$ such that $(v,v') \in E$. We say that $v$ is the ancestor \emph{associated with} $v'$.
    \item Path $P$ from vertex $v_0 \in V$ to vertex $v_n \in V$ is finite set of edges $P \subseteq E$, such that: $$P = \{(v_0, v_1),(v_1, v_2),\dots(v_{n-1}, v_n)\}.$$
      The empty set $P = \emptyset$ is a trivial path from $v$ to $v$ for every $v \in V$.
    
\end{itemize}
\end{definition}

With the isolation tree starting at the root, we define a rooted tree.

\begin{definition}
Rooted tree is a digraph $T = (V,E)$ with a single vertex $r \in V$ designated as the root, with exactly one path from $r$ to $v$ for each $v \in V$.
\end{definition}

To be able to evaluate the isolation tree, we have to define several tree-specific attributes.

\begin{definition}
For rooted tree $T = (V,E)$ we define:
\begin{itemize}
    \item depth of $v \in V$ is a size of path $P$ from root to $v$.
    \item vertex $v \in V$ is called a leaf if it has no descendants. Vertices that have descendants are called internal vertices.
    \item to traverse tree $T$ is to take path $P$ from root to leaf. If $(v,v') \in P$, we say that we \emph{visit} $v$ and $v'$ during traversing.
\end{itemize}
\end{definition}
