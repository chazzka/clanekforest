
\section{Theory}
\label{sec:theory}
Since this article is centered around the anomaly detection algorithm topics, several terms that will be seen across the rest of the article have to be introduced.
\begin{description}
    \item[Datapoint] Datapoint is any observable data with \(n\) dimensions.
    \item[Regular datapoint] Regular is a datapoint included in the given dataset. Its features are expected.
    \item[Anomaly] Anomaly is a datapoint that differs significantly from other observations.
    \item[Outlier] Outlier is an anomaly included in the given dataset.
    \item[Novelty] Novelty is an anomaly that is not present in the given dataset during
learning. Novelties are usually supplied later during evaluation.
    \item[Supervised algorithm] Supervised algorithm is being trained on all of the available data labels.
    \item[Unsupervised algorithm] Unsupervised algorithm usually does not get any labels for the training data.
    \item[Semi-supervised algorithm] Semi-supervised algorithm is usually trained on data of only one class.
\end{description}