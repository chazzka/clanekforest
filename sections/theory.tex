\section{Glossary}
\label{sec:theory}

Since this article is centered around anomaly detection algorithms, several key terms used throughout the article are introduced below.

\begin{itemize}
    \item \emph{Data point} refers to any observable data with \(n\) dimensions.
\item Regular point is a data point included in the given dataset. Its features are expected.
\item \emph{Anomaly} is a data point that differs significantly from other observations.
\item \emph{Outlier} is an anomaly included in the given dataset.
\item \emph{Novelty} is an anomaly that is not present in the given dataset during learning. Novelties are supplied later during evaluation.
\item \emph{Supervised} algorithm is an algorithm that is trained on all available data labels.
\item \emph{Unsupervised} algorithm is an algorithm that does not get any labels for the training data.
\item \emph{Semi-supervised} algorithm is an algorithm that is trained on data of only one class.

\end{itemize}