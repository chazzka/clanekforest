\section{Examples}
\label{sec:examples}

This section provides minified examples to understand a given problem better. The same example is shown in the original and enhanced novelty approaches, respectively.

\subsection{Given problem}

Consider a minified example depicted in \ref{fig:example_data}. In this example, two chunks of data are in the top left and bottom right corners, respectively. Then, we selected a specific data point that shares one dimension similar to the first chunk and the other with the second one.

Note that the data have no specific distribution. The point in the left bottom has a value of $P_x = [5,5]$ and is not present for the learning phase.

\begin{figure}[htbp]
\centering
\includesvg[width=0.9\textwidth]{figures/example6_experiment.svg}
\caption{Example figure}
\label{fig:example_data}
\end{figure}

\paragraph{Solution: Original approach}
With the original approach being fully unsupervised, we feed the whole input (excluding the $P_x$) to the forest and examine the resulting tree.

First, the trivial binary tree $T_0$ is created (xx), with ranges being the min-max values of input.
By doing the recursive steps, the whole tree is constructed.
Figure \ref{fig:example_noutlier_tree_color} shows the constructed binary tree based on the input data.

The evaluation of $P_x$ is as follows:
\begin{itemize}
    \item In the first step, after randomly selecting the dimension $d=1$ and splitpoint s as in (x) $s = 64.37$, the point fits the right node $\langle 3.0 ) \times \langle 7.0)$
    \item In the second step, the dimension $d=1$ and splitpoint $s = 4.35$ are randomly chosen. The point fits the left node $\langle 5.0 ) \times \langle 7.0)$.
    \item This continues until the final leaf (marked grey in Figure \ref{fig:example_noutlier_tree_color}) is reached. This leaf has a depth of 4.
    
\end{itemize}

% todo: tady vidis ze ta depth neni zas tak velka oproti jinym

The path length value is obtained as in (xx).
$$h_T(P_x) = 3 + c(4) = 3+ 2 \cdot (H_3 - \frac{3}{4}) = 3 +2 \cdot [\bigl(1 + \frac{1}{2} + \frac{1}{3}\bigr) - \frac{3}{4}] = \frac{31}{6}$$

% todo: vypocet anomaly skore? (0.705)

% todo: protoze je vysledek > 0.6, bod je regular 

\begin{figure}[htbp]
\centering
\includesvg[angle=90,inkscapelatex=false,width=1\textwidth]{figures/example6_Noutlier_tree_colored.svg}
\caption{Example figure}
\label{fig:example_noutlier_tree_color}
\end{figure}



\paragraph{Solution: Novelty approach}

The novelty approach, on the other hand, is a semi-supervised method. We feed the whole input (excluding the $P_x$) to the forest and examine the resulting tree.

First, the trivial binary tree $T_0$ with initial ranges is created (xx).
By doing the recursive steps, the whole tree is constructed.
Figure \ref{fig:example_novelty_tree_color} shows the constructed binary tree based on the input data.

The evaluation of $P_x$ is as follows:
\begin{itemize}
    \item In the first step, after randomly selecting the dimension $d=1$ and splitpoint s as in (x) $s = 54$, the point fits the right node $\langle 3.0 ) \times \langle 54.0)$
    \item In the second step, the dimension $d=0$ and splitpoint $s = 28.5$ are randomly chosen. The point fits the left node $\langle 5.0 ) \times \langle 7.0)$.
    \item This continues until the final leaf (marked grey in Figure \ref{fig:example_novelty_tree_color}) is reached. This leaf has a depth of 3.
    
\end{itemize}

% todo: tady vidis ze ta depth je minimalni oproti jinym

The path length value is obtained as in (xx).
$$h_T(P_x) = 3 + c(4) = 3+ 2 \cdot (H_3 - \frac{3}{4}) = 3 +2 \cdot [\bigl(1 + \frac{1}{2} + \frac{1}{3}\bigr) - \frac{3}{4}] = \frac{31}{6}$$

% todo: vypocet anomaly skore? (0.705)

% todo: protoze je vysledek > 0.6, bod je regular 

\begin{figure}[htbp]
\centering
\includesvg[angle=90,inkscapelatex=false,width=1\textwidth]{figures/example6_Novelty_tree_colored.svg}
\caption{Example figure}
\label{fig:example_novelty_tree_color}
\end{figure}
