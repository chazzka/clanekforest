\section{Isolation Forest}
\label{sec:isolation_forest}

Isolation Forest
\cite{liu2008isolation} \cite{liu2012isolation} is an outlier
detection, unsupervised ensemble algorithm. This approach is well-known for successfully identifying outliers using recursive partitioning (forming a decision tree) to decide whether the analyzed datapoint is an anomaly. The fewer partitions that are required to isolate, the more probable it is for a datapoint to be an anomaly.

% \paragraph{Isolation tree}
% Isolation Tree is a rooted tree constructed with a subset of \(A\) items
% (datapoints) with the size \(s=|A|\).

% \begin{enumerate}
%     \item To build an isolation tree, it is not necessary to have a large set; it
%   may even be undesirable
%   \item Well-chosen small \(s\) can help eliminate \emph{masking} and
%   \emph{swamping}


% \begin{itemize}
%     \item \textbf{Masking} When the number of anomalies is high, it is possible that some of those aggregate in a dense and large cluster, making it more difficult to separate the single anomalies and, in turn, to detect such points as
%   anomalous.
%     \item \textbf{Swamping} When normal instances are too close to anomalies, the number of
%   partitions required to separate them increases, making it more difficult for the Isolation Forest to discriminate between
%   anomalies and normal points.
% \end{itemize}

% \item There are two types of vertices

% \begin{itemize}
%     \item \textbf{Internal vertex} Internal vertex contains a condition (a feature and a limit) and two children (one
%   representing a fulfilled condition and the other an unfulfilled one)
%     \item \textbf{External vertex} External vertex is created if the conditions of the parents are met (or not met) by
%   one or none of the elements from the sample or the maximum depth of
%   the tree \(l\) is reached, usually \(l=\ln_2(s)\). It contains the
%   evaluation of \(h(x)\) using the distance from the root. If the max
%   length of the tree is reached the \emph{distance} is estimated using
%   \(h(x)=e+c(n)\), where \(e\) is the distance from the root, \(n\) is
%   the number of elements from the sample satisfying the conditions of
%   the parents, \(c(n)=2\,(H_{n-1}-\frac{n-1}{n})\) and \(H_{n-1}\) is
%   \(n-1\) harmonic number.
% \end{itemize}

% \end{enumerate}

\begin{definition}
Let $\mathsf{s}=(s_0, \dots, s_d, \dots, s_{n-1})\in \mathbb{R}^n$ be a datapoint. Then we say that $\pi_d(\mathsf{s}):=s_d$ is a \emph{projection} of datapoint $\mathsf{s}$ onto dimension $d$  yielding $s_d$.
\end{definition}

\begin{definition}
Let $Z$ be a finite subset of $\mathbb{R}^n$:
$$Z \subseteq \mathbb{R}^n ;\quad n \in \mathbb{N}.$$

For each dimension \(d \in\{0, \dots, n - 1\}\), let
$$Z_d = \{ \pi_d(\mathsf{s})\ |\ \mathsf{s} \in Z \}.$$

Then we define hyperrectangle $R(Z)$ \emph{surrounding} $Z$, such that
$$R(Z) = r_0 \times r_1 \times \cdots \times r_{n-1},$$ where $r_d = \langle \min Z_d, \max Z_d \rangle$ for each $d \in \{0,1, \dots, n-1\}.$

\end{definition}


\subsection{Constructing decision tree $T$}

\begin{enumerate}

    \item Each leaf or internal vertex is a hyperrectangle $R$. 
    \item Function relation $\rho$ assigns split point $z$ and dimension $d$ to internal vertex $v$ .
    $$(v, (z,d)) \in \rho$$
    % \item The maximum possible height of a tree is controlled by the \emph{max\_depth} parameter.
    \item The sample \(S\) contains \emph{b} number of input datapoints of $n$ dimensions, that is
    $$S \subseteq \mathbb{R}^n ;\quad |S| = b; \quad b \in \mathbb{N}.$$
    \item The ending condition of the leaves is satisfied if vertex \(R\) satisfies \(| S \cap R | = 1\).
%     \item Leaves' ending condition is satisfied by one of two criteria: 
% \begin{enumerate}
%     \item Vertex \(R\) satisfies \(| S \cap R | = 1\).
%     \item Vertex's depth reached the \emph{max\_depth} value.
% \end{enumerate}
\end{enumerate}


\subparagraph{Basis step}

The trivial rooted tree \(T_0\) is a tuple with
vertices \(V_0 = \{R(S)\}\) and edges \(E_0 = \emptyset\), i.e. 
\[T_0= (V_0, E_0) = (\{R(S)\},\emptyset).\]

The function relation $\rho_0$ initially has no assignments, i.e.
$$\rho_0 = \emptyset.$$

\subparagraph{Recursive step}

% The steps to reach the tree \(T_{j+1}\) from \(T_{j} = (V_j, E_j)\) and $\rho_{j+1}$ from $\rho_{j}$ are
% as follows:
% Let us create \(T_{j+1}\) from \(T_{j} = (V_j, E_j)\) and $\rho_{j+1}$ from $\rho_{j}$.
% The creation of \(T_{j+1}\) from \(T_{j} = (V_j, E_j)\) and $\rho_{j+1}$ from $\rho_{j}$ starts by forming a new hyperrcentagle R.
% to reach the tree \(T_{j+1}\) from \(T_{j} = (V_j, E_j)\) and $\rho_{j+1}$ from $\rho_{j}$
is to reach the tree \(T_{j+1}\) from \(T_{j} = (V_j, E_j)\) and $\rho_{j+1}$ from $\rho_{j}$.

Let \(L_j \subseteq V_j\) be a subset of leaves not satisfying the
ending condition
% By selecting a random dimension $d_R$ (such that $|r_{d_R}| > 1$) create two new vertices \(R_l, R_r\) for each leaf \(R \in L_j\)\
% \[R =  r_0 \times \cdots \times r_{d_R} \times \cdots \times r_{n-1} \] with a random split point $z_R \in r_{d_R}$.
and \(R \in L_j\)\ be a leaf such that
\[R =  r_0 \times \cdots \times r_{d_R} \times \cdots \times r_{n-1}. \]
We first select a random dimension $d_R$ (such that $|r_{d_R}| > 1$) and a random split point $z_R \in r_{d_R}$.


Split point $z_R$ splits the $R \cap S$, creating two disjunctive sets, $S_l$ and $S_r$ respectively, such that
\begin{align*}
S_l &= \{ \mathsf{s} \in{R \cap S}\ |\ \pi_{d_R}(\mathsf{s})\le z_R\},&
S_r &= \{ \mathsf{s} \in{R \cap S}\ |\ \pi_{d_R}(\mathsf{s}) > z_R\}.
\end{align*}

Then we obtain left and right hyperrectangles \(R_l\), \(R_r\) as
follows:
\begin{align*}
R_l &= R(S_l),&
R_r &= R(S_r).
\end{align*}

Each vertex \(R \in L_j\) is associated with two new
edges \((R,R_l ), (R, R_r)\) and is assigned with $(z_R,d_R)$ by function relation $\rho_{j+1}$ as


\begin{align}
   \rho_{j+1} &= \rho_j \cup \bigcup_{R \in L_j} \{(R, (z_R, d_R))\}, \\
   V_{j+1} &= V_j \cup \bigcup_{R \in L_j} \{R_l, R_r\}, \\
   E_{j+1} &= E_j \cup \bigcup_{R \in L_j} \{(R, R_l), (R,R_r)\},\\
   T_{j+1} &= (V_{j+1}, E_{j+1})
\end{align}
i.e.~${R_l, R_r} \subset R$ are leaves and $R$ is an inner vertex in the new tree
\(T_{j+1}\).\footnote{Tree \(T_{j+1}\) is actually a Hasse diagram of the ordered set
\((V_{j+1},\subseteq)\)}


The algorithm moves to the next recursion step unless there is an equality of two consecutive trees \(T_k = T_{k+1}\). This happens when all leaves satisfy their ending condition, i.e., \(L_k = \emptyset\).
Thus, the desired tree $T$ is the tree $T_k$, and the finite relation $\rho$ is $\rho_k$.


\begin{example}
\label{example:original_tree_create}
Consider now an example of creating a new Isolation tree based on the given input sample $S$ as shown in Figure \ref{fig:example_noutlier_gnu}.

\begin{align*}
    S = \{&[25,100],[30,90],[20,90],[35,85],\\
    &[25,85],[15,85],[105,20],[95,25], \\
    &[95,15],[90,30],[90,20],[90,10]\}
\end{align*}


\begin{figure}[htbp]
\centering
\includesvg[width=0.9\textwidth, inkscapelatex=false]{figures/example66_Noutlier_gnu.svg}
\caption{Original solution. Rectangles created by recursive splitting.}
\label{fig:example_noutlier_gnu}
\end{figure}

Tree $T$ is created by starting with tree $T_0$ and expanding further as described by the recursive step until the final condition is met.
Figure \ref{fig:example_noutlier_tree_color} shows the finished tree $T$, trained on dataset $S$.

\begin{enumerate}
    \item Basis step is to create a tree $T_0$ and a function relation $\rho_0$. 
    Observe that $E_0$ and $\rho_0$ are empty because there is just a single root node without any connections.
\begin{align*}
R &= R(S) = \langle 15, 105 \rangle \times \langle 10, 105 \rangle\\
V_0 &= \{R\}\\
E_0 &= \emptyset\\
T_0 &= (V_0, E_0)\\
\rho_0 &= \emptyset
\end{align*}

Figure \ref{fig:example_noutlier_gnu} shows $R$ as the largest and the brightest colored rectangle.

\item In order to reach $T_1$ from $T_0$, dimension $d_R=1$ and split point $z_R = 72.63$ were chosen. Figure \ref{fig:example_noutlier_gnu} shows split points as red lines in their respective rectangles. The recursive step is as follows:

$$S_l = \{[105,20],[95,25],[95,15],[90,30],[90,20],[90,10]\}$$
$$S_r = \{[25,100],[30,90],[20,90],[35,85],[25,85],[15,85]\}$$

$$R_l = R(S_l) = \langle 90, 105 \rangle \times \langle 10, 30 \rangle$$
$$R_r = R(S_r) = \langle 15, 35 \rangle \times \langle 85, 100 \rangle$$
\begin{align*}
\rho_1 &= \{ (R, (72.63, 1))\}\\
V_1 &= \{R, R_l, R_r\}\\
E_1 &= \{(R,R_l), (R,R_r)\}\\
T_1 &= (V_1, E_1)
\end{align*}
\item The tree $T_1$ has left leaf $R_l$ and right leaf $R_r$; the ending condition is not met, that is $L_1 = \{R_l,R_r\}$. We continue the recursive step with two vertices.

Left vertex $R_l$:
\begin{align*}
z_{R_l} &= 103.08\\
d_{R_l} &= 0\\
S_{ll} &= \{[95,25],[95,15],[90,30],[90,20],[90,10]\}\\
S_{lr} &= \{[105,20]\}\\
R_{ll} &= \langle 90, 95 \rangle \times \langle 10, 30\rangle\\
R_{lr} &= \langle 105, 105 \rangle \times \langle 20, 20\rangle
\end{align*}

Right vertex $R_r$:
\begin{align*}
z_{R_r}&= 20.32\\
d_{R_r}&= 0\\
S_{rl}&= \{[20,90],[15,85]\}\\
S_{rr}&= \{[25,100],[30,90],[35,85],[25,85]\}\\
R_{rl}&= \langle 15, 20 \rangle \times \langle 85, 90 \rangle\\
R_{rr}&= \langle 25, 35 \rangle \times \langle 85, 100 \rangle
\end{align*}
\item With the hyperrectangles prepared, we can assemble new vertices and edges and create a new $T_2$:
\begin{align*}
\rho_2 &= \{(R,(72.63,1), (R_l, (103.08, 0)), (R_r, (20.32, 0)) \}\\
V_2 &= \{ R, R_l, R_r, R_{lr}, R_{lr}, R_{rl}, R_{rr} \}\\
E_2 &= \{ (R,R_l),(R,R_r), (R_l,R_{ll}), (R_l,R_{lr}), (R_r,R_{rl}), (R_r,R_{rr}) \}\\
T_2 &= (V_2, E_2)
\end{align*}
\item With the $T_2$ created, we now have to check for the ending condition of the leaves. Since $|R_{lr}\cap S|=|S_{lr}| = 1$, the ending condition for leaf $R_{lr}$ is met, and the new set of leaves $L_2$ for the next recursion step is
$$L_2 = \{R_{ll},R_{rl},R_{rr}\}.$$

This continues until we reach tree $T_5$ such that $T_5=T_6$, which is the desired tree $T$ as shown in Figure \ref{fig:example_noutlier_tree_color}. 

\end{enumerate}

\end{example}


\subsection{Evaluating decision tree $T$}
The evaluation of desired element $a$ starts in the root $R$ of the previously built tree $T$.
In a root, by applying function relation, $\rho(R) = (z,d)$, we obtain split point $z$ and dimension $d$.
The root $R$ of a tree $T$ has two descendants $R_l$, $R_r$, such that
$\forall r_l\in R_l; \pi_d(r_l) \le z$ and $\forall r_r\in R_r; \pi_d(r_r)  > z$.

If $\pi_d(a)\le z$, we visit $R_l$, else, that is $\pi_d(a) > z$, we visit $R_r$.
We continue in this manner until we reach the leaf. Figures \ref{fig:example_noutlier_gnu} and \ref{fig:example_noutlier_tree_color} show the final depths of individual leaves. Based on it, we can decide on the level of outlierness. The deeper the evaluated point in the tree, the less anomalous it gets.

\begin{example}
\label{ex:regular_point_evaluation_original}
    Consider now the evaluation of $a = [105,20]$ on tree $T$ built in Example \ref{example:original_tree_create}.

    We start with the root $R = \langle 15,105\rangle \times \langle 10, 100 \rangle$.
    By applying function $\rho(R)$, we obtain split point $z = 72.63$ and dimension $d = 1$.
    Root $R$ has two descendants 
\begin{align*}
    &R_l = \langle 90,105\rangle \times \langle 10, 30 \rangle,&
    &R_r = \langle 15,35\rangle \times \langle 85, 100 \rangle,\\
    \intertext{such that}
    &\forall r_l\in R_l; \pi_1(r_l) \le 72.63,&
    &\forall r_r\in R_r; \pi_1(r_r) > 72.63.
\end{align*}
Now, by applying projection $\pi_1$ on $a$, we obtain $20$, which is less than split point $z = 72.63$, i.e.
$$\pi_1([105,20]) = 20 < 72.63.$$
We visit the vertex $R_l$ because the value obtained by applying the projection on any element of $R_l$ is smaller than $72.63$.

We reached the next recursive step. With vertex $R_l$ visited, we apply function $\rho(R_l)$, obtaining a new split point $z = 103.08$ and a new dimension $d = 0$.
Vertex $R_l$ has two descendants 
\begin{align*}
    &R_{ll} = \langle 90,95\rangle \times \langle 10, 30 \rangle,&
    &R_{lr} = \langle 105,105\rangle \times \langle 20,20 \rangle,\\
    \intertext{such that}
    &\forall r_{ll}\in R_{ll}; \pi_0(r_{ll}) \le 103.08,&
    &\forall r_{lr}\in R_{lr}; \pi_0(r_{lr}) > 103.08.
\end{align*}

We visit the vertex $R_{lr}$ because the value obtained by applying projection $\pi_0$ on any element of $R_{lr}$ and also on $a$ is more than $103.08$.

Since $R_{lr}$ has no more descendants, we reached the final leaf with a depth of~$2$.
\end{example}


\begin{example}
\label{ex:novelty_point_evaluation_original}
    Consider now the evaluation of $a' = [25,20]$ on tree $T$ built in Example \ref{example:original_tree_create}. Note that $a'$ was not contained in the training set for building a tree.

    We start with the root $R = \langle 5,105\rangle \times \langle 10, 100 \rangle$.
    By applying function $\rho(R)$, we obtain split point $z = 72.63$ and dimension $d = 1$.
%     Root $R$ has two descendants 
% \begin{align*}
%     &R_l = \langle 90,105\rangle \times \langle 10, 30 \rangle,&
%     &R_r = \langle 15,35\rangle \times \langle 85, 100 \rangle,\\
%     \intertext{such that}
%     &\forall r_l\in R_l; \pi_1(r_l) \le 72.63,&
%     &\forall r_r\in R_r; \pi_1(r_r) > 72.63.
% \end{align*}
% Now, by applying projection $\pi_1$ on $a'$, we obtain $20$, which is less than split point $z = 72.63$, i.e.
We visit $R_l = \langle 90,105\rangle \times \langle 10, 30 \rangle$ 
because $\pi_1([25,20]) = 20 \le 72.63.$
% We visit the vertex $R_l$ because the value obtained by applying the projection on any element of $R_l$ is less than $72.63$.

We reached the next recursive step. Obtaining a new split point $z = 103.08$ and new dimension $d = 0$, we visit $R_{ll} = \langle 90,95\rangle \times \langle 10, 30 \rangle$, since $\pi_0([25,20]) \le 103.08$.
% Vertex $R_l$ has two descendants 
% \begin{align*}
%     &R_{ll} = \langle 90,95\rangle \times \langle 10, 30 \rangle,&
%     &R_{lr} = \langle 105,105\rangle \times \langle 20,20 \rangle,\\
%     \intertext{such that}
%     &\forall r_{ll}\in R_{ll}; \pi_0(r_{ll}) \le 103.08,&
%     &\forall r_{lr}\in R_{lr}; \pi_0(r_{lr}) > 103.08.
% \end{align*}

% We visit the vertex $R_{ll}$ because the value obtained by applying the projection on any element of $R_{ll}$ is less than $103.08$.

This continues recursively until the leaf $R_{llllr}$ is reached. Note that this is the leaf with the point $[90,20]$. The reached depth is 5, as shown in Figure \ref{fig:example_noutlier_tree_color} (marked grey).


Observe from Figure \ref{fig:example_noutlier_gnu} that after just two steps, the $[25,20]$ is no longer a part of any of the further evaluated rectangles.

% Figure \ref{fig:example_noutlier_gnu} shows that $[25,20]$ is evaluated in a way that it is left in the vertex with the point $[90,20]$, thus being evaluated as such.

\end{example}


Note that each element that was part of the batch during the training --- tree building --- phase is always contained in each vertex it visits. See $a \in R_{lr} \subset R_{l} \subset R$ in Example \ref{ex:regular_point_evaluation_original}.
Note that this is not true for elements unseen during the training phase (such as novelty points).
See $a' \in R$, but $a' \notin R_l$ (and of course $a' \notin R_r$) in Example \ref{ex:novelty_point_evaluation_original}.

\begin{figure}[htbp]
\centering
\includesvg[angle=90,inkscapelatex=false,width=1\textwidth]{figures/example66_Noutlier_tree.svg}
\caption{Tree constructed using the original approach}
\label{fig:example_noutlier_tree_color}
\end{figure}
