\section{Methods - toto se smaže}
\label{sec:methods}
Traditional approaches for anomaly detection consist of either novelty
detection or outlier detection. Novelty detection is an anomaly
detection mechanism where we search for unusual observations 
discovered due to their differences from the training data. Novelty
detection is a semi-supervised anomaly detection technique, whereas
outlier detection uses unsupervised methods. With novelty detection, the
training data is not polluted by anomalous elements, and we are
interested in detecting whether a new observation is an anomaly. In this context, such points are also called novelties. This is a crucial
distinction. The outlier detection is usually presented with data
containing both anomalies and regular observation; it then uses
mathematical models that try to make a distinction between them. The
novelty detection, on the other hand, is usually presented with data with
little to zero anomalies (the proportion of anomalies in the dataset is
called a contamination), and later, when conferred with an anomalous
observation, it makes a decision.

Consider the following example: Figure \ref{fig:example0} contains random datapoints
arranged in a way they form a cluster-like shape. Say this data is our
regular observations.

\begin{figure}[htbp]
\centering
\includesvg[width=0.9\textwidth]{figures/example0_gnu.svg}
\caption{Example figure}
\label{fig:example0}
\end{figure}

When an unsupervised, outlier detection algorithm tries to analyze such
data, it sees the datapoints as a cluster containing both regular and
anomalous observations. Figure X shows the result of evaluating
classical Isolation Forest on such a dataset.

\begin{figure}[htbp]
\centering
\includesvg[width=0.9\textwidth]{figures/example0_5_gnu.svg}
\caption{Example figure}
\label{fig:example05}
\end{figure}

Figure \ref{fig:example05} shows regular observations \(x\) and anomaly observations \(y\)
marked by Isolation Forest
(\texttt{batch\_size\ 128,\ trees\_count:\ 100,\ zbytek\ default}).
Figure x shows that approx. \(10\%\) of observations are anomalies. This
is not unwanted behavior in the sense of outlier detection but is false
positive observation in the sense of novelty detection because this
data are regular observations that should be marked so.

Another problem is with the unsupervised separation itself. Consider data polluted by anomalies in close to \(1:1\) ratio. Finding the line
itself is evident. Deciding which observations are anomalies without some domain knowledge, on the other hand, is close to impossible.