Consider a minified example depicted in \ref{fig:example_data}. In this example, two chunks of data are in the top left and bottom right corners, respectively. Then, we selected a specific data point that shares one dimension similar to the first chunk and the other with the second one.

Note that the data have no specific distribution. The point at the bottom left has a value of $P_x = [5,5]$ and is not present for the learning phase.

\begin{figure}[htbp]
\centering
\includesvg[width=0.9\textwidth]{figures/example6_experiment.svg}
\caption{Example figure}
\label{fig:example_data}
\end{figure}



\paragraph{Solution: Original approach}
With the original approach fully unsupervised, we feed the whole input (excluding the $P_x$) to the forest and examine the resulting tree.

First, the trivial binary tree $T_0$ is created (xx), with ranges being the min-max values of input.
By doing the recursive steps, the whole tree is constructed.
Figure \ref{fig:example_noutlier_tree_color} shows the constructed binary tree based on the input data.

The evaluation of $P_x$ is as follows:
\begin{itemize}
    \item In the first step, after randomly selecting the dimension $d=1$ and split point $z$ as in (x) $z = 64.37$, the $P_x$ visits the node $\langle 102.0, 105.0\rangle \times \langle 3.0, 7.0\rangle$
    Note that it is enough for the point to fit only the selected dimension.
    \item In the second step, the point visits the node $\langle 102.0, 105.0\rangle \times \langle 5.0, 7.0\rangle$.
    \item This continues until the final leaf $\langle 102.0, 105.0\rangle \times \langle 5.0, 5.0\rangle$ marked gray in Figure \ref{fig:example_noutlier_tree_color}) is reached. This leaf has a depth of 4.
    
\end{itemize}


If we look at the constructed tree in Figure \ref{fig:example_noutlier_tree_color}, we can see that the resulting leaf is relatively deep, considering the depth of the deepest leaf.






\paragraph{Solution: Novelty approach}

The novelty approach, on the other hand, is a semi-supervised method. We feed the whole input (excluding the $P_x$) to the forest and examine the resulting tree.

First, the trivial binary tree $T_0$ with initial ranges is created (xx).
By doing the recursive steps, the whole tree is constructed.
Figure \ref{fig:example_novelty_tree_color} shows the constructed binary tree based on the input data.

The evaluation of $P_x$ is as follows:
\begin{itemize}
    \item In the first step, since the root node has dimension $d=1$ and split point $z = 54$ assigned during the training phase, the point visits the node $\langle 3.0, 54.0\rangle \times \langle 3.0, 105.0\rangle$
    \item In the second step, the point visits the node $\langle 3.0, 28.5\rangle \times \langle 3.0, 105.0\rangle$.
    \item This continues until the final leaf (marked grey in Figure \ref{fig:example_novelty_tree_color}) is reached. This leaf has a depth of 3.
    
\end{itemize}